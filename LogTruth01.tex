My aim here is to \emph{define} a notion of \emph{logical truth} as that term is to be used in this monograph.
This may or may not coincide with the idea of logical truth in any other philosophical treatise, it is not my intention to defend this terminology, or to make any claims about it other then those explicit in this monograph.%
%
\endnote{The intended conception of logical truth is, however, close to the concepts of logical truth and of analyticity adopted by Rudolf Carnap during an important stage in his philosophical development, that ending soon after the publication of Quine's ``Two Dogmas of Empiricism''.
After that date both were modified as a result of Quine's critique, without significant change to the substance of Carnap's philosophical beliefs.

His concept of amalyticity remained the same in essence, but the presentation was re-organised to give a single definition of analyticity as a property of sentences in a language with a given semantics, rather than a general prescription of how language specific definitions of analyticity might be presented.

As to Logical Truth, Quine's critique was more incisive, since in his account of this idea in \emph{Menaing and Necessity}\cite{carnap47} Carnap had stayed too close to the narrow conception of logical truth found in Wittgenstein's \emph{Tractatus Logico-Philosophicus1921}\cite{wittgenstein1921}.
His resolution of this problem is found in \emph{Meaning Postulates}\cite{carnap52,carnap56}, which admits into the determination of logical truth the meanings of all concepts in the language rather than just the `logical' by allowing 'meaning postulates' in the semantic definition which capture the sense of the non-logical concepts.

Thenceforth Carnap's conception of logical truth reflected its accidental identification with the narrow conception in Wittgenstein's \emph{Tractatus} and is closer to the notion of first-order validity favoured in the growing discipline of \emph{mathematical logic}.
Though conceding the term `logical truth' in that way, he continued to relate the concept of analyticity to logical truth by speaking instead of `logical truth in the wider sense'.
See, for example `The Philosophy of Rudolf Carnap' \cite{carnap56} III/III/15, p 917.
}

A logical truth will be defined as a particular kind of sentence in a declarative language, which as defined here will involve a collection of syntactic entities and a semantic mapping assigning meaning to those entities.

A notion of semantic reduction will then be defined, whereby the sentences of one declarative language are mapped into the sentences of some other declarative language in a way which, in a suitable sense, preserves meaning.
Such a notion of reduction determines a partial order on the expressiveness of declarative languages.
Whenever language A is reducible to language B, then we may say that language B is at least as expressive as language A.
This gives us a partial ordering of declarative languages in terms of expressiveness, and it is clear from well established insights into semantics that there will be no maximal language in this ordering.

We next consider certain paths in this partial ordering, in which the language syntax is fixed, but the semantics varies in ways which strictly increase the expressiveness of the resulting language, and consider the relationship between such families of declarative languages which is induced by the relationship of semantic reduction.
One such family is generated from the syntax of first order set theory with semantics determined by the sets of models complying with ever stronger large cardinal axioms.
We argue (but cannot prove) that this family is universal for logical truth.

\subsection{Language}

A language\endnote{There are many conceptions of and approaches to the concept of language across and within various academic disciplines.
  The definition offered here is not offered as better than any other, except hopefully, as a vehicle for the epistemological synthesis presented here.
It is, from the perspective of those many others, a narrow conception of language, which is suitable for the proposed \emph{foundational} role, as contributing something important to the understanding of  a wider variety of epistemological phenomena, particular in relation to the possibility of reasoning with or about reprentations of knowledge.}
 is a way of representing information, knowledge or conjecture about some other thing.
The information is represented by entities which we call \emph{syntax} and the relationship between the syntax and those things about which it carries information is called \emph{semantics}.





