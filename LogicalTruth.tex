Absolute certainty in the truth of any proposition may not be possible, but for a certain class of propositions a degree of confidence can be obtained which is more than sufficient for any practical or theoretical purpose, if nevertheless leaving doubt in the minds of the most determined radical sceptics.

The perception of a divide between a domain in which certainty could be realised from a broader compass of more doubtful propositions dates back at least two and a half millenia to the ancient Greek philosophers, who found in theoretical mathematics a domain in which systematic deductive reasoning could consistently deliver reliable truth, but whose application of reason more broadly proved incapable of securing concensus or consistency.

Plato sought a precise demarcation of the distinction through his talk of distinct `worlds', that of Platonic ideals and that of sensory impressions thus connecting the possibility of certainty both with subject matter and epistemic method.
In this division Plato sought a sphere of authority for rational thought broader that the mathematics in which its effectiveness had been demonstrated, and in this was provided a model for many later philosophers who sought the imprimateur of logical certainty for their philosophical theories.
This tendency among philosophers was in modern times called \emph{rationalism} and historically has been attributed to important philosophers such as Descartes, Spinoza and Leibniz.

Aristotle, following Plato based his conception of science on the model of axiomatic geometry, introducing the idea of `demonstrative science', differing primarly in how the principles from which demonstration begin are determined.
The central feature of modern science in which it differed from the science of Aristotle was its empirical method, which firmly set it apart from logic and mathematics.


\subsection{Isaiah Berlin}

Isaiah Berlin's take on The Enlightenment\cite{berlinRR} comes in two parts.
First, ``three legs upon which the whole Western tradition rested'':
\begin{enumerate}
  \item All genuine questions can be answered.

    In principle, by someone.  Perhaps only God.
\item  The answers are knowable.
\item All the answers are compatible (with each other).
  It is a logical truth, Berlin says, that one true proposition cannot contradict another.
\end{enumerate}

and then, the extra twist added by the Enlightenment:
\begin{quotation}
That the knowledge is not to be obtained by revelation, tradition, dogma, introspection..., only by the correct use of reason, deductive or inductive as appropriate to the subject matter.

This extends not only to the mathematical and natural sciences, but to all other matters including ethics, aesthetics and politics.
\end{quotation}
and... that virtue is knowledge.

This is a simple description of an unattainable ideal, elements of which are important to this synthesis.
It will not be expected that all questions have an answer, for it is convenient sometimes to work with entities for which we have only incomplete descriptions, but it is an aspiration that any question definite enough to be amenable to deductive reasoning, either in its establishment or its application, can be accommodated within the synthesis.

There are now strong reasons to doubt that the answer to any properly formulated question can be discovered and established.
In many aspects of the proposed synthesis, absolutes are known to be unrealisable, and it is more important to be confident in the answers which do come than for such answers to be always forthcoming.

That all the answers be compatible is possibly the most crucial requirement in a system intended for large scale deductive elaboration, for in default of coherence, no result can be trusted.
