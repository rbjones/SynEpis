The philosophy of Rudolf Carnap is the closest predecessor I know to the position described in this momograph, but is often misrepresented or misunderstood.
For those who are acquainted with his philosophy, my own position might be easier to understand if juxtaposed and contrasted with my understanding of Carnap (whether or not that understanding is sound), and for others a sketch of his philosophy with some minimal context might nevertheless be helpful.

I am not a scholarly student of Carnap's philosophy, and only came to appreciate its relevance to my own thinking fairly late, partly because what I now see as the main thrust of his philosophy, as clearly outlined in his own intellectual autobiography \cite{carnap1963}, seems not to have been much spoken of by his critics, who have too often focussed on features of his philosophy which were either misrepresented (such as, that he was a phenomenalist) or transient (the verification principle \cite{carnap1937}, his syntactic phase).

The following account leans towards his work on logical analysis and neglects his work on confirmation theory and testing, because my own thinking and this monograph contribute little to those areas.
It pivots around his work on logical syntax \cite{carnap1935}, partly because at this time his lectures in London on that topic \cite{carnap1937} give a concise survey of the main aspects of his philosophy at that time, and both his earlier ideas, as described for example in his intellectual autobiography \cite{carnap1963} and his mature philosophy are both covered well in the Schilpp volume of the Library if Living Philosophers \cite{carnap63a}.
The other important works \cite{carnap1956,carnap1950,carnap1990}  relate to his semantic phase and the criticism and repudiation of his philosophy by W.V.~Quine.
