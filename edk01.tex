This chapter offers a sketch of how abstract semantics arose, the latest stage in an evolutionary process which ran for about 4 Billion years.

The sketch is intended to provide background and motivation for the \emph{philosophical kernel} which follows, in which a notion of \emph{purely abstract} logical truth is described as a foundation for the representation of declarative knowledge in general, which also in some ways facilitates even broader conceptions of knowledge, particularly in relationship to reasoning about such knowledge.
Once that philosophical kernel is in place we will be considering how that foundation for rigorous reasoning can provide a basis for reasoning in less abstract domains.

Declarative knowledge is a late appearance in the evolution of life on earth, and has only gradually approached the ideal form in which it is defined here, which depends not only on fully fledged languages rather than the semiotic representations which preceded them, but also on a semantic precision which has only been approached much later.
For nine tenths of the history of evolution knowledge, if we aknowledge it as such, has been such as might be considered procedural rather than declarative, i.e. it served a purpose in survival and reproduction without achieving clarity of meaning.
