This chapter offers a sketch of how abstract semantics arose, the latest stage in an evolutionary process which ran for about 4 Billion years.
Though long running, the pace of the development accelerated and semantics does not appear in anything like the form we now make of it until those billions have shrunk to thousands.
It would be very difficult to understand how semantics arose, and how it reached the form on which this epistemological synthesis builds, without some appreciation of what preceded it.

Possibly the single most important landmark on that evolutionary journey (after abiogenesis, the evolution of life itself), was the apprearnance of \emph{homo sapiens} and with him, language, the coming of age of semiotic systems.

The sketch is intended to provide background and motivation for the \emph{philosophical kernel} which follows, in which a notion of \emph{purely abstract} logical truth is described as a foundation for the representation of declarative knowledge in general, for deductive reasoning in the context of any coherent body of declarative knowlede.
 
The idea of abstract semantics is closely tied to that of declarative language, and a first characterisation of declarative language is a language defined using a truth conditional semantics, or one for which a semantics can be supplied.

Declarative knowledge is a late appearance in the evolution of life on earth, and has only gradually approached the ideal form in which it is defined here, which depends not only on fully fledged languages rather than the semiotic representations which preceded language, but also on a semantic precision which has only been approached much later.
For nine tenths of the history of evolution knowledge has been procedural rather than declarative, i.e. it served a purpose in survival and reproduction without posessing objective factual content.

\section{Sketch}

OK, this is about the evolution of semantics.

The contention is that it is driven by the demand for ever nore elaborate and extensive reasoning.
This is a fundamental shift away from the traditional evolutionary progressions, in which planning ahead is not supported and each step in an a devlopment must be advantageous independently of what further steps it might enable.




Biological evolution, in its most elementary form (with near random variation and ``natural''selection) requires each small step to provide adaptive advantage not just the possibility of future advantage.



\section{Three}

For the purpose of this narrative, ``semantics'' is what you need to make declarative language definite in meaning and precise.
This is also what it takes to underpin deductive reason and ensure the soundness of arbitrarily lengthy and complex deductive chains and in that way to elaborate the behaviour of abstract models of the real world and thus anticipate the behaviour of engineered systems designed for practical purposes.

It is only now (within the last hundred years) that 
we have become able properly to describe the concept if semantics whose evolution I discuss.
It hinges on the idea of a declarative language as one which has sentences for which the truth conditions can be precisely given.

Though I am talking here of a linguistic phenomenon, and the kind of declarative language if concern dates back no more than maybe 300,000 years, the evolutionary story goes back to the beginning of life on earth, and the conception of declarative language at stake was not properly instantiated until the twentieth century.

So the appearance of declarative language alongside \emph{homo sapiens} was a rough and ready affair, which nevertheless may be thought to have had its accelerating impact on human well-being because of its tentative approximation to an idea which we can only now articulate.

Likely the first to approach such an articulation were the philosophers of ancient greece, who in their transformation of geometry and arithmetic into theoretical disciplines laid the first outlines of an axiomatic method which permitted deductive reasoning to be both extended and reliable.
In that culture Aristotle articulated the elitist conception of philosophy and his demonatrative science as of merit because persued purely for love of knowledge rather than for some more concrete purpose.


