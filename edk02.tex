The evolution of semantics is a slender thread which is pivotal to the epistemological theses which underpin this synthesis.

There are many ways to give structure to the four billion years of evolution here on earth.
The largest scale structure of interest here concerns the evolution of intelligence, which has happened (I suggest) twice, in wholly different ways.

The first time it was biological evolution.
The Darwinian evolution of species \cite{darwin-oos}, which took life on earth from single celled prokaryotes all the way to the species of genus \emph{homo}, of which more than one may have been `intelligent', but only one now remains, \emph{homo sapiens}.

Intelligence comes in degrees and varieties.
In using the term in a black and white way here I am adopting, for present purposes, the criterion that intelligence `proper' is the ability to engage in collaborative design and construction sufficiently well to ultimately engineer intelligent artifacts.

The second time, the evolution which created intelligence was cultural, and not very Darwinian, involving intelligent design as a source of variation, and intelligent selection rather than natural selection (we may debate where the rather artificial boundary between the two might lie).
This second evolution of intelligence is not quite yet complete, for by my chosen criteria of intelligence, it will only be complete when we have artificial intelligence capable of engineering new generations of intelligence.

This division of evolution into two stages of similar significance bbut very different duration (Billions of years against hundred thousands), is significant from the point of view of semantics, and the evolution of semantics exposes developments which have been crucial to the re-invention of intelligence.

The second phase, in which cultural evolution takes the lead, is enabled by the evolution of oral language, and thenceforth the continuing accelerating evolution of way of representing, communicating, storing and exploiting knowledge which have been effective in part because of their approximation to an ideal which the evolution of semantics has only very recently permitted to be clearly articulated.


