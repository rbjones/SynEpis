% $Id: part1.tex $ #
\documentclass[10pt,titlepage]{book}
\usepackage{makeidx}
\usepackage{graphicx}
\usepackage{booktabs}
\usepackage{amsmath}
\usepackage{amssymb}
\usepackage{unicode-math}
\usepackage[unicode]{hyperref}
\usepackage{endnotes}

\usepackage{paralist}
\usepackage{relsize}
\usepackage{verbatim}
\usepackage{enumerate}
\usepackage{longtable}
\usepackage{url}

\usepackage{fontspec}

% Greek Characters

\setmainfont{TeX Gyre Pagella}[Ligatures=TeX]
\setmathfont{TeX Gyre Pagella Math}[Ligatures=TeX]
%\setmainfont{TeX Gyre Pagella}
\newfontfamily\greekfont{GFS Didot}

\usepackage[paperwidth=5.25in,paperheight=8in,hmargin={0.75in,0.5in},vmargin={0.5in,0.5in},includehead,includefoot]{geometry}
\hypersetup{pdfauthor={Roger Bishop Jones}}
\hypersetup{pdftitle={An Epistemological Synthesis}}
\hypersetup{colorlinks=true, urlcolor=red, citecolor=blue, filecolor=blue, linkcolor=blue}
%\usepackage{html}
\renewcommand{\thefootnote}{\roman{footnote}}

\newcommand{\ignore}[1]{}
\newcommand{\rbjdeff}[1]{\textbf{\itshape{#1}}}
\newcommand{\rbjdef}[1]{\rbjdeff{#1}\label{LABEL#1}\index{#1}}
\newcommand{\rbjuse}[1]{\hyperref[LABEL#1]{\itshape{#1}}}
\newcommand{\hreg}[2]{\href{#1}{#2}\footnote{\url{#1}}}
\newcommand{\groknote}[1]{\vfil
  {\bf Note for Grok:}
    \begin{quote}
      {\it #1}
    \end{quote}
    \vfil
  }

\makeatletter
\renewcommand\enoteheading{\section*{\notesname}%
    \@mkboth{\notesname}{\notesname}}

\newcommand{\resetnotes}{%
  \ifnum\value{endnote}>0%
  \theendnotes%
  \setcounter{endnote}{0}%
  \else%
  \fi%
}

\newcommand{\chapnote}[1]{%
  \chapter{#1}
  \input{{#1}.tex}
  \resetnotes
}

\makeatother

\pagestyle{plain}
\makeindex

\title{\bf\LARGE An Epistemological Synthesis}
\author{Roger~Bishop~Jones}
\date{\small 2025-03-06}


\begin{document}
%\frontmatter

%\begin{abstract}
% Presentation of the most fundamental aspects of a positivist philosophical system articulated around a position on knowledge representation.
%
%\end{abstract}
                               
\begin{titlepage}
\maketitle

%\vfill

%\begin{centering}

%{\footnotesize
%copyright\ Roger~Bishop~Jones;
%}%footnotesize

%\end{centering}

\end{titlepage}

\ \

\ignore{
\begin{centering}
{}
\end{centering}
}%ignore

\setcounter{tocdepth}{2}
{\parskip-0pt\tableofcontents}

%\mainmatter


\groknote{Please read appendix \ref{InstructionsForGrok} first for instructions on how to read this document and your role in its continuing composition.}

\chapter{Introduction}

\emph{Epistemology} is the name used by philosophers for \emph{the theory of knowledge}.
More scientifically oriented studies, or cross disciplinary collaborations may be presented as \emph{cognitive science}.

Philosophical approaches to epistmology generally regard knowledge as an established phenomenon, to be observed an analysed, or possibly an area in which articulating or prescribing an ideal might be appropriate.
Plato's conception of knowledge as ``justfied true belief'' is an early and influential example.

Computer science has now reached the point at which the engineering of \emph{intelligent} systems is advancing.
This has largely been achieve by emulating the mechanisms which are credited with effecting intelligence in humans, and the results of this are now coming close to replicating the kinds of intelligence found in \emph{homo sapiens}.
One test, proposed by Alan Turing more than half a century ago, seems now to have been met.
Insofar as one can now tell whether a chat interface is operated by natural or synthetic intelligence, it now rests primarily on matters other than intelligence.
So, for example, it is easy to tell a chatbot, because often they will respond immediately and at length, in a way which would be physically impossible for a human even if he had that miuch mental agility.

However, though they in many cases will outperform humans, in certain asoects which one might think important, they do not.
At this moment, there has been considerable concen over the safety of these AIs, which is 




There are many approaches to epistemology, which for present purposes may be classified as:
\begin{itemize}
\item Analytic
  In which linguistic or logical analysis is used to make clear the meaning of the relevant concepts (such as `know') to obtain theoretical understanding.
\item Descriptive or Naturalistic
  In which observation of the relevant practice rather than analysis of the language is the 
\item Prescriptive
\item Synthetic
  \end{itemize}
  


\cite{couturat1901logique}


\resetnotes

\chapter{A Philosophical Kernel}\label{PK}

%A kernel, as I use the concept here, is a core which provides essential and fundamental elements upon which a larger system of some kind is built.

This chapter presents a kernel for an epistemological philosophical system.
A key feature of this kernel is the definition of a notion of \emph{logical truth} which is considered foundational for knowledge and epistemology.
In preparation for promoting a particular formalisation of that conception of logical truth, I describe an ordering on logical systems according to their expressiveness, relative to which a maximally expressive system will be sought.

In the next chapter I will identify a universal family of logical systems and a \emph{logical} kernel for that family.

The first, which is the subject of this chapter, is the Kernel or fundamental core of a philosophical system within the Western tradition which is generally considered to have begun with the philosophers of Classical Greece.

In that early beginning, starting around 600 BC with Thales, and progressing through the so called `pre-socratics' to the great intellectual achievements of Plato and Aristotle, philosophy was the love of knowledge, 

The philosophy has a kernel because it is an example of `first philosophy', an idea introduced by Aristotle in his \emph{Metaphysics} \cite{aristotleMetap}.


\resetnotes

\chapter{Evolution}

\cite{donald1991,murray2017evolution}

\ignore{
\chapter{Synthetic Epistemology and Foundational Abstract Semantics}


\section{Evolution and Epistemology}

To give a first account of the way in which evolution and epistemology are entwined in this narrative, I must provide characterisations of those two concepts, both of which I take in a broad sense sympathetic to the ideas whose exposition they abet.

\subsection{What is Evolution?}

The term evolution is used here for any process of progressive or incremental change whose long term effects are realised through some kind of differential proliferation.
`Differential prolifearation' occurs when certain kinds of entities proliferate (multiply and/or disperse, replication admitting some degree of variation) at rates which vary according to their particular characteristics.
The basic principle here is that those kinds of entity which proliferate most profusely within some environmental niche will come to numerically dominate in that niche.
Those small changes which improve proliferation are considered `adaptive', and over time these small changes may yield the major transformations which are seen (for example) in the evolution of species.

However, this conception of evolution lacks some of the characteristics usually seen in the evolution of species.
In the classical conception, the variations which are essential to evolutionary progress are supposed random, and proliferation is moderated by `natural selection' which determines which variants prove most prolific.
These characteristics may be helpful in advertising a natural process of evolution which does not depend on divine intervention, and may be accurate descriptions of biological evolution, but are not found in all the kinds of evolution which are of interest here.
Prebiotic evolution, cultural evolution, or the kinds of evolution which may yet emerge as artificial intelligence and synthetic biology mature to dominate evolution in the future.

\subsection{What is Epistemology?}

Though talking about knowledge is nearly as old as language itself, epistemology, the philosophical theory of knowledge, probably begins with Plato, who spoke of knowledge as \emph{justified true belief}.
Insofar as we may infer the meaning of the term from its usage, which is diverse, this is a narrow characterisation.
It is psychologistic, anthropocentric, and addresses only the kind of rigorous knowledge we associate with science, to which philosophy often aspires.

The kind of epistemology in which this monograph engages I call \emph{synthetic}, and it yeilds a synthesis or construction of a conception of knowledge primarily defined by the manner in which knowledge is \emph{represented}, diversifying the metrics evaluating supposed knowledge away from `justifiction' to admit criteria appropriate to the full diversity of knowledge, including, for example, procedural  knowledge, or \emph{knowing how}.

The epistemology here is \emph{synthetic}, which can be understood by contrast with three other approaches to epistmology which are:
\begin{itemize}
\item[analytic]

  This is the kind of philosophy which might be most appropriate to the analytic tradition in philosophy, particularly its most recent manifestations in the early to mid twentieth century.
  It takes language \emph{as it is}, perhaps even with an emphasis on \emph{ordinary language} and enquires what terms like `know' and `knowledge' mean in that established usage.
  It may be noted that in talking about `analysis' here, we are primarily speaking of analysis of language rather than logical analysis, and that this process properly yeilds synthetic truths about a natural phenomenon rather than logical truths.
\item[prescriptive]
  epistemology becomes prescriptive when it is concerned to determine how the relevant language \emph{should} be used rather than how it \emph{is} used.
  \emph{natural}
  Natural epistemology, or epistemology naturalised, is a kind of epistemology beginning later in the twentieth century, initiated by W.V.Quine \cite{quine1969epistemology}, in which knowledge is considered a natural phenomenon and should be persued by the methods of empirical science.
\end{itemize}

\resetnotes

\chapter{The Evolution of Declarative Knowledge}\label{EDK}

%The evolution of semantics is a slender thread which is pivotal to the epistemological theses which underpin this synthesis.

There are many ways to give structure to the four billion years of evolution here on earth.
The largest scale structure of interest here concerns the evolution of intelligence, which has happened (I suggest) twice, in wholly different ways.

The first time it was biological evolution.
The Darwinian evolution of species \cite{darwin-oos}, which took life on earth from single celled prokaryotes all the way to the species of genus \emph{homo}, of which more than one may have been `intelligent', but only one now remains, \emph{homo sapiens}.

Intelligence comes in degrees and varieties.
In using the term in a black and white way here I am adopting, for present purposes, the criterion that intelligence `proper' is the ability to engage in collaborative design and construction sufficiently well to ultimately engineer intelligent artifacts.

The second time, the evolution which created intelligence was cultural, and not very Darwinian, involving intelligent design as a source of variation, and intelligent selection rather than natural selection (we may debate where the rather artificial boundary between the two might lie).
This second evolution of intelligence is not quite yet complete, for by my chosen criteria of intelligence, it will only be complete when we have artificial intelligence capable of engineering new generations of intelligence.

This division of evolution into two stages of similar significance bbut very different duration (Billions of years against hundred thousands), is significant from the point of view of semantics, and the evolution of semantics exposes developments which have been crucial to the re-invention of intelligence.

The second phase, in which cultural evolution takes the lead, is enabled by the evolution of oral language, and thenceforth the continuing accelerating evolution of way of representing, communicating, storing and exploiting knowledge which have been effective in part because of their approximation to an ideal which the evolution of semantics has only very recently permitted to be clearly articulated.



\resetnotes


\chapter{The Fundamental Triple Trichotomy}\label{FTT}

%Consideration of the domain in which extended deductive reasoning can be made safe has lead to a focus on purely abstract logical truths.
In this chapter I will talk about other important domains of discourse and how logical truths can enable deductive reasoning in those domains.

\section{Hume's Forks}\label{HF}

The triple-trichotomy is and elaboration on a theme fundmnetal to the philosophy of David Hume, who gave a central place to the dstinction which became known as `Hume's Fork' 


David Hume was a philosopher of the Scottish Enlightenment.
The enlightenment was a period of ascendency in the place of reason in
the discussion of human affairs, when science had secured its
independence from the authority of church and state and had a new
confidence in its powers occasioned particularly by the successes of
Newtonian physics.

Hume looked upon the philosophical writings of his contemporaries and
found in them two principle kinds, an ``easy'' kind which appealed to
the sentiments of the reader, and a ``hard'' kind which trawled deeper
and appealed to reason.
This latter kind, ``commonly called'' metaphysics, was preferred by
Hume, but found nevertheless, by him, to be lacking, infested with religious 
fears and prejudices.
Hume's feelings about these aspects of philosophy were not vague
misgivings.
He had a specific epistemological criterion which he saw these
philosophical doctrines as violating.

Hume's project involves an enquiry into the nature of human reason for
the purpose of eliminating those parts of metaphysics which go beyond
the limits of knowledge, and establishing a new metaphysics on a
solid foundation limited to those matters which fall within the scope
of human understanding.

David Hume wrote his philosophical {\it magnum opus}, {\it A Treatise on
  Human Nature} \cite{hume39} as a young man.
He was disappointed to find his work largely ignored and otherwise
misunderstood, and thought perhaps that his presentation had been at
fault.
To improve matters he wrote a shorter work more tightly focussed
on the core messages which he thought of greatest importance.
This he called {\it An Enquiry into Human Understanding}
\cite{hume48}.

In a central place both logically and physically in this more concise
account of his philosophy he says:

\begin{quote}
``ALL the objects of human reason or enquiry may naturally be divided
  into two kinds, to wit, Relations of Ideas, and Matters of Fact.'' 
\end{quote}

We shall see that Hume is here identifying a single dichotomy which
corresponds to all three of the distinctions which here concern us.
In his next two paragraphs he expands in turn on the kinds he has thus
introduced.

\subsection{Relations of Ideas}

\begin{quote}
``Of the first kind are the sciences of Geometry, Algebra, and
Arithmetic; and in short, every affirmation which is either
intuitively or demonstratively certain.
That the square of the hypotenuse is equal to the square of the two
sides, is a proposition which expresses a relation between these
figures.
That three times five is equal to the half of thirty, expresses a
relation between these numbers.
Propositions of this kind are discoverable by the mere operation of
thought, without dependence on what is anywhere existent in the
universe.
Though there never were a circle or triangle in nature, the truths
demonstrated by Euclid would for ever retain their certainty and
evidence.''
\end{quote}

Hume is distinctive here among empiricist philosophers in having a
broad conception of the {\it a priori} (though he does not use that
term here), allowing that the whole of mathematics is {\it a priori} (a concept encapsulated in Hume by the phrase ``discoverable by the mere operation of thought ...'').
In this he may be contrasted, for example, with Locke who allowed
only certain rather trivial logical truths to be knowable {\it a
  priori}.
Nevertheless, Hume's conception of the {\it a priori} remains narrow
by comparison with the rationalists, and in particular, as Hume will
later emphasize, excludes metaphysics.

\subsection{Matters of Fact}

\begin{quote}
``Matters of fact, which are the second objects of human reason, are not ascertained in the same manner; nor is our evidence of their truth, however great, of a like nature with the foregoing. The contrary of every matter of fact is still possible; because it can never imply a contradiction, and is conceived by the mind with the same facility and distinctness, as if ever so conformable to reality. That the sun will not rise to-morrow is no less intelligible a proposition, and implies no more contradiction than the affirmation, that it will rise. We should in vain, therefore, attempt to demonstrate its falsehood. Were it demonstratively false, it would imply a contradiction, and could never be distinctly conceived by the mind.''
\end{quote}

The evolution of the following three dichotomies is the theme of this chapter, though we will find other related dichotomies which feature in the history.

The terms which I will use to speak of them, in this chapter are:
\begin{itemize}
\item{analytic/\-synthetic}
\item{necessary/contingent}
\item{a priori/a posteriori}
\end{itemize}

As I shall use these terms these are divisions of different kinds of
entity, by different means.
For that reason they cannot be said to be identical, and their extents clearly depend upon exactly how the relevant technical concepts are defined, but with suitable and reasonable definitions, these dichotomies prove to be very closely coupled.

The first is a division of sentences, understood in sufficient context
to have a definite meaning, and is a division dependent upon that
meaning.
Meaning is not an univocal term, it is particularly uncertain in meaning,
But in this context the requirement is very specific, only one possible component of meaning is required, which is the truth conditions of the sentence.
The truth conditions of a sentence are an assignment of truth values for he sentence in every possible condition (what is a `condition' depends on the language, for natural languages a condition would include both a state of the world and sufficient context to disambiguate the sentence if its meaning is in any way context sensitive). 

The second is a division of {\it propositions}, which may be
understood for present purposes as {\it meanings} of sentences in
context.
What a proposition is need not be settled except that it must include, again, truth conditions, the context of any sentence having been settled to determine the proposition which it expresses.
Such a proposition is considered necessary if it is true in every relevant circumstance, the range of circumstances having been fixed by the language and constituting the subject matter of the language.
The concepts of analyticity and necessity are, by this kind of definition, logically related.
A sentence is analytic if once disambiguated by appropriate context, it is seen to exppress a necessary proposition.
These exact definitions are not to be found in Hume, and are not philosophically uncontroversial, but are presented here so that we can speak of that position adopted by some later philosophers which Hume may be thought to have anticipated.

The division is made according to whether the proposition
expressed must under all circumstances have the same truth value, or
whether its truth value varies according to circumstance.
In this we are concerned with two particular notions of necessity,
those of logical and of metaphysical necessity, the latter being
sometimes taken to be broader than the former.
A part of the role of Hume's fork in positivist philosophy is to
banish metaphysical necessity insofar as this goes beyond logical necessity.

The third is for our purposes also a division of propositions, on a
different basis.
It concerns the status of claims or of supposed knowledge of
propositions.
It is expected that such a claim must in some way be
{\it justified} if we are to accept it, and that the kind of
justification required depends upon the proposition to be justified.
The justification is \emph{a priori} if it makes no reference to observations
about the state of the world, i.e. to sensory observations or other results obtained on the basis of such evidence.
The distinction in this case may be described as an epistemic distinction, since it concerns what kind of justification we may expect for the kind of proposition in question.

The suggested identity between the first two concepts has a pale reflection in relation to this epistemic distinction.
In this case we do not assert an identity but rather recomend that an \emph{a priori} justification be required for necessary propositions, and that an \emph{e posteriori} justification be required for contingent propositions, thus closely if indirectly coupling the three dichotomies.

\subsection{The Place of The Fork in Hume's Philosophy}

The mere statement of the fork (which we shall see, is not original in
Hume) is of lesser significance than the role which it plays in Hume's
philosophy, which serves to clarify the distinction at stake and draw
out its significance.

Hume's philosophy, like Descartes' comes in two parts of which the
first is sceptical in character, and the second constructive.
In both cases the sceptical part clears the ground for a new approach
to philosophy which is then adopted in the constructive phase.

For our present purposes we are concerned principally With the first
sceptical phase of Hume's philosophy, because of the delineation of the scope
of deductive reason, and hence of the analytic/\-synthetic dichotomy
which is found in Hume's sceptical arguments.
This delineation is baldly stated in Hume's first description of the
distinction between ``relations between ideas'' and ``matters of
fact'', for there Hume tells us that no matter of fact is
demonstrable.

This bald statement would by itself have little persuasive force if it
were not followed up with more detail, even though ultimately this
detail does not so much underpin the distinction as depend upon it.

Hume's further discussion begins with the consideration of what
matters of fact can be known `beyond the present testimony of 
our senses or the records of our memory'.
The inference beyond this immediate data is invariable causal, we
infer from the sensory impressions or memories to the supposed causes
of those impressions.
But these are not logical inferences, causal necessity is for Hume no
necessity at all (even less the inference from effect to cause).
Hume's central thesis that matters of fact are not demonstrable is
in this way reduced first to the logical independence of cause and
effect, and then to the distinction between deductive (and hence sound)
inference and inductive inference (whereby we infer causal
regularities and their consequences).

Given that Hume considers all inferences from senses to be based on
induction, and sees no validity in causal inference, it follows that
from information provided directly to us by the senses nothing further
can be deduced which is not simply a restatement, selection or summary
of the information itself.
Further enlightenment from this sceptical doctrine is primarily the
application of this principle to various kinds of knowledge.
In the process Hume does a certain amount of 



\section{The Triple Trichotomy}

The `triple trichotomy' is a presentation of these extended domains of reasoning as a two dimensional matrix, one dimension associated with the distinct domains, the other with different kinds of characteristics which they posess.

\begin{table}[h]
    \centering
    \begin{tabular}{l | l | l | l}
         & \rotatebox{45}{Semantics} & \rotatebox{45}{Evaluation} & \rotatebox{45}{Modality} \\
        \midrule
        Spiritual & non-natural  & propriety & normative \\
        \midrule
        Empirical & concrete & utility & contingent \\
        \midrule
       Logical & abstract & proof & necessary \\
    \end{tabular}
    \caption{The Fundamental Triple Trichotomy}
    \label{tab:example}
\end{table}

In the above table, abstract semantics for logical truths (which are to be established by deductive proof and are necessary rather than contingent, or normative). is shown in the lower left corner as befits concepts which are regarded as foundational to the whole.

I will talk through this table in the sections which follow, working from bottom up and left acrosss.

\section{Semantics}

David Hume's philosophy provides us with a first view of the three domains, through two important distinctions.
The first is what later became known as the \emph{analytic/synthetic} distinction, that described by Hume as `relations between ideas' and `matters of fact'.
Because Hume talks of it in terms of subject matter, this is readily understood as a semantic distinction.
Hume also considers both those two categories as concerned with what \emph{is}, rather than what \emph{ought to be}, and says that these normative claims are not derivable from mere descriptions, one cannot derive an `ought' from an `is', giving us three distinct categories each logically beyond its predecessors.

So how can the abstract semantics which yields logical truth be foundational for these domains which seem logically beyond it?
It is foundational because we can mimic the structure of these extended domains in pure abstractions, and then describe map the abstract ontology onto the concrete entities and the normative concepts.

\section{Evaluation}

In the classical conception of knowledge as justified true belief, we see that to establish something as knowledge we must be able to show conclusively that it is true.



\resetnotes
}%ignore

\appendix

\ignore{
\chapter{The Philosophy of Rudolf Carnap}\label{PRC}

The philosophy of Rudolf Carnap is the closest predecessor I know to the position described in this momograph, but is often misrepresented or misunderstood.
For those who are acquainted with his philosophy, my own position might be easier to understand if juxtaposed and contrasted with my understanding of Carnap (whether or not that understanding is sound), and for others a sketch of his philosophy with some minimal context might nevertheless be helpful.

I am not a scholarly student of Carnap's philosophy, and only came to appreciate its relevance to my own thinking fairly late, partly because what I now see as the main thrust of his philosophy, as clearly outlined in his own intellectual autobiography \cite{carnap1963}, seems not to have been much spoken of by his critics, who have too often focussed on features of his philosophy which were either misrepresented (such as, that he was a phenomenalist) or transient (the verification principle \cite{carnap1937}, his syntactic phase).

The following account leans towards his work on logical analysis and neglects his work on confirmation theory and testing, because my own thinking and this monograph contribute little to those areas.
It pivots around his work on logical syntax \cite{carnap1935}, partly because at this time his lectures in London on that topic \cite{carnap1937} give a concise survey of the main aspects of his philosophy at that time, and both his earlier ideas, as described for example in his intellectual autobiography \cite{carnap1963} and his mature philosophy are both covered well in the Schilpp volume of the Library if Living Philosophers \cite{carnap63a}.
The other important works \cite{carnap1956,carnap1950,carnap1990}  relate to his semantic phase and the criticism and repudiation of his philosophy by W.V.~Quine.

\resetnotes

}%ignore

\appendix

\chapter{Instructions for Grok3}\label{InstructionsForGrok}

I have written this monograph with copious assistance from Grok 3, an AI chatbot from xAI.

Notwithstanding the invaluable contribution Grok has made, the writing is all mine.

The following section contains the general instructions which I have used to brief Grok.
Naturally, these instructions have evolved over the course of the writing and this their state at the end.
Different stages in the process demand different instructions, and I have tried to keep instructions of a general character  in a separate file (presented below), with supplementary files giving more specific asks for particular stages.

\section{General Guidance for Grok}

My principle intellectual ambitions are now primarily philosophical, with a generous and flexible sense of the scope and methods of philosophy, but, nevertheless, a focus on theoretical foundations.

I aim to practice philosophy by producing written accounts of the ideas I am working on, which are oriented toward philosophical support for the effective application of modern advances in logic.
I now think of these as monographs, but expect that this may change as I learn to work effectively with Grok.
These notes are therefore written for Grok, and I will couch them in first and second person language.

I have decided that the best way to deal with the readership targeting is to write for an audience of two, you, Grok and me, Roger Jones.
I want to produce in the first instance a monograph which seems to me to adequately capture the ideas I had hoped to express, and which is clear enough for you to get a fairly deep understanding of them.
I don't yet have any well thought out ideas about how to check your understanding, so the first idea is conversation.
Maybe I will use the common textbook practice of including questions at the end of each chapter for you to answer and I can then review your answers with you.

Once we have a manuscript which I think is a good statement and which you understand well, I will ask you to prepare (with my help) materials for a variety of other target audiences and other channels of communication, which I hope will include making the relevant ideas available via a grok3 conversation (which will be possible if in no other way, but attaching the manuscipt to a grok session, though I will be hoping for a way of gathering feedback from these sessions which will require something more sophisticated, probably supported by future developments to Grok in X about which I can only speculate),

As to channels, the first product will be a PDF suitable for distribution electronically or printing on demand.
This could be converted into an ebook in multiple formats.
I am inclined to think of X as a main channel for promulgation, and imagine that you would be able to support a bot account on X (in due course) which would explain the ideas.
I also anticipate serialising the monograph on substack, and would want you to convert the chapters one at a time into posts for substack.
You may have some ideas on what other channels might be good ways of promulgating the work.

Given the breadth of the considerations, particularly the divide between evolutionary thinking, philosophy, logic and cognitive science/engineering, I would like it to be possible for those who interest and/or competence in only some of these areas to find something of interest even though perhaps skimming of omitting parts of the document outside their competence or interest.
So I shall be constantly trying to sketch informally in introductory parts and concetrate detail and depth into separate chapters or sections.

You are to act as my research assistant and reviewer of the developing work, but you will not be contributing directly to the writing of the first manuscript, i.e. the words will be mine, though I hope and expect that your indirect contribution will be substantial.
When it comes to derived works, the ideas will still be mine (though refined in discussion with you), but the words will be yours.

In all responses concision is very important.
Prolixity will impede my progress rather than advance it, you will need to cultivate a conservative sense of what is relevant and important.
Note also, that I am not looking for creative suggestions (unless I should explicitly ask for some).
Beyond fact checking I am looking to kmow whether what I am writing is intelligible.

Your general knowledge of all the matters touched on in the exposition is superior to mine, except in those parts where the monograph presents original material, and my first hope is that you will fact check everything which I write and let me know if there are any factual errors.

My second hope is that you will mention to me other work which I ought to be aware of because of its proximity to the subject matters I am addressing.
In this I am not asking you to compile a list of those things which seem most relevant even if tenuously 

The third desideratum is that you take cognisance of the intended or likely readership and advise me on matters addressed which may be dificult for that audience to comprehend, and perhaps make suggestions as to how improvements can be made.
Similarly with any matters with which they may seem likely to disagree, particularly if their disagreement is soundly based.

When I attach a document, or re-attach one, I would like you to automatically read the document.
This will usually be the current draft manuscript, which includes as an appendix these notes.
I will also be including in the body of the monograph notes specifically for you to understand what I am doing and where I am going, for which I will wrap in a tex command which can hide them in the final manuscript.
These sections of the document will be in quote blocks marked ``Note for Grok:''.




\listoftables
%\listoffigures

%\phantomsection
\addcontentsline{toc}{section}{Bibliography}
\bibliographystyle{rbjfmu}
\bibliography{rbj3}

%\phantomsection
\renewcommand{\indexname}{Index of Defined Terms}
\addcontentsline{toc}{section}{Index of Defined Terms}
{\twocolumn[]
{\small\printindex}}

%\vfill

%\tiny{
%Started 2024-10-19
%}%tiny


\end{document}

% LocalWords:
