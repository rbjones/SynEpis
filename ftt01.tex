Consideration of the domain in which extended deductive reasoning can be made safe has lead to a focus on purely abstract logical truths.
In this chapter I will talk about other important domains of discourse and how logical truths can enable deductive reasoning in those domains.

\section{Hume's Forks}\label{HF}

The triple-trichotomy is and elaboration on a theme fundmnetal to the philosophy of David Hume, who gave a central place to the dstinction which became known as `Hume's Fork' 


David Hume was a philosopher of the Scottish Enlightenment.
The enlightenment was a period of ascendency in the place of reason in
the discussion of human affairs, when science had secured its
independence from the authority of church and state and had a new
confidence in its powers occasioned particularly by the successes of
Newtonian physics.

Hume looked upon the philosophical writings of his contemporaries and
found in them two principle kinds, an ``easy'' kind which appealed to
the sentiments of the reader, and a ``hard'' kind which trawled deeper
and appealed to reason.
This latter kind, ``commonly called'' metaphysics, was preferred by
Hume, but found nevertheless, by him, to be lacking, infested with religious 
fears and prejudices.
Hume's feelings about these aspects of philosophy were not vague
misgivings.
He had a specific epistemological criterion which he saw these
philosophical doctrines as violating.

Hume's project involves an enquiry into the nature of human reason for
the purpose of eliminating those parts of metaphysics which go beyond
the limits of knowledge, and establishing a new metaphysics on a
solid foundation limited to those matters which fall within the scope
of human understanding.

David Hume wrote his philosophical {\it magnum opus}, {\it A Treatise on
  Human Nature} \cite{hume39} as a young man.
He was disappointed to find his work largely ignored and otherwise
misunderstood, and thought perhaps that his presentation had been at
fault.
To improve matters he wrote a shorter work more tightly focussed
on the core messages which he thought of greatest importance.
This he called {\it An Enquiry into Human Understanding}
\cite{hume48}.

In a central place both logically and physically in this more concise
account of his philosophy he says:

\begin{quote}
``ALL the objects of human reason or enquiry may naturally be divided
  into two kinds, to wit, Relations of Ideas, and Matters of Fact.'' 
\end{quote}

We shall see that Hume is here identifying a single dichotomy which
corresponds to all three of the distinctions which here concern us.
In his next two paragraphs he expands in turn on the kinds he has thus
introduced.

\subsection{Relations of Ideas}

\begin{quote}
``Of the first kind are the sciences of Geometry, Algebra, and
Arithmetic; and in short, every affirmation which is either
intuitively or demonstratively certain.
That the square of the hypotenuse is equal to the square of the two
sides, is a proposition which expresses a relation between these
figures.
That three times five is equal to the half of thirty, expresses a
relation between these numbers.
Propositions of this kind are discoverable by the mere operation of
thought, without dependence on what is anywhere existent in the
universe.
Though there never were a circle or triangle in nature, the truths
demonstrated by Euclid would for ever retain their certainty and
evidence.''
\end{quote}

Hume is distinctive here among empiricist philosophers in having a
broad conception of the {\it a priori} (though he does not use that
term here), allowing that the whole of mathematics is {\it a priori} (a concept encapsulated in Hume by the phrase ``discoverable by the mere operation of thought ...'').
In this he may be contrasted, for example, with Locke who allowed
only certain rather trivial logical truths to be knowable {\it a
  priori}.
Nevertheless, Hume's conception of the {\it a priori} remains narrow
by comparison with the rationalists, and in particular, as Hume will
later emphasize, excludes metaphysics.

\subsection{Matters of Fact}

\begin{quote}
``Matters of fact, which are the second objects of human reason, are not ascertained in the same manner; nor is our evidence of their truth, however great, of a like nature with the foregoing. The contrary of every matter of fact is still possible; because it can never imply a contradiction, and is conceived by the mind with the same facility and distinctness, as if ever so conformable to reality. That the sun will not rise to-morrow is no less intelligible a proposition, and implies no more contradiction than the affirmation, that it will rise. We should in vain, therefore, attempt to demonstrate its falsehood. Were it demonstratively false, it would imply a contradiction, and could never be distinctly conceived by the mind.''
\end{quote}

The evolution of the following three dichotomies is the theme of this chapter, though we will find other related dichotomies which feature in the history.

The terms which I will use to speak of them, in this chapter are:
\begin{itemize}
\item{analytic/\-synthetic}
\item{necessary/contingent}
\item{a priori/a posteriori}
\end{itemize}

As I shall use these terms these are divisions of different kinds of
entity, by different means.
For that reason they cannot be said to be identical, and their extents clearly depend upon exactly how the relevant technical concepts are defined, but with suitable and reasonable definitions, these dichotomies prove to be very closely coupled.

The first is a division of sentences, understood in sufficient context
to have a definite meaning, and is a division dependent upon that
meaning.
Meaning is not an univocal term, it is particularly uncertain in meaning,
But in this context the requirement is very specific, only one possible component of meaning is required, which is the truth conditions of the sentence.
The truth conditions of a sentence are an assignment of truth values for he sentence in every possible condition (what is a `condition' depends on the language, for natural languages a condition would include both a state of the world and sufficient context to disambiguate the sentence if its meaning is in any way context sensitive). 

The second is a division of {\it propositions}, which may be
understood for present purposes as {\it meanings} of sentences in
context.
What a proposition is need not be settled except that it must include, again, truth conditions, the context of any sentence having been settled to determine the proposition which it expresses.
Such a proposition is considered necessary if it is true in every relevant circumstance, the range of circumstances having been fixed by the language and constituting the subject matter of the language.
The concepts of analyticity and necessity are, by this kind of definition, logically related.
A sentence is analytic if once disambiguated by appropriate context, it is seen to exppress a necessary proposition.
These exact definitions are not to be found in Hume, and are not philosophically uncontroversial, but are presented here so that we can speak of that position adopted by some later philosophers which Hume may be thought to have anticipated.

The division is made according to whether the proposition
expressed must under all circumstances have the same truth value, or
whether its truth value varies according to circumstance.
In this we are concerned with two particular notions of necessity,
those of logical and of metaphysical necessity, the latter being
sometimes taken to be broader than the former.
A part of the role of Hume's fork in positivist philosophy is to
banish metaphysical necessity insofar as this goes beyond logical necessity.

The third is for our purposes also a division of propositions, on a
different basis.
It concerns the status of claims or of supposed knowledge of
propositions.
It is expected that such a claim must in some way be
{\it justified} if we are to accept it, and that the kind of
justification required depends upon the proposition to be justified.
The justification is \emph{a priori} if it makes no reference to observations
about the state of the world, i.e. to sensory observations or other results obtained on the basis of such evidence.
The distinction in this case may be described as an epistemic distinction, since it concerns what kind of justification we may expect for the kind of proposition in question.

The suggested identity between the first two concepts has a pale reflection in relation to this epistemic distinction.
In this case we do not assert an identity but rather recomend that an \emph{a priori} justification be required for necessary propositions, and that an \emph{e posteriori} justification be required for contingent propositions, thus closely if indirectly coupling the three dichotomies.

\subsection{The Place of The Fork in Hume's Philosophy}

The mere statement of the fork (which we shall see, is not original in
Hume) is of lesser significance than the role which it plays in Hume's
philosophy, which serves to clarify the distinction at stake and draw
out its significance.

Hume's philosophy, like Descartes' comes in two parts of which the
first is sceptical in character, and the second constructive.
In both cases the sceptical part clears the ground for a new approach
to philosophy which is then adopted in the constructive phase.

For our present purposes we are concerned principally With the first
sceptical phase of Hume's philosophy, because of the delineation of the scope
of deductive reason, and hence of the analytic/\-synthetic dichotomy
which is found in Hume's sceptical arguments.
This delineation is baldly stated in Hume's first description of the
distinction between ``relations between ideas'' and ``matters of
fact'', for there Hume tells us that no matter of fact is
demonstrable.

This bald statement would by itself have little persuasive force if it
were not followed up with more detail, even though ultimately this
detail does not so much underpin the distinction as depend upon it.

Hume's further discussion begins with the consideration of what
matters of fact can be known `beyond the present testimony of 
our senses or the records of our memory'.
The inference beyond this immediate data is invariable causal, we
infer from the sensory impressions or memories to the supposed causes
of those impressions.
But these are not logical inferences, causal necessity is for Hume no
necessity at all (even less the inference from effect to cause).
Hume's central thesis that matters of fact are not demonstrable is
in this way reduced first to the logical independence of cause and
effect, and then to the distinction between deductive (and hence sound)
inference and inductive inference (whereby we infer causal
regularities and their consequences).

Given that Hume considers all inferences from senses to be based on
induction, and sees no validity in causal inference, it follows that
from information provided directly to us by the senses nothing further
can be deduced which is not simply a restatement, selection or summary
of the information itself.
Further enlightenment from this sceptical doctrine is primarily the
application of this principle to various kinds of knowledge.
In the process Hume does a certain amount of 



\section{The Triple Trichotomy}

The `triple trichotomy' is a presentation of these extended domains of reasoning as a two dimensional matrix, one dimension associated with the distinct domains, the other with different kinds of characteristics which they posess.

\begin{table}[h]
    \centering
    \begin{tabular}{l | l | l | l}
         & \rotatebox{45}{Semantics} & \rotatebox{45}{Evaluation} & \rotatebox{45}{Modality} \\
        \midrule
        Spiritual & non-natural  & propriety & normative \\
        \midrule
        Empirical & concrete & utility & contingent \\
        \midrule
       Logical & abstract & proof & necessary \\
    \end{tabular}
    \caption{The Fundamental Triple Trichotomy}
    \label{tab:example}
\end{table}

In the above table, abstract semantics for logical truths (which are to be established by deductive proof and are necessary rather than contingent, or normative). is shown in the lower left corner as befits concepts which are regarded as foundational to the whole.

I will talk through this table in the sections which follow, working from bottom up and left acrosss.

\section{Semantics}

David Hume's philosophy provides us with a first view of the three domains, through two important distinctions.
The first is what later became known as the \emph{analytic/synthetic} distinction, that described by Hume as `relations between ideas' and `matters of fact'.
Because Hume talks of it in terms of subject matter, this is readily understood as a semantic distinction.
Hume also considers both those two categories as concerned with what \emph{is}, rather than what \emph{ought to be}, and says that these normative claims are not derivable from mere descriptions, one cannot derive an `ought' from an `is', giving us three distinct categories each logically beyond its predecessors.

So how can the abstract semantics which yields logical truth be foundational for these domains which seem logically beyond it?
It is foundational because we can mimic the structure of these extended domains in pure abstractions, and then describe map the abstract ontology onto the concrete entities and the normative concepts.

\section{Evaluation}

In the classical conception of knowledge as justified true belief, we see that to establish something as knowledge we must be able to show conclusively that it is true.


