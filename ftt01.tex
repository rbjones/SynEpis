Consideration of the domain in which extended deductive reasoning can be made safe has lead to a focus on purely abstract logical truths.
In this chapter I will talk about other important domains of discourse and how logical truths can enable deductive reasoning in those domains.

The `triple trichotomy' is a presentation of these extended domains of reasoning as a two dimensional matrix, one dimension associated with the distinct domains, the other with different kinds of characteristics which they posess.

\begin{table}[h]
    \centering
    \begin{tabular}{l | l | l | l}
         & \rotatebox{45}{Semantics} & \rotatebox{45}{Evaluation} & \rotatebox{45}{Modality} \\
        \midrule
        Spiritual & non-natural  & propriety & normative \\
        \midrule
        Empirical & concrete & utility & contingent \\
        \midrule
       Logical & abstract & proof & necessary \\
    \end{tabular}
    \caption{The Fundamental Triple Trichotomy}
    \label{tab:example}
\end{table}

In the above table, abstract semantics for logical truths (which are to be established by deductive proof and are necessary rather than contingent, or normative). is shown in the lower left corner as befits concepts which are regarded as foundational to the whole.

I will talk through this table in the sections which follow, working from bottom up and left acrosss.

\section{Semantics}

David Hume's philosophy provides us with a first view of the three domains, through two important distinctions.
The first is what later became known as the \emph{analytic/synthetic} distinction, that described by Hume as `relations between ideas' and `matters of fact'.
Because Hume talks of it in terms of subject matter, this is readily understood as a semantic distinction.
Hume also considers both those two categories as concerned with what \emph{is}, rather than what \emph{ought to be}, and says that these normative claims are not derivable from mere descriptions, one cannot derive an `ought' from an `is', giving us three distinct categories each logically beyond its predecessors.

So how can the abstract semantics which yields logical truth be foundational for these domains which seem logically beyond it?
It is foundational because we can mimic the structure of these extended domains in pure abstractions, and then describe map the abstract ontology onto the concrete entities and the normative concepts.

\section{Evaluation}

In the classical conception of knowledge as justified true belief, we see that to establish something as knowledge we must be able to show conclusively that it is true.


