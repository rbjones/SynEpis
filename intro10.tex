This monograph, which I have styled 'synthetic epistemology', is constructively concerned with the nature of knowledge.
It constitutes, not an examination of some epistemological \emph{fait accomplit}, the true nature of knowledge, but rather the engineering of an epistemological \emph{desideratum}, how we might do well to think of knowledge.
Consideration of what knowledge has been, and is, forms a part of the deliberations, but is not my purpose, which is to contribute technically and philosophically to what it will be.

Among the many forms which knowledge takes, declarative knowledge, primarily consisting in that kind of knowledge which can be expressed or represented in declarative language (for which a \emph{definition} will be forthcoming), is relatively recent, having its beginnings probably about the same time as anatomically modern \emph{homo spaiens}.
Such knowledge has characteristics derived from those of declarative language, and which vary widely according to the domain of discourse.
In some domains, such as mathematics, it can be extremely precise in meaning, with deductive methods of confirmation which are highly reliable.
In others, particularly those reaching into the deepest recesses of the human mind, or the highest achievements of art and literature may be subjective in character, imprecise in meaning or lack consensus as to truth.
Many will hold these latter truths more important than the dry inhuman theories of mathematics, but cultural refinement is a luxury which only those whose material needs are well catered for can enjoy, and it is mathematics and science, with their need for precision and certainty, which have served best to fultill those needs with room to spare for culture.

That a foundation addressing precision of meaning and certainty of truth in those domains most conducive to them, can benefit broader domains can be illustrated by considering three domains distinguished by the philosopher David Hume.
At the centre of Hume's philosophy was the distinction between propositions concerning \emph{relations between ideas} and those addressing matters of fact.
Foundationally more peripheral but probably no less important in the grand scheme were his two observations, firstly that an ``ought'' cannot logically be derived from an ``is'' and secondly that reason should be a slave of the passions.

\vbox to 1cm{\vfil ************* \vfil}

This I do through some foundational ideas about the \emph{representation} of knowledge.
Calling these ideas foundational is intended to suggest that some small kernel is to be offered in terms of which a fuller diversity of knowledge may be understood.

This includes the suggestion that \emph{declarative} knowledge, 
This includes the nomination of a preferred `foundation system', a precisely defined but informally presented family of logical systems.

A full description of this system depends upon an appropriate philosophical context, which is itself foundational in a similar sense, and must therefore be provided in the account, together with counter-arguments to some of the skeptical claims likely to be offered against it.

In addition to these presentations defining and underpinning the suggested epistemic foundations, I hope to provide grounds for belief in the practical importance of adopting some such uniform foundation, and suggest in this historical context in the evolution of intelligence the adoption of such a system (within a generation) is nigh likely.

The ideas are set in the context of an account of our situation at a crucial point in multiple evolutionary progressions contributing to the evolution of intelligence, among which the evolution of knowledge and epistemology feature prominently.


