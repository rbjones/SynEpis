There are many factors which have contributed to the accomplishment and prosperity of \emph{homo sapiens}.
One of the more important must surely be the cultural aggregation of increasingly precise knowledge, enabling the development of agriculture and animal husbandry, and the engineering of useful artifacts and congenial habitats.

Knowledge comes in many forms, even in that tiny fragment of the universe which we might hope to comprehend.
In the microcosmos of planet Earth we are only just in a position, not only to accumulate knowledge, but to consider those diverse forms and the evident progressions in which they appear.

The diversity of knowledge across the universe exceeds human comprehension, but within the limits to which human intelligence has begun to understand, there may be reason to believe that the kinds of kmowledge which exist are progressing, and that we are now at the cusp of a particularly significant epistemic advance.
It is the purpose of this monograph to explore the nature and significance of that special advance and discuss its potential impact on the ways in which knowledge is gathered, organised and exploited.

Epistemology, the philosophical study of knowledge, dates back to the phi\-lo\-so\-phers of Classical Greece, the name derived from the Greek word for knowledge \greekfont{ἐπιστήμη}
 (\emph{epis\-tēmē}).
This monograph, which I have styled an `epistemological synthesis', is constructively concerned with the nature of knowledge.
It constitutes, not an examination of some epistemological \emph{fait accomplit}, the true nature of knowledge, but rather the engineering of an epistemological \emph{desideratum}, how we might do well to think of knowledge.

\section{What is Knowledge}

Sometimes philosophers view knowledge from a narrow anthropocentric perspective, epitomised by the controversy over whether knowledge is justified true belief.
This particular perspective prioritises the question of what constitutes `justification', and fuels debate about whether knowledge is possible.
In its broader usage, in philosophy and beyond, knds of knowledge are more diverse and the usage of the term is malleable.
I ordinary usage one may be said to `know' a fact having once been told by a source who, though by no means an authority, is not known to be a fraud.
Beyond that, in the most distant reaches of our evolutionary history, the most primitive organisms responding appropriately to an environmental stimulus, might be said to have made use of environmental knowledge acquired through sensory mechanisms.

\section{What is Knowledge}

The usage of the term \emph{knowledge} in this monograph will be loose and wide ranging.
In this introductory discussion, 


\section{}

Among the many forms which knowledge takes, declarative knowledge, primarily consisting in that kind of knowledge which can be expressed or represented by indicative sentences (or \rbjdef{declarative language}, for which a \emph{definition} will be forthcoming), is relatively recent, having its beginnings probably about the same time as anatomically modern \emph{homo sapiens}.

Such knowledge has characteristics derived from those of declarative language, and which vary widely according to the domain of discourse.
In some domains, such as mathematics, it can be extremely precise in meaning, with deductive methods of confirmation which are highly reliable.
In others, particularly those reaching into the deepest recesses of the human mind or the highest achievements of art and literature, may be subjective in character, imprecise in meaning or uncertain as to truth.

Many will hold these latter truths more important than the dry inhuman theories of mathematics, but cultural refinement is a luxury which only those whose material needs are well catered for can enjoy.
It is mathematics and science, with their need for precision and certainty, which have served best to fulfill those material needs leaving to spare for culture.

\ignore{
That a foundation addressing precision of meaning and certainty of truth, in those domains most conducive to them, can benefit broader domains can be illustrated by considering three domains distinguished by the philosopher David Hume.
At the centre of Hume's philosophy was the distinction between propositions concerning \emph{relations between ideas} and those addressing matters of fact.
Foundationally more peripheral but probably no less important in the grand scheme were his two observations, firstly that an ``ought'' cannot logically be derived from an ``is'' and secondly that reason should be a slave of the passions.
}

Declarative language, it seems likely, has its origins around about the same time as anatomically modern \emph{homo sapiens}, at the culmination of a period of rapid growth in the size of the human brain which furnished us with the intellectual pre-eminince we now enjoy, as well as the necessary anatomical features to support the articulartion and comprehenshion of oral language, but the conception of declarative language which we now have is very much more recent and represents an idealisation of the natural languages which appeared then and have evolved since.

Natural languages themselves fall short of that ideal, but the development of mathematics gradually moved language towards the ideal in those domains where precision, objectivity and assurance were most needed, culminating in the deevlopment in the twentieth century of the formal logial systems which provide almost perfect realisations of the ideal as well as the metatheoretic advances permitting the ideal to be fully articulated. 

This I do through some foundational ideas about the \emph{representation} of knowledge.
Calling these ideas foundational is intended to suggest that some small kernel is to be offered in terms of which a fuller diversity of knowledge may be understood.

This includes the nomination of a preferred `foundation system', a precisely defined but informally presented family of logical systems.
A full description of this system depends upon an appropriate philosophical context, which is itself foundational in a similar sense, and must therefore be provided in the account, together with counter-arguments to some of the skeptical claims likely to be offered against it.

In addition to these presentations defining and underpinning the suggested epistemic foundations, I hope to provide grounds for belief in the practical importance of adopting some such uniform foundation, and suggest in this historical context in the evolution of intelligence the adoption of such a system (within a generation) is nigh likely.

The ideas are set in the context of an account of our situation at a crucial point in multiple evolutionary progressions contributing to the evolution of intelligence, among which the evolution of knowledge and epistemology feature prominently.


