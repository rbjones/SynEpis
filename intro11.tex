There are many factors which have contributed to the accomplishment and prosperity of \emph{homo sapiens}.
One of the more important must surely be the cultural aggregation of increasingly precise knowledge, enabling the development of agriculture and animal husbandry, and the engineering of useful artifacts and congenial habitats.

The development of that culture depended upon the evolution of languge, and upon the growing and coherent body of general knowledge apploicable to particular circumstances and problems by inference from observed generalities to relevant particulars.
This worked because declarative language approximates to aspects of an ideal which Isaiah Berlin sketched as `three pillars of Western Civilisation' in his analysis of the roots of romanticism \cite{berlinRR}.
