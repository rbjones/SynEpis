This monograph wraps an evolutionary tale around an epistemological synthesis.

The epistemological synthesis is built upon a logical core, first engineered for narrow mathematical purposes (though not without wider ambition), but here re-purposed as a universal substrate for the representation of knowledge.
It is presented as \emph{foundational} in providing a primitive base for the largest of superstructures, and hence its advocacy is a variant of the much derided philosophical tendency to \emph{foundationalism}.
The philosophy constructed around it here may also be seen as a reversion to the ancient Aristotelian pursuit of \emph{first philosophy}, the idea that philosophy has a contribution to make which is in some sense prior to and more general than the sciences.
In this case, that the aggregation of a single shared body of coherent knowledge about the world we live in is a desirable end and demands an appropriate foundation for the representation of knowledge which is common to all the sciences (and beyond).

What has evolution to do with this?
There are two evolutionary stories which contribute to understanding the nature of the proposed logical foundations and the thesis I present about their importance for the future.

The first is the story which has lead to the emergence of the logical systems in question, which represent, I suggest, a kind of terminus beyond which futher developments of a foundational nature are likely to be marginal (though we are just at the beginning of finding the best ways to build on these foundations).
This is a long story from which I present just those elements which are most signficant in leading to the supposed foundational terminus.

The second is an even more tenuous tale of how the future will unfold, and the evolutionary imperatives which will ensure that the identified foundational systems will become the dominant paradigm for rhe representation of knowledge.

This is not science, the principle ideas promoted here are not such as are susceptible to conclusive demonstration, and do not present clear opportunities for falsification.
Though this kind of criterion for solid scientific truth has an important place in science, life normally presents us with choices which have to be made on more tenuous grounds, and philosophy probes beyond the uncetainties of everyday life into matters less easy to comprehend and susceptible to greater doubts.

\section{Evolution and Epistemology}

I have just offered evolution as an explanation of the origin of the logical foundations at the core of this synthesis.
