This monograph wraps an evolutionary tale around an epistemological synthesis.

The epistemological synthesis is built upon a logical core, first engineered for narrow mathematical purposes (though not without wider ambition), but is here re-purposed as a universal substrate for the representation of knowledge.
It is presented as \emph{foundational} in providing a primitive base for the largest of superstructures, and hence its advocacy is a variant of the much derided philosophical tendency to \emph{foundationalism}.
The philosophy constructed around it here may also be seen as a reversion to the ancient Aristotelian pursuit of \emph{first philosophy}, the idea that philosophy has a contribution to make which is in some sense prior to and more general than the sciences.
In this case, that the aggregation of a single shared body of coherent knowledge about the world we live in is a desirable end and demands an appropriate foundation for the representation of knowledge which is common to all the sciences (and beyond).

What has evolution to do with this?
There are two evolutionary stories which contribute to understanding the nature of the proposed logical foundations and the thesis I present about their importance for the future.

The first is the story which has lead to the emergence of the logical systems in question, which represent, I suggest, a kind of terminus beyond which further developments to such logical foundations are likely to be marginal (though we are just at the beginning of finding the best ways to build on these foundations).
This is a long story from which I present just those elements which are most signficant in leading to the supposed foundational terminus.

The second is a tale of how the future may unfold, and the evolutionary imperatives which will ensure that the identified foundational systems will become the dominant paradigm for rhe representation of knowledge.

This is not science, the principal ideas promoted here are not such as are susceptible to conclusive demonstration, and do not present clear opportunities for falsification.
Though this kind of criterion for solid scientific truth has an important place in science.
Life normally presents us with choices which have to be made on more tenuous grounds.
Engineers routinely work with models of the real world which are known to be adequate applicable approximations, such as the frameword of Newtonian physics now known to be false.
Philosophy probes beyond the uncertainties of everyday life into matters less easy to comprehend and susceptible to greater doubts.

The main substance of this monograph is offered as philosophical, as a contribution to epistemology.
That epistemology is conceived of as the the foundation of a broader philosophical system, which will be lightly sketched.
Though foundational in character, it is only possible to understand and articulate this system in the context of the evolutionary story which has lead to its core components.
The projection of that story into the future is an important motivator for the system, and a principle ground for the speculation that proposed system will eventually prevail.

At its most portentous, this is the story of the evolution of intelligence, and will become the story of its proliferation through the cosmos, in consequence of an evolutionary imperative which defies stasis.
These exotic and speculative ideas about the future of intelligence have an important role to play in expounding and motivating the epistemological synthesis, but are colourful accessories to the primary purpose of this monograph which is philosophical.

\section{Sketch of a Sketch}

It is my aim to provide in this introduction a most concise sketch of the core elements of this proposal, leaving for the remainder of the monograph to fill in and fill out that picture.

Here I preview what those core elements are.

\begin{itemize}
\item Remarks on the Foundationalism
\item
\end{itemize}

\section{First Philosophy and Foundations}

I am in this monograph seeking to practice, a foundationalist first philosophy.
From a lomg evolutionary perspective, I see us as part of an evolutionary trajectory which from the highest level may be seen as the evolution of intelligence and its proleferation across the galaxy and into the cosmos.
That whole process consists of and is fuelled by the aggregation of knowledge on a massive and accelerating scale.

Because this is a kind of evolutionary process we can to some extent anticipate how it will go, and since it is reasonable to 


First a key point about the foundational nature of this proposal and my  to `first philosophy'.
This foundationalism, if it be called that, does not rest on absolutes.
It does not regard any truth as absolutely certain, any statement completely precise, or any account of semantics absolutely unambiguous.
But I will sometimes talk as if that were the case.
The precision with which the proposed semantics allows abstract concepts to be defined and truths about those concepts to be established is extraordinary by comparison with any prior system.
Consequently, I will occasionally make (perhaps oxymoronic) qualified claims to absolutes, using words like `quasi'-universal or `practically'-complete.
If ever I should fail to include such qualifications to what appear to be absolute claims, then the reader should supply his own pinch of salt.

More directly pertinent to the idea that this is \emph{first philosophy} are the following divergencies from the methods of first philosophies most famous practitioners.


\begin{itemize}
\item It is Aristotle who coined the term `first philosophy'\footnote{If anyone can be credited with coining a term in a language yet to be devised}%
 in the volume which was later to be named `Metaphysics'.
  In this he conceived first philosophy as addressing those matters which were independent of particular sciences, and hence,in some sense prior to and more general than them.
  The most important of these issues was the nature of substance, the features of all that exists, rather than those particular to the subject matters of the individual sciences.
  The foundational stance proposed here regards ontology as \emph{conventional}, most especially abstract ontology, in which abstract entities have just those properties which we ascribe to them, provided onlt that the attributed properties are logically consistent.
  This doesn't mean that the choice is wholly arbitrary, and we will discuss a very small number of options for abstract ontology in our foundations among which a choice can be made on pragmatic grounds, for the choices do have practical implications.

  Despite divergence on exactly what problems should be addressed by first philosophy, the foundationalism here adopted remains very much in the spirit of Aristotle in holding that there are important matters to be addressed by philosophers which are in some sense prior to and providing support for the conduct of the sciences.
  
\item Descartes' `Method of Doubt' is the other best know example of \emph{first philosophy}.
  This method consists in doubting everthing that can be doubted, and then progressing from the slender remains by deductive reasoning.
  Descartes found he could doubt everything but his own existence, and somehow managed on that basis to deduce that there is a god who is constutionally capable of allowing a world in which anything perceived clearly enough could possibly be false.

  To understand the foundationalism presented here and the arguments with which I support it you must first appreciate the distinction between object and metatheory, closely related to that between an object language (one about which we are talking, perhaps defining) and a meta language (the distinct language in which our discussion takes place).
  It is the object system which is the proposed foundation system, and it is our definition which begins with nothing and procedes to the simplest system which can provide an adequate foundation.
  In doing this, we use a well established meta-language, together with a handful of neologisms necessary to present a small number of novel features or ideas, and some knowledge of the 4 billion years of evolution which has brought us to this point, with very particular attention to the evolution of the language of mathematics and the quite recent reduction of mathematical reasoning to logical foundations.
  So, this foundational first philosophy is not conducted in a vacuum or on a tabular rasa.

\item The foundationalism advanced here is broadly similar to the logicism which sought to reduce mathematics to logic, and builds upon the technical advances which came with that project, but continued to progress after the core idea fell out of favour.
  However the scoping is explicitly more broad, the proposed foundations offered as a (quasi-)universal basis for abstract semantics, thereby providing a basis for deductive reasoning about knowledge of all kinds rather than specifically underpinning mathematics.
  
\end{itemize}


\section{What is Knowledge?}

Epistemology is the theory of knowledge.
Later I will talk about some of the various kinds of epistemology in order to make clear the synthetic epistemology which I am attempting in this monograph, but for this introduction, doing a bit of synthetic epistmology by talking about knowledge will suffice.

In the broad conception of knowledge which concerns me here,\footnote{Which encompasses the whole of semiotics.} knowledge is a natural phenomenon in which one physical structure carries information about some other aspect of the real world.
Often the relationship is underpinned by a causal connetion between the representation and the structure represented which results in the state of the representation reflecting the state of the causing system and therefore conveying information to anyone observing the relationship who understands how that relationship works.

In some cases this causal relationship is very direct and obvious, for example in perception, when a causal chain from the perceived structure mediated by sense organs results in modifications to synaptic weights in the brain of the observer which represents information about the perceived structure.
If the perceiving person then writes an account of his experience, we have a very different representation of information about that structure, and a more complex and uncertain causal relationship between the two.
This kind of knowledge, in the brain, is known as `explicit', and its appearance in language as 'declarative'.
Declarative knowledge, knowing that, may be distinguished from procedural knowledge, knowing how.
Its representation, again in synaptic connections and weights, is a physical system which effects a useful behaviour in response to a suitable physical context, which result in some appropriate physical activity.
Even where our knowledge may be of how to accomplish some mental gymnastics, the result will at least have some physical representation in the brain, and most likely will be transcribed into some other physical medium.

Having emphasised the physical nature of both the represenatation of knowledge and its subject, I should aknowledge that, increasingly as human civilisation has advanced, important bodies of knowledge are most naturally seen as speaking about non-physical abstract matters.
This is a natural consequence of the pragmatics of generalisation, in which patterns applicable to many distinct physical situations are captured without reference to any specific instance of their applicable instances.
Naturally, when we, as epistemologists, talk about these phenomenon, the inevitable (desirable and intended) generalisations and abstraction, will tempt us into considering abstractions from the concrete representations and attribute the representation of knowledge to abstract forms.
Nothwithstanding that the representation and the subject will not allways be physical, it will be useful in this discussion to speak as they always are so that we can focus the following talk about abstractions on the relationship between representation and subject matter, which I will speak of as `semantics' however distant it might feel from linguistics.

Semantics is the relationship between knowledge and its subject matters, and the existence of some such relationship is the crucial condition for some physical structure to represent or constitute knowledge.

This concern with the representation of knowledge has not been central to most philosophy, which additional criteria about the fidelity of the representation and the grounds for belief in that fidelity (justification) have been central, and have lead to some philosophers regarding that those criteria can never be realised and that knowledge is therefore unattainable, and others to devote considerable time and energy to the refutation of scepticism.
This we may think of as a debate about a particularly fastidious conception of knowledge, while usage in the broad sweep of everyday contexts is much more accomodating.
In this context, the conception of knowledge as requiring not only faithful representation but conclusive evidence of that fidelity is a narrowing of the scope of knowledge which would be fatal to the kinds of evolutionary argument which I will in the course of this monograph advance for the future preponderance of informations systems built around the representations of knowledge which this epistemological synthesis advances.

It is now time to talk about semantics, the role of abstract ontolgy in precisely describing the semantic relationships between knowledge and its subject matter, and hence the role of abstract structures in the representation of knowledge.

\section{Semantics and Abstraction}

\section{The Structure of this Monograph}

The epicenter of this epistemological synthesis is the idea of a quasi-universal logical foundation system, and the exposition has three principle aims:
\begin{enumerate}
\item To define and explain the term `quasi-universal foundation system'
\item To present a preferred quasi-universal foundation system, together with support for the claim that it is one.
\item To explain how such a system provides a basis for a distributed shared knowledge base which is suitable for use by an inteligence network expanding across the galaxy and beyond.
\item To motivate the proposed foundations through an account of their origin, and a projection of their role into the future, assessing their role in the light of the evolutionary imperatives which will govern which kinds of intelligent systems will predominate as transgalaxic proliferation progresses.
  \end{enumerate}
  
  
  

  
\ignore{

\subsection{Epistemic Perspectives on Evolution}

\subsection{Intelligence and Episteme}

\subsection{The Evolution of Intelligence}
}

