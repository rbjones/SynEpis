
The broadest context for this epistemological synthesis is a perception of the Earth as one point of origin%
\footnote{It would be surprising if Earth were the only place where intelligence has evolved, but we know of no other, so the proliferation of intelligence is spoken of here as terracentric, and having the same roots as \emph{homo sapiens}.
Some limited aspects of the synthesis will address interaction with intelligence from other sources, at least addressing the coherent merging of coherent bodies of knowledge.} %
 for the evolution of intelligence and its proliferation throughout the Cosmos.

The idea of intelligence includes the ability to gather knowledge and apply it effectively, and the evolution of life on earth provides a progression of capabilities of that kind, beginning with simple stimulus response mechanisms and advancing ultimately to the capabilities found in \emph{homo sapiens} for discovering scientific knowledge of the world and applying it in the development of technology and the engineering of habitats, tools and machinery, transforming the well-being and reproductive success of \emph{homo sapiens}.

We may see in that evolutionary process the gathering of knowledge little by little about what organisms reproduce successfully in each environmental niche and the incorporation of that knowledge as DNA into the genomes of the species.
But \emph{homo sapiens} is now rapidly breaking the bounds of that perspective.
With our ability to engineer cogenial habitats in hostile environments, the idea of adaptation to a particular enironmental niche may now have been eclipsed, except insofar as the Earth as a whole constitutes a single niche in the greater Cosmos.

Can our engineering prowess take us out of even that niche?
So many believe, and there are many aspects of our developing knowledge and technology which are now breaking bounds.
After a brief period of advancement in information technoogy, cultural evolution is on the brink of engineering artifacts which surpass human intelligence, and which already contribute substantially to successive design increments aong that trajectory.
At the same time knowledge and technologiy which will transform biological evolution into a similar cycle of increment by design which will bring biological evolution to a similar pace as cultural evolution.
Alongside these radical pace changing advances in knowledge and innovation, the ambition to spread humanity across the solar system and into the galaxy beyond isseizing the imagination, and the artificial intelligence and robotics which will be necessary to that end has any imagination, it will see that the spread of intelligence across the galaxy will best be lead by intelligent progeny of humanity which have broken free of the contraints imposed by biology.

In the continuing evolutionary competition for preponderance through proliferation, knowledge will be gold-dust, and the way in which that knowledge is aquired, shared and applied will be refined by those evolutionary imperatives.


\cite{gordon1993}


