
The broadest context for this epistemological synthesis is a perception of the Earth as one point of origin for the evolution of intelligence and its proliferation throughout the Cosmos.

It would be surprising if Earth were the only place where intelligence has evolved, but we know of no other, so the proliferation of intelligence here is thought of as terracentric and having the same roots as \emph{homo sapiens}.

The idea of intelligence includes the ability to gather knowledge and apply it effectively, and the evolution of life on earth provides a progression of capabilities of that kind, beginning with simple stimulus response mechanisms and advancing ultimately to the capabilities found in \emph{homo sapiens} in discovering scientific knowledge of the world and applying it in the development of technology and the engineering of habitats, tools and machinery transforming the well-being and reproductive success of \emph{homo sapiens}.

At the same time, we may see in that evolutionary process the gathering of knowledge little by little about what organisms reproduce successfully in each environmental niche and the incorporation of that knowledge as DNA into the genomes of the species.

\cite{gordon1933}

