The purpose of this epistemological synthesis is to present a particular family of logical systems, and one special member of that family, as suitable for the representation of knowledge in large, shared, distributed repositories of knowledge.

There are two primary rationales supporting this idea.
The first is that the family of systems in question may be said to be quasi-universal for the representation of declarative language.
The second is that the identified special member of that family has special characteristics which make practically convenient as a universal representation.
There is another member of the family which will also be helpful in the exposition, by virtue of being, if not quite so convemient in practical application, simpler for the exposition of the philosophical basis of the case.

Life and culture, as it has evolved on earth so far, is in all probability shaped in its detail by the environment in which it evolved, and by the limitations of the evolutionary process in demanding that all change is built from small increments each advancing reproductive fitness.
We may doubt that life, evolved in this way, could be optimal for proliferation beyond earth, and hence wonder how we might anticipate the demands of such.
But we see nevertheless, that certain simple ideas despite their arising in the course of this particular evolutionary trajectory, must surely, because of their simplicity and general utility, arise in any other evolutionary process which progresses intelligence.

Mathematics and Logic are a special haven for such ideas, and it might be argued that the whole of mathematics constitutes such general ideas not bound to the particularities of planet earth.
I will not press that case, but rather focus on certain key ideas which have appeared in the work of philsophers working to improve the foundations of mathematics, which combine simplicity approaching that of the natural numbers, but the greater expressive power needed to underpin the more advanced number systems devised to support modern science with its numerical precision in its support of diverse applications.
These appeared after the fact, relatively late in the developmemt of the field of mathematical analysis, when, despite demonstrable applicability of the theories mathematicians became concerned about the clarity of the underlying ideas (notably the use of number systems involving infinitesimal quantities) and the lack of rigour in reasoning about these new mathematical domains.

As the nineteenth century progressed, the rigorisation of analysis through the formulation of key concepts without use of infinitesimals, the construction of the required (``real'') number systems from natural numbers to the reduction of the natural numbers to logic by Frege and his philosophical maxim that ``mathematics = logic + definitions'', declaring that mathematics properly conducted, consisted wholly in the derivation of the logical consequences of the definitions of the relevant mathematical concepts.
The logical foundations devised by Frege for this purpose proved to be inconsistent, but were soon followed by coherent  systems devised for similar purposes, subsequently refined and simplified.

==========

The core idea is that declarative languages can be ordered according to expressiveness, and that a language A is as expressive as language B if the sentences of B can be mapped into those of A in a way which preserves meaning.
Of course, to make that definite details need to be definedd, like what is a declarative language and what mappings are deemed to preserve meaning.
The the suggestion is that there are universal languages, and we can identify one which is particularly suitable for use as a representation system for knowledge.
Note that this notion of ordering is reflexive, assuming that the identity mapping is meaning preserving, so its not the same as the relationship ``A defines the semantics of B''.

Now there is a problem in this, which is usually associated with Alfred Tarski.
