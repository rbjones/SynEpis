As we move from an era in which the intelligence of \emph{homo sapiens} has reshaped the world and enabled this species to flourish and dominate the ecosphere into an uncertain future in which the talents of humans may be eclipsed by artificial intelligence, one thing seems still probable.
That is, that the accumulation of declarative knowledge will continue to be a crucial enabler for continued progress.

Indeed the emergence and expected growth in intelligent information systems will surely cause exposive growth in the knowledge base which informs the future.
At the beginning of the 19th Century there was a crisis in mathematics as a result of declining standards of rigour which had resulted from practical pressures to adopt analytic methods which were fruitful but poorly understood.
Throughout the 19th Century progressive efforts were made to reconstruct mathematical analysis in a semantically clear and logically rigourous manner, culminating in fundamental theories and logical systems which overturned a status quo substantially stable (particularly in logic) since Aristotle.
The formal logical systems then devised provided greater semantic clarity, and ultimate standards of rigour at the cost of to great a burdon of detail for practical use by working mathematicians.
