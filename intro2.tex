
Knowledge comes in many forms, even in this tiny fragment of the Universe which we might hope to comprehend.
In the microcosmos of planet Earth we are only just in a position not only to accumulate knowledge but to consider those forms and the evident progressions in which they appear.

The diversity of knowledge across the universe exceeds human comprehension, but within the limits to which human intelligence has begun to understand there may be reason to believe that the kinds of kmowledge which exist are progressing, and that we are at this ``moment'' at the cusp of a particularly significant advance.
It is the purpose of this monograph to explore the nature amd significance of that special advance.

Epistemology, the philosophical study of knowledge, dates back to the phi\-lo\-so\-phers of ancient Greece, the name derived from the Greek word for knowledge, \greekfont{ἐπιστήμη} (episteme).

Declarative or propositional knowledge is the kind of knowledge expressed in or represented by indicative or declarative sentences (those which may be true or false) in the kinds of language natural to human societies.
Such languages in their oral forms are probably coeval with homo sapiens, but it is not until much later, after the invention of written language and convenient media, that we could expect any historical record of epistemological thinking.
This we find in the ancient Greek philosophers, whose epistemological thinking may have been provoked by the advances they made in turning mathematics into a theoretical discipline by the systematic use of reason, and the failures they experienced in applying those methods more broadly.

Those conflicting experinces of the effectiveness of reason may be attributed to the relationships found between those distinct applications and an idealisation of propositional knowledge which is central to and foundational for the synthesis which I seek in this monograph.

