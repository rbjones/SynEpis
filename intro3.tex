
The philosopher and historian of ideas, Isaiah Berlin, when seeking to describe the transition from the age of enlightenment to that of romanticism \cite{berlinRR}, put forward an account of three pillars upon which he supposed the entire Western tradition to have rested, supplemented by a further element which he attributed to enlightenment thought.
Without endorsing the possibility that the diversity of thought in the eighteenth century could credibly be captured within such a schematic, I nevertheless find in it a starting point for an analysis of propositional language which serves my purposes in this epistemological synthesis.

Berlin's three pillars were:

\begin{enumerate}
\item That all `genuine' questions have answers.

\item That the answers can be known (not clear how or by whom)

\item That the true answers to genuine questions are all logically compatible.
\end{enumerate}

The particular element added by enlightenment thought was that the answers are to be found by \emph{reason} (``deductive or inductive as appropriate to the subject matter''), rather than by revelation, tradition, dogma or introspection.

The suggestion that there might have been a long standing concensus along these lines is refuted at every turn by the philosophical controversies which have raged since the time of Thales, between radical sceptics and the `dogmatists' they opposed, later between philosophical tendencies such as \emph{rationalism} and \emph{empiricism}, and among innumerous more subtle positions and reconciliations.

Nevertheless, there may be behind Berlin's intuition something fundamental and transformative which comes with and underpins the utility of propositional language and deductive reason.
I suggest here that the particular utility of propositional language in enabling the development of human culture and civilisation and in lifting \emph{homo sapiens} above the subsistence level, comes from its approximation to an ideal around which philosophers have studied the nature of knowledge and the ways in which it might be established.

The signficance of this moment in the history of ideas and of this particular conceptiom of the fundamental issues at stake, for my story, is that it consitutes an important milestone in the development, application and analysis of propositional language leading up to a major transition in the midst of which we now find ourselves.

\section{A Historical Sketch}

Some structure:

\begin{itemize}
\item The beginnings of life, DNA: 3 Billion years ago.
\begin{itemize}
\item Cells -> Organisms -> Animals.
\item The central nervous system: 500 Million years ago
\item The evolution of memory.
\end{itemize}
\item Propositional Language: 300,000 years ago
\item Quasi Universal Logical Systems: contemporary.
\end{itemize}

The place where we now find ourselves, from the perspective which I present here, may best be understood if presented as the outcome of an evolutionary and a historical process.
A first step toward describing that process is to mention the major landmarks on the route.
Like many things in the evolution of life on earth, progress along this trail seems to have accelerated, substantially and rapidly, so that our landmarks are very far from evenly spaced.

The logical center point around which the narrative hangs is the evolution of \emph{propositional} or \emph{declarative} language, the kind of language which is found primarily in indicative sentences of natural languages, whose 



While demurring from further analysis of the merits of Berlin's story as an account of enlightennent thought and its transformation by romanticism, related questions concerning the nature and value of propositional language and the use of reason, particularly deductive reason, is central to my synthesis.
Though propositional language probably dates back some 300,000 years, it was not until the philosophers of ancient Greece, maybe 3,000 years ago, that its analysis began.
The sucesses of systematic deductive reason in Greek mathematics and its dismal failures in most other domains provided a taunting enigma for philosophers seeking to underpin their theories with the authority realised by Axiomatic Geometry.
