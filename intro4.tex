The purpose of this monograph is to present some ideas on knowledge representation and management.

Since it is primarily philosophical in character and is concerned entirely with knowledge, it is natural for me to think of it, and to present it, as \emph{epistemology}.
I also think of it as \emph{architectural design}, at its earliest and most abstract stages, of certain aspects of knowledge management systems, the structures which such systems manage.
It is nevertheless primarily the former, and as such I have titled the monograph `Synthetic Epistemology', which I will expand upon shortly.

A principle aim of the ideas is to secure precision of meaning in the representation of knowledge and confidence in the results of deductive reasoning in the context of a body of such knowledge, for which the maintenance of logical coherence in such bodies will be considered essential.\footnote{So called \emph{non-monotonic} reasoning and relevance logics al within the scope but are not foundational, insofar as they have well degined semantics they can be supported in a system based on classical, two-valued, foundations.}

Despite this emphasis on precision of meaming, much of the language used in the exposition will be broadly and imprecisely construed, as befits terms which are drawn from natural languages.






Epistemology permeates philosophy, the boundaries between it and other philosophical disciplines is porous, and in epistemology proper we find only those epistemological concerns which are not local to some other branch of philosophy.
There are many other parts of philosophy that are vitally concerned with various aspects of theory about knowledge, and into which I am likely to trespass.
Among the more important of these are the philosophies of language, logic, mathematics, science and artificial intelligence.

At a time of rapid progress and high expectations in the engineering of artificial intelligence, the relevance of knowledge representation and knowledge manageent to those developments is certainly an important motivator for these ideas.
This has not been prominent in the recent major advances, but if not necessarily figuring in the essentials for engineering intelligence, I believe is is desirable (and inevitable) in their application.
My sense of the inevitability comes from observation of the history and evolution of knowledge, epistemology and related disciplines over the last three billion years, culminating not only in the present successes in engineering intellience, but also in developments in language and logic reevant to the representation of knowledge.

----


The central features of the ideas presented here are that:
\begin{itemize}
\item A single widely distrubed, consistent and geneally accessible repository of knowledge underpinnied by a single abstract representation with a minimal primitive vocabulary, a clearly defined semantics and a sound but strong formal inference system.
\item Lingistic pluralism in presenting the underlying representation in forms convenient for a wide variety of users (human and machine) and pusposes.
\item Rules for the expansion of the primitive vocabulary in verificably consistent ways by conservative extension, which underpin the effective linguistic pluralism by semantic embedding.
\item Heirarchical naming rules which uniquely assign the namings used by different extensions to single authorities.


\end{itemize}


It is a philosophical contribution to a problem of engineering design, given that engineering has now advanced so far as to design and construct cognitive artifacts, machines that know.

The design of knowledge bases has tradtionally been an important part of research on \emph{artificial intelligence}, but though the research continues, it has been eclipsed by the achievements of large scale neural net simulations, which achieve their results independently of that research.
