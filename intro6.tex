\section{Foundational Philosophy}

The central purposes of this monograph are two.
The first is to present some ideas about the representation of knowledge, constituting a suggestion as to how this might be done.
The second, in the course of explaining and underpinning those ideas, is to present the most fundamental elements of a philosophical system, encompassing epistemology, metaphysics and the philosophies of language and logic. 
The ideas are to be made intelligible (to such extent as is within my competence) by a skeletal evolutionary history seeking to expose the key features of the system as an almost inevitable outcome of continued evolutionary progression.\footnote{Not necessarily progress!}

As well as viewing knowledge and epistemology\footnote{The philosophical theory of knowledge.} through an evolutionary lens, I will present evolution itself as substantially epistemic in character, as involved in the aggregation of knowledge (broadly construed), and as itself evolving.

In thinking of evolution as an epistemic phenomenon going back beyond the earthly origins of life and forward, into the proliferation of intelligence across the galaxy, some stretching of our concepts of evolution and knowledge is needed.
In this introduction I will first sketch the broad conceptions of evolution, knowledge and epistemology adopted, and then the highlights of that evolutionary trajectory which takes us from the beginnings of life on earth through to our most fundamental contemporary epistemological understandings, upon which the ensuing proposal for knowledge representation is based.

\section{Evolution}

In Charles Darwin's seminal work\cite{darwin-oos} the origin of species is attributed to a process, which we now know as evolution, consisting in descent with modification guided by natural selection.
Later refinements of this idea arise primarily from advances in scientific knowledge of the mechanics of inheritance (genetics) and tie the concept more narrowly to the kinds of biological evolution we see here on earth.
Despite the focus of those modern evolutionary syntheses, the idea of evolutionary progression has found a much wider currency, most conspicuously in ideas about cultural change, but also in relation to technology, language and even the cosmos as a whole. 

In those broader contexts, to talk of evolution may mean little more than gradual change.
I'm looking for a conception intermediate between those two, broader than Darwin's and the modern syntheses (particularly in extending beyond biological evolution), but definite enough to provide a basis for some high level reasoning about evolutionary outcomes.
To that end I propose to identify \emph{evolution} as a progression of changes involving proliferation with variations (of some kind of organisms or entities), whose outcome is primarily determined by the principle that features conducive to most rapid proliferation will ultimately prevail, numerically: proliferators ultimately predominate.
That a process is evolutionary in this broad sense may support cautious inferences about evolutionary outcomes, if a general claim can be supported about factors affecting proliferation.
Cuation is particularly necessary in concluding what cannot evolve\footnote{E.g. that altruism cannot evolve.}, and arises from the complexity of the ecosystems in which evolution takes place.

To emphasise the breadth of this conception I note that it does not require that there are self-replicating entities of any kind, and is therefore potentially applicable to prebiotic evolution where there is evolving differential proliferation of progressively more complex molecules eventually leading to the chemistry necessary to support primitive life forms (possibly via pre-biotic self-replication). 

\section{Epistemology}

Viewed epistemically, evolution may be thought of as a process of aggregation of knowledge about how to proliferate in the available environmental contexts or niches, which are themselves continuously transformed by that evolutionary process, by the differential replication which drives the evolutionary process (and which makes simplistic projections perilous).

For a sense of the evolutionary trends which are important here, we need a broad conception of knowledge, going far beyond the idea that knowledge is come kind of true belief and embracing more of the diversity which the concept of knowledge has in its general use beyond philosophy.
Examples of such use include the idea that the genomes of living organisms encode knowledge of how those organisms can be constructed and how the vital processes of life and reproduction proceed.
This is a highly persistent form of knowledge which is passed down through generations changing only at the pace of biological evolution.
By complete contrast, even the earliest forms of life in various degrees and manners responsive to their immediate environment, sensing aspects of that environment which they can use to improve their success in growing and replicating.
Any such process involves, we might informally claim, obtaining knowledge of those relevant aspects of the environment and responding accordingly, even though the material representation of that knowledge in early life forms will be no more than a fleeting state of a chemical pathway.
These two kinds of `knowledge' have very different characteristics, and considering them as falling in the scope of epistemology (which is convenient for the purposes of this monograph) draws into epistemology at least some aspects of the full range of the field of semiotics.


These contexts are not usually stable, the mere fact of one characteristic of some entity or that kind of entity itself becoming more prevalent changes the environment, if in no other way but increasing demand and competition for the resources those entities require to grow and replicate.
Over time, a sequence of such modest changes, each advantageous in its time, may effect a substantial transformation not only of its host, but of an entire ecosystem.

In this process there is a tendency to greater complexity in the entities concerned, indeed complexity is demanded for an organism to replicate



This epistemic perspective on evolution demands a broad conception of knowledge, not the singular focus on knowledge as 'justified true belief' credible enough among philosophers to have inspired debate within my lifetime, though the foundational emphasis upon which this monograph is built emphasises the importance in that broad scheme of foundations closely related to that narrow focus.
That broad conception of knowledge encompasses DNA not only as providing exact codings of the structure of the proteins necessary for life, but also less well understand information determining how the organism develops from embryo through childhood, reproductive adulthood into senescence and death.

The genome of each viable biological species carries the information about how to construct an organism capable of surviving in some particular environmental niche long enough to reproduce more or less similar organisms.
Cultural evolution aggregates knowledge primarily in language, preserved, transmitted and applied by a variety of technologies themselves the product of cultural evolution.

What is it that is evolving?
Many different aspects of life on earth have been evolving over the last four billion years.
For my present purposes a very high level perspective will be helpful, and over a timescale of about 6 billion years since the formation of the earth a high level view which connects the whole gamut of evolutionary currents to epistemological considerations is provided by the evolution and proliferation of \emph{intelligence}.
It is now thought to be about 4.54 billion years since the formation of the earth, which is the time-frame for evolution to reach the point at which the evolution of intelligence becomes evolution by intelligent design.
Of those 4.54 billion years, maybe one billion were pre-biotic evolution leading to the first living organisms, three and a half billion were biological evolution, and that 0.04 billion more than enough for the establishment and acceleration of cultural evolution, enabled by the biological evolution of `intelligence' in the large brains of genus homo.

The resulting intelligence transformed culture into an evolutionary phenomenon moving at a radically higher and hyper-exponentially accelerating pace.

It took around 4 billion years of evolution to deliver intelligence in the genus \emph{homo}, and more or less at the same time language and oral culture.
That moment was an important milestone in an accelerating evolutionary process, which took the primary method of evolution from being genetic to being cultural.
That cultural evolution has itself accelerated over the past 300,000 years, with milestones like the invention of agriculture and animal husbandry, the building of civilisations, the creation of written language, and the progression in written and other media for the storage, transmission and exploitation of knowledge.

Though what remains of that 5 billion year window is maybe one billion, the hyper-exponential acceleration seen over the long term in evolutionary progress suggests that the progression during that period will be more substantial than over the previous four billion years.

\section{The Evolution of Language}

The purpose of this monograph is to argue the case for a particular kind of \emph{foundational} knowledge representation, and to support that case within an appropriate foundational philosophical system.
The case is presented using an evolutionary account of the origin of this kind of language, extrapolated to suggest that the prevalence of the proposed kind of representation is virtually inevitable as the evolution and proliferation of intelligent systems progresses through its sixth billion years.

The evolution of language was an important discontinuity in our evolutionary history, since the pace of the cultural evolution which it enabled was much more rapid than that of biological evolution or its more pedestrian cultural predecessor, and would accelerate as bodies of cultural knowledge grew.
  Without language, knowledge of skills can only be passed by showing and observing, or learned afresh from experience.
Similarly, without language, knowledge of places and situations and the resources, opportunities, deficiencies or perils they present can only be obtained through personal acquaintance, not shared through culture.
  
Once language fuelled the evolution of culture, a number of important trends began which ultimately lead to the epistemic insights presented in this monograph.
Spoken language progressively accelerated the culutural aggregation of knowledge.
At first this progress would have been very slow.
Though it is not known when language first appeared, that language is supported in homo sapiens by innate mental capabilities (and cerebral structures) found in no other living species, suggests that its evolution took place during a period of rapid cerebral growth which came to an enddrew to a close with homo sapiens.

Throughout the quarter million years (or more) since homo sapiens evolved, it is likely that language and culture grew and refined together.
Just as in earlier stages of biological evolution, learning to cope with new environments demanded new kinds of perceptual discrimination and motor skills closely integrated with appropriate kinds of memory, later cultural advances demanded new concepts or the refinement of old ones for their collaborative progression and exploitation.

\section{The Evolution of Language}

In this sketch oriented towards the evolution of kmowledge and epistemology, we can divide the progress into two parts, those before and after the appearance of \emph{declarative language}.
There is no established understanding of exactly where this dividing line belongs, but its exact placement is not important to this narrative, so we may think of this as being approximately the point at which \emph{homo sapiens} appears, perhaps the time at which sufficient capabilities in the brain were realised, at the end of a million years of rapid growth in the brain.

A distinctive characteristic of declarative language is its connection with the concept of truth.
It is only with this that the classical conception of knowledge as some kind (the true kind)  of belief be comprehended.
Before language, the correspondences between neural structures and events could only be judged by their utility in the struggle for survival and reproduction.
With the benefit of language, beliefs can be expressed in sentences which have truth conditions, and by assessing compliance in reality we can assign a truth value to them (the question of justification will be considered later).

This transition is not an instantaneous flash at the point that declarative language appears.
Of course, such languages must be developed over time, so there is no point.
But also, biologists studying the evolution of memory find, notwithstanding my remark about the dependence of `truth' judgements on language, kinds of memory which they designate `semantic', `declarative' or `explicit'.
It seems that the kinds of discrimination found in language pre-date their expression in language.

When declarative language appeared, its development was guided by similar evolutionary imperatives as those memtal discriminations which preceded it. which served their various practical purposes, and which language was designed to communicate between individuals, across groups and through populations.
For the language to serve that purpose, it needed only to connect sufficiently well with those pragmatic issues, for which any precision in truth conditions might not at first be required.

It is only because of a particular thread in the subsequent evolution of language that we can discuss its origin in this way, coloured by a later analysis of this kind of language.
We may now say, that declarative language, throughout its evolution, may have worked well for the progression of communal knowledge throught the evolution of culture because it approximates (at first perhaps, rather poorly) an ideal which it is only possible to describe with hindsight.

One such attempt is that of the philosopher and historian of ideas, Isaiah Berlin, who in his account of the roots of romanticism speaks of three pillars on which 'the whole western tradition' rested up to the time of the enlightenment.
These were:

\begin{itemize}
\item
  \item
    \item
\end{itemize}
  



\end{document}
