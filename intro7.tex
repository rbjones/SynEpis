
This monograph makes an epistemological point in the context of a narrative about the evolution of intelligence.

The making of that point belongs to what I term \emph{synthetic epistemology}.
The point is, that a particular family of logical systems is a specially good representation system for knowledge, the general adoption of which would be advantageous, and may even be inevitable.
My \emph{making} of that point, beyond mere description of the system, falls into two main parts, an evolutionary account of its origin and a foundational philosophical kernel providing a context in which the supporting rationale may be undestood.

The `philosophical kernel' is an indivisible whole integrating metaphysics, language, logic and epistemology, motivated by the role it may play in the advancement of mathematics, science, technology and engineering and their use in the proliferation of intelligence.

The evolutionary story describes key aspects of that part of the origin of intelligence which began here on Earth a few billion years ago, progressing through this present moment of supposed epistemological insight into a future in which the progeny of earthly intelligent life spreads across the galaxy and beyond.
In this story, the past trajectory is intended to inform the readers understanding of how these logical systems arose and their nature, while the present and future trajectories say something about the significance of them in facilitating the advance of intelligence across the galaxy.

The introduction in this chapter to that story addresses two phases of the progess, the biological evolution of life on earth up to the point at which intelligent life in the form of \emph{homo sapiens} emerged, marking also the beginning of a period of accelerating cultural evolution.
Thatwhich culminates in that intelligent life realising inorganic intelligent systems by a process involving both cultural evolution and collaborative design.

There are two closely related divisions of the idea of knowledge which are important here.
They are, the distinction between knowing \emph{that} and knowing \emph{how}, and that beween \emph{declarative} and \emph{procedural} knowledge.

The classical characterisation of knowledge as \emph{justified true belief}, relates most clearly to declarative knowledge, but the idea of procedural knowledge embraces a much broader range of phenomenon in which some kind of learning process yeilds subsequent behavoural advantage.
The kinds of knowledge which consist in true beliefs depend upon beliefs having \emph{truth conditions}.
Such conditions are normally associated with meaningful notations, since the idea of truth conditions arose (very recently) in the analysis of the meaning of declarative sentences.

So we might expect that declarative knowledge goes back only as far as declarative language, while procedural knowledge goes back as far as behavioural adaptation.
Behavioural adaptation goes back to some of the most primitive organisms, and beyond them is found in the process of biological evolution, which may be said to code into the DNA of each species knowledge of proteins advantageous to that species.

Declarative language is much more recent, not known outside the genus homo and probably exclusive to homo sapiens, dating back no more than 300,000 years.






