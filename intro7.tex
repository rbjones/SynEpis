
This monograph makes an epistemological point in the context of a narrative about the evolution of intelligence.

The making of that point belongs to what I term \emph{synthetic epistemology}.
The point is, that a particular family of logical systems is a good representation system for knowledge, the general adoption of which would be advantageous, and may even be inevitable.
My \emph{making} of that point, beyond mere description of the system, falls into two main parts, an evolutionary account of its origin and a foundational philosophical kernel providing a context in which the supporting rationale may be undestood.

The `philosophical kernel' is an indivisible whole integrating metaphysics, language, logic and epistemology, motivated by the role it may play in the advancement of mathematics, science, technology and engineering and their use for the benefit of humanity and the proliferation of intelligence.

The evolutionary story describes key aspects of that part of the origin of intelligence which began here on Earth a few billion years ago, progressing through this present moment of supposed epistemological insight into a future in which the progeny of earthly intelligent life spreads across the galaxy and beyond.
In this story, the past trajectory is intended to inform the readers understanding of how these logical systems arose and their nature, while the present and future trajectories say something about the significance of them in facilitating the advance of intelligence across the galaxy.








The introduction in this chapter to that story addresses two phases of the progress.
The biological evolution of life on earth, up to the point at which intelligent life in the form of \emph{homo sapiens} emerged, marking also the beginning of a period of accelerating cultural evolution which culminates in the realisation of inorganic intelligent systems by a process involving both cultural evolution and collaborative design.
The continued evolution of intelligence may then be expected to proceed by design incrementation at a rapid pace.
The biological evolution of intelligent primates took around 4 Billion years.
Those primates will have designed intelligent artifacts in less than 400,000 years, and collaboration between those artifacts and their biological predecessors will yield progressively more advanced intelligent artifacts at an astounding pace.

The engineering of artifical intelligence has, for most of its history included research on knowledge representation, which has been eclipsed by the use of neural net related techniques storing knowledge in parameters broadly corresponding to the weights associated with synaptic connections between neurons.
These place intelligent artifacts on a similar footing to intelligent animals, but do not contribute to the continuous advancements in the languages of logic, mathematics and science which have brought us to the present day, and which may be thought to require continuing development to secure best advantage of the greater intellectual power which intelligent artifacts will deliver.

There are two closely related divisions of the idea of knowledge which are important here.
They are, the distinction between knowing \emph{that} and knowing \emph{how}, and that beween \emph{declarative} and \emph{procedural} knowledge.
These two kinds of knowledge seem very distinct.
The classical characterisation of knowledge as \emph{justified true belief}, relates most clearly to declarative knowledge, but the idea of procedural knowledge embraces a much broader range of phenomenon in which some kind of learning process yeilds subsequent behavoural advantage, without depending on claims which might be believed or justified.
Procedural knowledge must be assesed not in terms of truth, but rather utility, and in a context where no purpose can be assigned, such as the early evolution of life on earth, in terms of adpative advantage, in terms of its effect of reproductive fitness.

The distinction between procedural and declarative knowledge therefore seems stark, and insofar as one might be thought to superseed the other, one might search for the time and circumstance at which that happened, for which the most plausible candidate might be that point at which oral declarative langage appeared.
Closer comsideration suggests that no decisive transition took place, but rather there has been continuous but accelerating evolution progressing a number of significant metrics of which the most important for this discussiom is in the direction of semantic precision.
Though what linguists might 





The kinds of knowledge which consist in true beliefs depend upon beliefs having \emph{truth conditions}.
Such conditions are normally associated with meaningful notations, since the idea of truth conditions arose (very recently) in the analysis of the meaning of declarative sentences.

So we might expect that declarative knowledge goes back only as far as declarative language, while procedural knowledge goes back as far as behavioural adaptation.
Behavioural adaptation goes back to some of the most primitive organisms, and beyond them is found in the process of biological evolution, which may be said to code into the DNA of each species knowledge of proteins advantageous to that species.

Declarative language is much more recent, not known outside the genus homo and probably exclusive to homo sapiens, dating back no more than 300,000 years.
Despite that recent origin of those complex structured expressions which linguists recognise as languages, communications between the individuals of a species date back almost to the beginnings of life on earth, for we see social behaviour even in primitive bacteria, and all social behaviours depend upon some form of communication whereby individuals influence each other, perhaps to share information about sources of nourishment or dangers or to coordinate behaviours which may be mutually beneficial.

The evolutionary trajectory of greatest interest in articulating the merits of the proposed knowledge representation systems has as enpoints first, the very beginnings of life on earth more than 3 Billion years ago, and latterly, in just the last few decads, the articulation of the logical system here proposed.

Within a whisker of abiogenesis, the origin of life, evolution had crafted the genomes of single celled organisms which, for the sake of successful growth and replication, were responsive to their environments and capable of growing and reproducing in their own environmental niches.
We may think of the genome as incapsulating knowledge of \emph{how} to survive and reproduce, but it is hard to see how that genome can be construed as declarative knowledge.

The run up to realising intelligent artifacts involves substantial bodies of scientific and engineering knowledge, much of it built on mathematical models.
Though undountedly \emph{knowing how} plays and important role, declarative knowledge is a crucial ingredient.
