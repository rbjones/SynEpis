
This monograph makes an epistemological point in the context of a narrative about the evolution of intelligence.

The making of that point belongs to what I term \emph{synthetic epistemology}.
The point is, that a particular family of logical systems is a good representation system for knowledge, the general adoption of which would be advantageous, and may even be inevitable.
My \emph{making} of that point, beyond mere description of the system, falls into two main parts, an evolutionary account of its origin and a foundational philosophical kernel providing a context in which the supporting rationale may be understood.

The `philosophical kernel' is an indivisible whole integrating metaphysics, language, logic and epistemology, motivated by the role it may play in the advancement of mathematics, science, technology and engineering and their use for the benefit of humanity and the proliferation of intelligence.

This introductory chapter is intended to present the most intelligible account of the key features of the whole which can be stated in brief.

Philosophically, the work is foundational, and that foundation is a semantic foundation for declarative language and knowledge.
It is a family of closely related core languages with identical abstract syntax and minimal vocabulary and deductive system, over which arbitrary broader vocabularies may be constructed by conservative extension, a safe generalisation of introduction by definition (a definition being the assignment of a name to some entity already known to exist).
The members of the family of logical system differ only in the size of their smallest models, and the strength of the axiom of infinity which reflects that size in the axioms of the system.
Unlike the conservative extensions whereby a rich volcabulary can be introduced, these axioms associated with the ontologically richer systems do carry risk of incoherence, so there is some advantage in cleaving to the earlier systems in the series, and fortunately, practical applications will not demand controversial ontologies.

In calling these foundational, I am claiming that all declarative language can be interpreted in them.
For this to be a definite claim, I must of course give a definition of the concept of declarative language, and in doing so we run into problems of regress, both in relation to semantics (and hence truth) and also concerning proof.
These problems have no solution which is absolutely immune to scepticism.

The evolutionary story describes key aspects of that part of the origin of intelligence which began here on Earth a few billion years ago, progressing through this present moment of supposed epistemological insight into a future in which the progeny of earthly intelligent life spreads across the galaxy and beyond.
In this story, the past trajectory is intended to inform the readers understanding of how these logical systems arose and their nature, while the present and future trajectories say something about the significance of them in facilitating the further evolution of intelligence and its proliferation across the galaxy.

A useful first division of knowledge in general is into knowing \emph{how} (procedural knowledge) and knowing \emph{that} (declarative knowledge).
Procedural knowledge may be said to pervade the history of life on earth, being exhibited by all forms of life which surivive and reproduce.
It is important in a full picture of how knowledge has evolved and lead us to where we are now.
Declarative knowledge is much more recent, and is confined to those kinds of knowledge which are expressible in declarative sentences, which can be stated in a meaningful sentence which expresses a claim on some subject matter which will be true, or false.
Declarative knowledge we may therefore expect to have existed only since the kinds of language in which declarative sentences may be expressed, and to depend vitally upon the language being sufficiently well defined, in its meanings, that the truth conditions of declarative sentences are definite.

Declarative knowledge has been enormously important in securing for humanity its present levels of prosperity.
It evolved during or shortly after a million years of rapid growth in the size of hominid brains which plateauxed
