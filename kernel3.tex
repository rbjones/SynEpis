The philosophical context for the epistemological synthesis presented in this monograph is a foundational philosophy.
The kernel of the philosophy provides the foundations upon which the whole is based.

The kernal is epistemological, but its description involves other aspects of philosophy which may also be thought of as foundational, including metaphysics (mainly ontology), and the philosophies of language and logic, all of which are involved in providing a description of the most fundamental concept, that of \emph{abstract logical truth}.

\section{Methodological Preliminaries}

A foundational approach to some problem domain consists in discovering some smaller and/or simpler domain (the foundation) to which all the problems in the more complex domain are reducible.

The classic conception of knowledge holds that for a true belief to count as knowledge, it must be supported by a conclusive justification, from which it would follow that a foundation to which knowledge is reducible should itself be beyond doubt.
This philosophy and the kernel on which it is based, is skeptical of any absolute claims, it does not assume that it is ever possible to assign meaning to language with absolute precision, or that any sentence can be known to be true with absolute certainty.
Fortunately we do not need such absolutes, life goes on without them.
Nevertheless, there are very great differences in the precision of language and the certainty with which truth can be ascertained.
There are variations from one domain to another, and within a domain the language may evolve to greater expressiveness and precision with the benefit of experience, and the methods for ascertaining truth can often be progressed to increase their reliability and reach.

This philosophical kernel is primarily concerned with \emph{logical foundationa}, which may now have been progressed close to a limit in semantic expressiveness, certainty of proof, and completeness, though in all these absolutes are not to be expected.

Of logical foundations it is often thought that we must chose betweem a system which is not itself defined in terms of or reducible to some other, or else that circularity of definition must be admitted by defining the most simple case in itself or in some more complex system.
Since neither of these approaches is wholly satisfactory in itself, but both are likely to contribute to making the system precise and reliable, I advocate doing both. 

In case it may be thought that without achieving the relevant absolutes, the purpose of foundational thinking is abrogated, it may be helpful to mention that the approach of mathematics to logical foundations which took place primarily in the 19th Century was not motivated by a desire for absolutes, but rather by the need to make precise mathematical ideas which had become so lacking in clarity that rigorous proof of mathematical properties involving them was no longer possible.
The concept in question were those of mathematical analysis which were founded on a conception of number which included the infinitesimally small, an idea which had never been made clear.



\section{Abstract Logical Truth}

The conception of logic truth adopted here is essentially similar to that of Rodolf Carnap, for which \emph{analytic truth} was a psuedonym, and in terms of which logical necessity was also defined.
The presentation here is dissimilar to Carnap's, particularly in the way in which linguistic pluralism is addressed, and we may also say that Carnap's philosophy was not \emph{foundational} in relation to logical truth 
