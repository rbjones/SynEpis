
This philosophical kernal is devoted to the concept of logical truth.
First to its definition, and then to the idea that there could be a practically universal foundation for logical truth, which also provides an important element in defining precisely the semantics of any declarative language, by providing an abstract semantics which yields concrete semantics once a correspondence is specified betwee concrete entities and abstractions which represent them.

This chapter is concerned with the philosophical aspects of these matters, covering methodological considerations and yielding definitions of core concepts such as \rbjuse{logical truth}, and of the idea of a practically universal foundation for logical truth.

In subsequent chapters there will be more definite descriptions of a preferred practically universal system, and discussions of how empirical knowledge and all other kinds of knowledge in which meanings are sufficiently definite to support deductive reasoning can be built upon the proposed logical foundation.

\section{Methodological Prelimaries}

Perhaps the best known example of this genre is found in Descarte's meditations on first philosophy \cite{descartes2013meditations}, and so I should mention some of most important ways in which my methods differ from those of Descartes.

Descartes' method is built on systematic doubt, so radical that Descartes attempts to doubt everything which can be doubted, finding at first only one thing beyond doubt, which turns out to be his \emph{cogito} the inference from his doubting to his existence.
By inference from this slender base he constructs his entire philosophy, ultimately claiming that the results of his inferences from clear and distinct ideas are indubitable.

The methods adopted here are different.
There are several different aspects of the philosophy which are foundational, and they do not all work in the same way, but there are some common features.

Firstly, I do not pretend that in any aspect of this philosophy I doubt all that can be doubted and begin with an alsmost blank slate.
I recognise that this synthesis is the product of 4 billion years of evolution on earth of which the final blip is the contrubutions of many great thinkers who have contributed to the advancement of our culture.

Thr foundational ideas proposed are not arrived at by systematic doubt, but after consideration of many aspects of the evolution of life on Earth, the cutural evolution enabled by language, and the advances to the present day in science, mathematics and logic giving us the means to precision of language and certaintly of truth in core domains, finally subject to the demands of formal verification using digital information technology.
So, no blank slate.

Beyond the philosophical perspective specific technical proposals are made for the representation of knowledge.
The foundational role of the kernel is to provide a context in which the rationale for and the key propoerties of this technical proposal can be understood.

The terminology used in describing this kernel will fall into three 
groups.
Firstly there are many concepts which are used in ways within their existing accepted usage, in some cases with broader application.
Where a special or more precise meaning than is already understood   is intended, an informal definition will be given, of an existing term which will be used in a special way, or of some neologism.
Such terms will shown in a \rbjdeff{special font} at the point of definition, and each point of use will hyperlink (in blue) to that definition (in electronic versions, otherwise it may be found via the index of defined terms).


\ignore{
{\tt teletype} {\bf \tt bold teletype} {\bf bold not teletype} {\bf{\emph{bf emph}} {\textbf\emph{textbf emph}

    \textbf{\itshape This is in bold italic with fontspec.}
}}}

\section{Logical Foundations}

\subsection{Language}

A foundation system is a kind of generic language.
There are two ways of thinking about language, and of defining language.

In describing first order logic, it is usual to speak of a first order language as having a fixed vocabulary, adding further named constants creates a new first order language.

This is not what happens with natural languages, which constantly evolve both in terms of adding new words, and by changes to the meaning of existing words.
That does give problems in maintaining logical coherence, avoiding equivocation may become nigh impossible.
So if we want to give precise definitions of semantics and provide reliable (sound) rules of deduction, holding the vocabulary fixed is a good idea.
When we come to practical applications its not so attractive.

Programming languages are not associated with a fixed vocabulary.
You get a variety of constructs and defined terms which come with the language, but the modus operandum is that the process of writing a program consists in chosing new names and adding them to the vocabulary with their definitions.
Formal logical systems designed for practical use are similar.

For the purposes of this monograph minimalistic logical kernels devised to be safely extendable are the order of the day.

\subsection{What is Logical Truth}


In essence it is Carnap's conception of logical truth which is presented here \endnote{This topic has been controversial among philosophers in the twentieth century, particularly because of the identification by Rudolf Carnap of logical truth with \emph{analyticity}\cite{carnap47,carnap63a} and the very influential attack by Quine \cite{quineTDE} on that concept.
  My own view is that Carnap's choice of terminology was good, and Quine's attack baseless, but it is not my purpose here to argue that case.}%
(though the presentation is very different, as is the philosophical context in which it is placed).
It has a purpose which could not be filled by any of the alternative conceptions of logical truth.
Disagreements about terminology (what word to use for a concept)  have no impact on the substance of this proposal.

A \rbjdef{logical truth} is a certain kind of sentence in a declarative language, and its definition depends upon the meaning of the sentence.
It is therefore necessary to define the concept of declarative language and its semantics before we come to a definition of logical truth.





Foundational thinking has to address the problem of foundational regress, the need to underpin a supposed foundation.
Foundational thinking is reductionist, it is the thesis that for some kind of knowledge.
