A kernel, as I use the concept here, is a core which provides essential and fundamental elements upon which a larger system of some kind is built.

This chapter presents a kernel for an epistemological philosophical system.
A key feature of this kernel is the definition of a notion of \emph{logical truth} which is considered foundational for knowledge and epistemology.
In preparation for promoting a particular formalisation of that conception of logical truth, I describe an ordering on logical systems according to their expressiveness, relative to which a maximally expressive system will be sought.

In the next chapter I will identify a universal family of logical systems and a \emph{logical} kernel for that family.

The first, which is the subject of this chapter, is the Kernel or fundamental core of a philosophical system within the Western tradition which is generally considered to have begun with the philosophers of Classical Greece.

In that early beginning, starting around 600 BC with Thales, and progressing through the so called `pre-socratics' to the great intellectual achievements of Plato and Aristotle, philosophy was the love of knowledge, 

The philosophy has a kernel because it is an example of `first philosophy', an idea introduced by Aristotle in his \emph{Metaphysics} \cite{aristotleMetap}.

