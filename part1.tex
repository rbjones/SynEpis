% $Id: part1.tex $
\documentclass[10pt,titlepage]{book}
\usepackage{makeidx}
\newcommand{\ignore}[1]{}
\usepackage{graphicx}
\usepackage[unicode]{hyperref}
\pagestyle{plain}
\usepackage[paperwidth=5.25in,paperheight=8in,hmargin={0.75in,0.5in},vmargin={0.5in,0.5in},includehead,includefoot]{geometry}
\hypersetup{pdfauthor={Roger Bishop Jones}}
\hypersetup{pdftitle={Oracular AI v2}}
\hypersetup{colorlinks=true, urlcolor=red, citecolor=blue, filecolor=blue, linkcolor=blue}
%\usepackage{html}
\usepackage{paralist}
\usepackage{relsize}
\usepackage{verbatim}
\usepackage{enumerate}
\usepackage{longtable}
\usepackage{url}
\newcommand{\hreg}[2]{\href{#1}{#2}\footnote{\url{#1}}}
\makeindex

\title{\bf\LARGE An Epistemological Synthesis}
\author{Roger~Bishop~Jones}
\date{\small 2024-09-22}


\begin{document}
%\frontmatter

%\begin{abstract}
% Presentation of the most fundamental aspects of a positivist philosophical system articulated around a position on knowledge representation.
%
%\end{abstract}
                               
\begin{titlepage}
\maketitle

%\vfill

%\begin{centering}

%{\footnotesize
%copyright\ Roger~Bishop~Jones;
%}%footnotesize

%\end{centering}

\end{titlepage}

\ \

\ignore{
\begin{centering}
{}
\end{centering}
}%ignore

\setcounter{tocdepth}{2}
{\parskip-0pt\tableofcontents}

%\listoffigures

%\mainmatter

\chapter{Introduction}

This monograph presents the most fundamental elements of a positivist philosophical system in the course of articulating a position on \emph{knowledge representation}.

Throughout the evolution of life on earth, \emph{knowledge} (broadly  conceived) has played an essential role.
Knowledge has come in diverse forms which have co-evolved with life itself, beginning with knowledge of protein structures and the biochemistry of life encoded in the genomes of evolving species, continuing with many stages in the evolution of memory in the central nervous systems of animals, and leading to the evolution of culture, enabled by the emergence of oral language in large brained primates\footnote{genus homo}.

In the latest tiny fragment\footnote{about one millionth} of that history (after oral language acquired the persistence of written forms) we find the efforts of philosophers to \emph{understand} knowledge (ultimately to be called \emph{epistemology}), and the emergence of new kinds of knowledge, such as theoretical mathematics and its adoption of elaborate systematic deductive reasoning to confirm conjectures.

From this point onwards, knowledge and the methods used to present, establish and apply it, ceased to be a natural phenomenon guided only by evolutionary imperatives, becoming, in part, the product of human ingenuity, an engineered artifact.

The ideas about knowledge representation presented here are conceived in the context of and as a culmination of this latter phase in the evolution of knowledge; as an almost inevitable but nevertheless synthetic staging point.
An intelligible sketch of the proposal and its philosophical underpinnings depends upon an account of the progress achieved during those last 3,000 years.

\section{Language, Culture and Reason}

For this purpose it is first necessary to say something about declarative language, as it had been for perhaps as long as the previous 300,000 years, with the benefit of hindsight.

Fully fledged oral languages first appeared in \emph{homo sapiens} enabling oral culture and the beginnings of cultural evolution, which was able to progress much faster than the genetic evolution which had previously prevailed.
This began a process of accelerating evolution which continues to this day, fuelled in large part by technological advancements in the ways in which language could be stored and manipulated.






The term \emph{proposition} is used here for the principle content of indicative sentences in natural language, that which they assert, subject to certain qualifications.
The role of an indicative sentence is to carve a space of possibilities into two groups, one in which the sentence is true, and one in which it is false.
The affirmation of a sentence or the proposition it expresses is the same as claiming that it is true, hence ruling out all those possibilities which render it false.

\section{Greek Philosophy}
  
In ancient Greece, the birthplace of theoretical mathematics and the axiomatic method, a conspicuous divide appeared between the successes of deduction in the narrow domain of mathematics, and its limitations elsewhere.
Plato's philosophy reflected this divide in his identification of two worlds, that of ideal forms among which the objects of mathematics are found, and that of appearances, the shadowy world accessible only through our unreliable senses illustrated in his allegory of the cave.
Only in the former could certain knowledge be attained.
Of the latter, mere opinion at best.

Plato's world of ideal forms was not confined to mathematics, and we may read into his account an attempt to demarcate a sphere for reason extending well beyond mathematics, creating a domain for rationalist philosophy which has persisted ever since but has never in fact attained the status of mathematics in terms of clarity, certitude and reliability.

The difficulties in Plato's position were immediately apparent and the philosophy of Aristotle attempted a more moderate position, providing in \emph{demonstrative science} a legitimation of empirical knowledge in which results are obtained by (syllogistic) deduction from first principles based on observation.

\section{Modern Science}

\section{The Rigorisation of Analysis}

\section{Logical Foundations}

\section{Notes}

To do:
\begin{itemize}
\item make a connection between semantics and deduction
\item draw the contrast between reason in mathematics and elsewhere
  \item mention the grounds for radical scepticism (and the nature of its moderation in empiricism?)
  \item three stages:
    \begin{itemize}
      \item
        ancient Greek rationalism and its moderation
      \item modern science
      \item rigorisation of analysis
    \end{itemize}
    \item from science to engineering?
\end{itemize}


\chapter{Logical Truth}

\appendix
  
\chapter{Introduction (to 2024-10-17)}

This monograph presents the most fundamental elements of a positivist philosophical system in the course of articulating a position on \emph{knowledge representation}.

The advancement of humanity to its present state of prosperity has depended, substantially if not exclusively, on the accumulation of scientific and technological knowledge facilitating the mastery of our environment.
Essential to that process has been \emph{language}, at first oral, then written, continually advancing in scope and precision, recorded in media progressively supporting greater ease, speed and scale of communication and application.

The accumulation of a largely coherent and widely distributed body of applicable knowledge has been possible because of the approximation of declarative language to an ideal upon which the proposed knowledge base structure is founded.
It depends upon sufficiently expressive languages, which first appeared with \emph{homo sapiens} some quarter of a million years ago, making it possible for knowledge in to be shared, distributed, and preserved across generations in an evolving oral culture.
A key feature of propositional language is that it allows both particular facts and general rules to be expressed, and admits the application of such knowledge to new circumstances by deductive inference.
Though much language is context sensitive in its expression of propositions, broadly applicable knowledge is expressible in context free ways which allow for its communication through space and time, and facilitate the aggregation and sharing of knowledge.

It is probable that language co-evolved with oral culture rather slowly at first.
It was not until much quite recently, within the last 10 thousand years, that persistent written forms of language appeared, leaving a historical trace informing our understanding of the subsequent evolution of language, and admitting culture to extend beyond what could be preserved in brains and communicated orally.

During this early period, almost the entire history of oral language, the use of language would have routinely involved elementary informal deductive inference, an ability inseparable from competence in language.
Signs of more elaborate deductive reasoning come with the emergence of arithmetical notations.
But more extensive use of reason in attempting to understand the world around us,


Propositional knowledge is that kind of knowledge which is expressed in one or more indicative sentences; a proposition is that which an indicative sentence expresses.
A proposition is a claim about some subject matter which divides the possibilities for that matter into two groups, those in relation to which the proposition holds, or is true, and those for which it is not.
These are sometimes known as the `truth conditions' of the proposition, which for most purposes can be identified with the proposition, which in turn gives an account of the meaning of the indicative sentence which expresses the proposition.

The meaning of a sentence may be sensitive to the context in which it is asserted, but this is not the case for propositions, as the term is used here.
This sensitivity means that the proposition expressed by the sentence, and the truth conditional account of its meaning, may vary according to the context in which the sentence is asserted.
It is generally possible to disambiguate sentences by making these contextual elements explicit, and some sentences lack contextual dependency, for example, those which express general scientific laws.




-------

The position on knowledge representation has the following principle features:
\begin{itemize}
\item It addresses only \emph{propositional knowledge}

\item It underpins deductive reasoning in the context of the 

  
\end{itemize}
suitable to an age in which the impediments to formal working are diminished an ubiquity of artificial intelligence for its management and humanised presentation.

The problem of knowledge presentation is one which has been important to those seeking to engineer `artificial intelligence', and remains an important area of research with its own annual conference organised by \href{https://kr.org}{Kr Inc.}.
A glimpse of knowledge representation from the perspective of Artificial Intelligence may be seen in \emph{}\cite{fikes20}, where it is said that "\emph{Effective knowledge representation and reasoning methods are a foundational requirement for intelligent machines}".


\phantomsection
\addcontentsline{toc}{section}{Bibliography}
\bibliographystyle{rbjfmu}
\bibliography{rbj3}

\addcontentsline{toc}{section}{Index}\label{index}
{\twocolumn[]
{\small\printindex}}

%\vfill

%\tiny{
%Started 2023-08-19


%\href{http://www.rbjones.com/rbjpub/www/papers/p038.pdf}{http://www.rbjones.com/rbjpub/www/papers/p038.pdf}

%}%tiny

\end{document}

% LocalWords:
