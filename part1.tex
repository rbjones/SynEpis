% $Id: part1.tex $ #
\documentclass[10pt,titlepage]{book}
\usepackage{makeidx}
\usepackage{graphicx}
\usepackage{booktabs}
\usepackage{amsmath}
\usepackage{amssymb}
\usepackage{unicode-math}
\usepackage[unicode]{hyperref}
\usepackage{endnotes}

\usepackage{paralist}
\usepackage{relsize}
\usepackage{verbatim}
\usepackage{enumerate}
\usepackage{longtable}
\usepackage{url}

\usepackage{fontspec}

% Greek Characters

\setmainfont{TeX Gyre Pagella}[Ligatures=TeX]
\setmathfont{TeX Gyre Pagella Math}[Ligatures=TeX]
%\setmainfont{TeX Gyre Pagella}
\newfontfamily\greekfont{GFS Didot}

\usepackage[paperwidth=5.25in,paperheight=8in,hmargin={0.75in,0.5in},vmargin={0.5in,0.5in},includehead,includefoot]{geometry}
\hypersetup{pdfauthor={Roger Bishop Jones}}
\hypersetup{pdftitle={An Epistemological Synthesis}}
\hypersetup{colorlinks=true, urlcolor=red, citecolor=blue, filecolor=blue, linkcolor=blue}
%\usepackage{html}
\renewcommand{\thefootnote}{\roman{footnote}}

\newcommand{\ignore}[1]{}
\newcommand{\rbjdeff}[1]{\textbf{\itshape{#1}}}
\newcommand{\rbjdef}[1]{\rbjdeff{#1}\label{LABEL#1}\index{#1}}
\newcommand{\rbjuse}[1]{\hyperref[LABEL#1]{\itshape{#1}}}
\newcommand{\hreg}[2]{\href{#1}{#2}\footnote{\url{#1}}}

\makeatletter
\renewcommand\enoteheading{\section*{\notesname}%
    \@mkboth{\notesname}{\notesname}}

\newcommand{\resetnotes}{%
  \ifnum\value{endnote}>0%
  \theendnotes%
  \setcounter{endnote}{0}%
  \else%
  \fi%
}

\newcommand{\chapnote}[1]{%
  \chapter{#1}
  \input{{#1}.tex}
  \resetnotes
}

\makeatother

\pagestyle{plain}
\makeindex

\title{\bf\LARGE An Epistemological Synthesis}
\author{Roger~Bishop~Jones}
\date{\small 2024-09-22}


\begin{document}
%\frontmatter

%\begin{abstract}
% Presentation of the most fundamental aspects of a positivist philosophical system articulated around a position on knowledge representation.
%
%\end{abstract}
                               
\begin{titlepage}
\maketitle

%\vfill

%\begin{centering}

%{\footnotesize
%copyright\ Roger~Bishop~Jones;
%}%footnotesize

%\end{centering}

\end{titlepage}

\ \

\ignore{
\begin{centering}
{}
\end{centering}
}%ignore

\setcounter{tocdepth}{2}
{\parskip-0pt\tableofcontents}

%\mainmatter

\chapter{Introduction}
This monograph wraps an evolutionary tale around an epistemological synthesis.

The epistemological synthesis is built upon a logical core, first engineered for narrow mathematical purposes (though not without wider ambition), but is here re-purposed as a universal substrate for the representation of knowledge.
It is presented as \emph{foundational} in providing a primitive base for the largest of superstructures, and hence its advocacy is a variant of the much derided philosophical tendency to \emph{foundationalism}.
The philosophy constructed around it here may also be seen as a reversion to the ancient Aristotelian pursuit of \emph{first philosophy}, the idea that philosophy has a contribution to make which is in some sense prior to and more general than the sciences.
In this case, that the aggregation of a single shared body of coherent knowledge about the world we live in is a desirable end and demands an appropriate foundation for the representation of knowledge which is common to all the sciences (and beyond).

What has evolution to do with this?
There are two evolutionary stories which contribute to understanding the nature of the proposed logical foundations and the thesis I present about their importance for the future.

The first is the story which has lead to the emergence of the logical systems in question, which represent, I suggest, a kind of terminus beyond which further developments to such logical foundations are likely to be marginal (though we are just at the beginning of finding the best ways to build on these foundations).
This is a long story from which I present just those elements which are most signficant in leading to the supposed foundational terminus.

The second is a tale of how the future may unfold, and the evolutionary imperatives which will ensure that the identified foundational systems will become the dominant paradigm for rhe representation of knowledge.

This is not science, the principal ideas promoted here are not such as are susceptible to conclusive demonstration, and do not present clear opportunities for falsification.
Though this kind of criterion for solid scientific truth has an important place in science.
Life normally presents us with choices which have to be made on more tenuous grounds.
Engineers routinely work with models of the real world which are known to be adequate applicable approximations, such as the frameword of Newtonian physics now known to be false.
Philosophy probes beyond the uncertainties of everyday life into matters less easy to comprehend and susceptible to greater doubts.

The main substance of this monograph is offered as philosophical, as a contribution to epistemology.
That epistemology is conceived of as the the foundation of a broader philosophical system, which will be lightly sketched.
Though foundational in character, it is only possible to understand and articulate this system in the context of the evolutionary story which has lead to its core components.
The projection of that story into the future is an important motivator for the system, and a principle ground for the speculation that proposed system will eventually prevail.

At its most portentous, this is the story of the evolution of intelligence, and will become the story of its proliferation through the cosmos, in consequence of an evolutionary imperative which defies stasis.
These exotic and speculative ideas about the future of intelligence have an important role to play in expounding and motivating the epistemological synthesis, but are colourful accessories to the primary purpose of this monograph which is philosophical.

\section{Sketch of a Sketch}

It is my aim to provide in this introduction a most concise sketch of the core elements of this proposal, leaving for the remainder of the monograph to fill in and fill out that picture.

Here I preview what those core elements are.

\begin{itemize}
\item Remarks on the Foundationalism
\item
\end{itemize}

\section{First Philosophy and Foundations}

I am in this monograph seeking to practice, a foundationalist first philosophy.
From a lomg evolutionary perspective, I see us as part of an evolutionary trajectory which from the highest level may be seen as the evolution of intelligence and its proleferation across the galaxy and into the cosmos.
That whole process consists of and is fuelled by the aggregation of knowledge on a massive and accelerating scale.

Because this is a kind of evolutionary process we can to some extent anticipate how it will go, and since it is reasonable to 


First a key point about the foundational nature of this proposal and my  to `first philosophy'.
This foundationalism, if it be called that, does not rest on absolutes.
It does not regard any truth as absolutely certain, any statement completely precise, or any account of semantics absolutely unambiguous.
But I will sometimes talk as if that were the case.
The precision with which the proposed semantics allows abstract concepts to be defined and truths about those concepts to be established is extraordinary by comparison with any prior system.
Consequently, I will occasionally make (perhaps oxymoronic) qualified claims to absolutes, using words like `quasi'-universal or `practically'-complete.
If ever I should fail to include such qualifications to what appear to be absolute claims, then the reader should supply his own pinch of salt.

More directly pertinent to the idea that this is \emph{first philosophy} are the following divergencies from the methods of first philosophies most famous practitioners.


\begin{itemize}
\item It is Aristotle who coined the term `first philosophy'\footnote{If anyone can be credited with coining a term in a language yet to be devised}%
 in the volume which was later to be named `Metaphysics'.
  In this he conceived first philosophy as addressing those matters which were independent of particular sciences, and hence,in some sense prior to and more general than them.
  The most important of these issues was the nature of substance, the features of all that exists, rather than those particular to the subject matters of the individual sciences.
  The foundational stance proposed here regards ontology as \emph{conventional}, most especially abstract ontology, in which abstract entities have just those properties which we ascribe to them, provided onlt that the attributed properties are logically consistent.
  This doesn't mean that the choice is wholly arbitrary, and we will discuss a very small number of options for abstract ontology in our foundations among which a choice can be made on pragmatic grounds, for the choices do have practical implications.

  Despite divergence on exactly what problems should be addressed by first philosophy, the foundationalism here adopted remains very much in the spirit of Aristotle in holding that there are important matters to be addressed by philosophers which are in some sense prior to and providing support for the conduct of the sciences.
  
\item Descartes' `Method of Doubt' is the other best know example of \emph{first philosophy}.
  This method consists in doubting everthing that can be doubted, and then progressing from the slender remains by deductive reasoning.
  Descartes found he could doubt everything but his own existence, and somehow managed on that basis to deduce that there is a god who is constutionally capable of allowing a world in which anything perceived clearly enough could possibly be false.

  To understand the foundationalism presented here and the arguments with which I support it you must first appreciate the distinction between object and metatheory, closely related to that between an object language (one about which we are talking, perhaps defining) and a meta language (the distinct language in which our discussion takes place).
  It is the object system which is the proposed foundation system, and it is our definition which begins with nothing and procedes to the simplest system which can provide an adequate foundation.
  In doing this, we use a well established meta-language, together with a handful of neologisms necessary to present a small number of novel features or ideas, and some knowledge of the 4 billion years of evolution which has brought us to this point, with very particular attention to the evolution of the language of mathematics and the quite recent reduction of mathematical reasoning to logical foundations.
  So, this foundational first philosophy is not conducted in a vacuum or on a tabular rasa.

\item The foundationalism advanced here is broadly similar to the logicism which sought to reduce mathematics to logic, and builds upon the technical advances which came with that project, but continued to progress after the core idea fell out of favour.
  However the scoping is explicitly more broad, the proposed foundations offered as a (quasi-)universal basis for abstract semantics, thereby providing a basis for deductive reasoning about knowledge of all kinds rather than specifically underpinning mathematics.
  
\end{itemize}


\section{What is Knowledge?}

Epistemology is the theory of knowledge.
Later I will talk about some of the various kinds of epistemology in order to make clear the synthetic epistemology which I am attempting in this monograph, but for this introduction, doing a bit of synthetic epistmology by talking about knowledge will suffice.

In the broad conception of knowledge which concerns me here,\footnote{Which encompasses the whole of semiotics.} knowledge is a natural phenomenon in which one physical structure carries information about some other aspect of the real world.
Often the relationship is underpinned by a causal connetion between the representation and the structure represented which results in the state of the representation reflecting the state of the causing system and therefore conveying information to anyone observing the relationship who understands how that relationship works.

In some cases this causal relationship is very direct and obvious, for example in perception, when a causal chain from the perceived structure mediated by sense organs results in modifications to synaptic weights in the brain of the observer which represents information about the perceived structure.
If the perceiving person then writes an account of his experience, we have a very different representation of information about that structure, and a more complex and uncertain causal relationship between the two.
This kind of knowledge, in the brain, is known as `explicit', and its appearance in language as 'declarative'.
Declarative knowledge, knowing that, may be distinguished from procedural knowledge, knowing how.
Its representation, again in synaptic connections and weights, is a physical system which effects a useful behaviour in response to a suitable physical context, which result in some appropriate physical activity.
Even where our knowledge may be of how to accomplish some mental gymnastics, the result will at least have some physical representation in the brain, and most likely will be transcribed into some other physical medium.

Having emphasised the physical nature of both the represenatation of knowledge and its subject, I should aknowledge that, increasingly as human civilisation has advanced, important bodies of knowledge are most naturally seen as speaking about non-physical abstract matters.
This is a natural consequence of the pragmatics of generalisation, in which patterns applicable to many distinct physical situations are captured without reference to any specific instance of their applicable instances.
Naturally, when we, as epistemologists, talk about these phenomenon, the inevitable (desirable and intended) generalisations and abstraction, will tempt us into considering abstractions from the concrete representations and attribute the representation of knowledge to abstract forms.
Nothwithstanding that the representation and the subject will not allways be physical, it will be useful in this discussion to speak as they always are so that we can focus the following talk about abstractions on the relationship between representation and subject matter, which I will speak of as `semantics' however distant it might feel from linguistics.

Semantics is the relationship between knowledge and its subject matters, and the existence of some such relationship is the crucial condition for some physical structure to represent or constitute knowledge.

This concern with the representation of knowledge has not been central to most philosophy, which additional criteria about the fidelity of the representation and the grounds for belief in that fidelity (justification) have been central, and have lead to some philosophers regarding that those criteria can never be realised and that knowledge is therefore unattainable, and others to devote considerable time and energy to the refutation of scepticism.
This we may think of as a debate about a particularly fastidious conception of knowledge, while usage in the broad sweep of everyday contexts is much more accomodating.
In this context, the conception of knowledge as requiring not only faithful representation but conclusive evidence of that fidelity is a narrowing of the scope of knowledge which would be fatal to the kinds of evolutionary argument which I will in the course of this monograph advance for the future preponderance of informations systems built around the representations of knowledge which this epistemological synthesis advances.

It is now time to talk about semantics, the role of abstract ontolgy in precisely describing the semantic relationships between knowledge and its subject matter, and hence the role of abstract structures in the representation of knowledge.

\section{Semantics and Abstraction}

\section{The Structure of this Monograph}

The epicenter of this epistemological synthesis is the idea of a quasi-universal logical foundation system, and the exposition has three principle aims:
\begin{enumerate}
\item To define and explain the term `quasi-universal foundation system'
\item To present a preferred quasi-universal foundation system, together with support for the claim that it is one.
\item To explain how such a system provides a basis for a distributed shared knowledge base which is suitable for use by an inteligence network expanding across the galaxy and beyond.
\item To motivate the proposed foundations through an account of their origin, and a projection of their role into the future, assessing their role in the light of the evolutionary imperatives which will govern which kinds of intelligent systems will predominate as transgalaxic proliferation progresses.
  \end{enumerate}
  
  
  

  
\ignore{

\subsection{Epistemic Perspectives on Evolution}

\subsection{Intelligence and Episteme}

\subsection{The Evolution of Intelligence}
}


\resetnotes

\chapter{The Evolution of Declarative Knowledge}\label{EDK}

The evolution of semantics is a slender thread which is pivotal to the epistemological theses which underpin this synthesis.

There are many ways to give structure to the four billion years of evolution here on earth.
The largest scale structure of interest here concerns the evolution of intelligence, which has happened (I suggest) twice, in wholly different ways.

The first time it was biological evolution.
The Darwinian evolution of species \cite{darwin-oos}, which took life on earth from single celled prokaryotes all the way to the species of genus \emph{homo}, of which more than one may have been `intelligent', but only one now remains, \emph{homo sapiens}.

Intelligence comes in degrees and varieties.
In using the term in a black and white way here I am adopting, for present purposes, the criterion that intelligence `proper' is the ability to engage in collaborative design and construction sufficiently well to ultimately engineer intelligent artifacts.

The second time, the evolution which created intelligence was cultural, and not very Darwinian, involving intelligent design as a source of variation, and intelligent selection rather than natural selection (we may debate where the rather artificial boundary between the two might lie).
This second evolution of intelligence is not quite yet complete, for by my chosen criteria of intelligence, it will only be complete when we have artificial intelligence capable of engineering new generations of intelligence.

This division of evolution into two stages of similar significance bbut very different duration (Billions of years against hundred thousands), is significant from the point of view of semantics, and the evolution of semantics exposes developments which have been crucial to the re-invention of intelligence.

The second phase, in which cultural evolution takes the lead, is enabled by the evolution of oral language, and thenceforth the continuing accelerating evolution of way of representing, communicating, storing and exploiting knowledge which have been effective in part because of their approximation to an ideal which the evolution of semantics has only very recently permitted to be clearly articulated.



\resetnotes

\ignore{
\chapter{A Philosophical Kernel 2}

The proposed system for the representation of knowledge is \emph{foundational}, it is a system to which other ways of representing knowledge are reducible in some sense (to be described).
It is therefore necessary in its articulation to speak in general of knowledge representation systems and of the idea of reduction relative to which the proposed system may be seen as universal in a broad class of such systems.

Though the proposal is \emph{epistemological}, its description depends upon \emph{metaphysics} (particularly, \emph{ontology}), \emph{philosophy of language}, and \emph{logic}, those four aspects of philosophy combining to articulate the most fundamemtal parts of the system.

\emph{Foundationalism} has been regarded as a failed doctrine by many philosophers for a good while, but this may well be because they consider only a straw man, in which foundational theses always make absolute claims about the most fundamental parts of their systems.
This I will not do, I aknowledge that neither in relation to meaning nor verification can one ever have absolute precision or certainty.
One can however, in certain domains which I will present as of particular importance, come closer to those ideals than any practical purpose demands.
Foundations accomplish that end, but they do so from a particular philosophical perspective, so technical adequacy will not necessarily by universally convincing.

The foundational problem in relation to both meaning and truth is how to terminate infinite regress in definition or justification, and this may be presented as a choice between finding a foundation which is self-evidently clear and conclusive, or rendering the foundation precise through the use of language ultimately to be defined in terms of it.
In this proposal, this choice is rejected in favour of doing both.

The foundational enterprise can be appreciated as the threading of language through the eye of a needle.
The whole of complex languages are to be defined in terms of very simple primitives, which are definable using a tiny part of the languages which can then be constructed upon them.

\section{Epistemological Abstraction}

Epistemology has often been in significant measure influenced by the world around us, and the ways in which human beings are built and acquire and deploy knowledge of themselves and the world around them.
It may also be influenced by or intimately concerned with the language with which we talk about knowledge, not least the meaning of the word ``know''.

The approach here, which we may think of as an approach to `abstract epistemology', seeks to minimise the extent to which epistemology reflects such earthly or anthropomorphic influnces.
How could such an epistmology arise, what would be the purpose of attempting to construct such an abstract epistemology, and how could it possibly succeed?

This moment in the evolution of intelligence provides a context in which this might be understood.
We stand, as I write, at a point at which many of the hallmarks of intelligence in humans are now to be seen in computational artifacts.
We are also venturing into synthetic biology which may transform the evolution of biological intelligent systems, and the ambition to send intelligent systems across the solar system and into the star systems beyond is on the ascendent.
It is likely that the promulgation of intelligence across the galaxy will ultimately be predominantly of non-biological intelligence, and that even that central core where biological life has penetrated will be populated by species well advanced beyond homo sapiens, growing progressively more distant from earth and homo sapiens.
These are among the motivations to consider epistemology in ways which stand back both from human language about knowledge and human ways of acquiring knowledge.

Pure mathematics provides examples of structures, knowledge of which will surely be universal among intelligence wherever it is found.
The natural numbers, those numbers which we use for counting discrete entities, are a simple example.
Abstraction to is the business of pure mathematics, in contact with an alien civilisation, it may be the mathematicians who would best suceed in communicating with their alien counterparts.

\section{Foundational Metaphysics}

The abstract foundation for epistemology here envisaged is a story couched in terms of abstract entities.
So what are they, and what abstract entities are there?

The distinction between abstract and concrete is not completely clean, since we can construct abstract entities with concrete constituents.
We are here concerned with purely abstract entities, and adopt a conventionalist position in relation to such entities.
The question what abstract entities there are is therefore to be determined by context, an aspect of context which might (or might not) be fixed by the language.
If it were meaningful to speak of there being absolute truths about what abstract entities exists, it would not impact this position, for the utility of \emph{chosing} a domain of abstract entities for the purpose of constructing an abstract model or for developing a mathematical theory is not dependent on what abstract entities do or do not `really exist'.

Concrete ontology is not far removed from this conventional stance, though the evaluation criteria which is makes sense to apply to concrete ontologies are more stringent.
Concrete ontology, we might expect, is primarily of utility in constructing models of the physical world, and may be subject to similar criteria.
Though philosophers have debated the validity of inductive reasoning to establish the truth of empirical generalisations, and have proposed alternatives such as continuous search for falsification, it is clear that the utility of a model of the empirical world may persist in the face of good evidence that it is literally false.
Newtons theories of motion and gravitation are the clearest examples, where accepted as false they are nevertheless more widely used than the theories which displaced them and are still thought to be `true'.

\section{Abstract Languages}



\resetnotes

\chapter{A Philosophical Kernel 3}

The philosophical context for the epistemological synthesis presented in this monograph is a foundational philosophy.
The kernel of the philosophy provides the foundations upon which the whole is based.

The kernal is epistemological, but its description involves other aspects of philosophy which may also be thought of as foundational, including metaphysics (mainly ontology), and the philosophies of language and logic, all of which are involved in providing a description of the most fundamental concept, that of \emph{abstract logical truth}.

\section{Methodological Preliminaries}

A foundational approach to some problem domain consists in discovering some smaller and/or simpler domain (the foundation) to which all the problems in the more complex domain are reducible.

The classic conception of knowledge holds that for a true belief to count as knowledge, it must be supported by a conclusive justification, from which it would follow that a foundation to which knowledge is reducible should itself be beyond doubt.
This philosophy and the kernel on which it is based, is skeptical of any absolute claims, it does not assume that it is ever possible to assign meaning to language with absolute precision, or that any sentence can be known to be true with absolute certainty.
Fortunately we do not need such absolutes, life goes on without them.
Nevertheless, there are very great differences in the precision of language and the certainty with which truth can be ascertained.
There are variations from one domain to another, and within a domain the language may evolve to greater expressiveness and precision with the benefit of experience, and the methods for ascertaining truth can often be progressed to increase their reliability and reach.

This philosophical kernel is primarily concerned with \emph{logical foundationa}, which may now have been progressed close to a limit in semantic expressiveness, certainty of proof, and completeness, though in all these absolutes are not to be expected.

Of logical foundations it is often thought that we must chose betweem a system which is not itself defined in terms of or reducible to some other, or else that circularity of definition must be admitted by defining the most simple case in itself or in some more complex system.
Since neither of these approaches is wholly satisfactory in itself, but both are likely to contribute to making the system precise and reliable, I advocate doing both. 

In case it may be thought that without achieving the relevant absolutes, the purpose of foundational thinking is abrogated, it may be helpful to mention that the approach of mathematics to logical foundations which took place primarily in the 19th Century was not motivated by a desire for absolutes, but rather by the need to make precise mathematical ideas which had become so lacking in clarity that rigorous proof of mathematical properties involving them was no longer possible.
The concept in question were those of mathematical analysis which were founded on a conception of number which included the infinitesimally small, an idea which had never been made clear.



\section{Abstract Logical Truth}

The conception of logic truth adopted here is essentially similar to that of Rodolf Carnap, for which \emph{analytic truth} was a psuedonym, and in terms of which logical necessity was also defined.
The presentation here is dissimilar to Carnap's, particularly in the way in which linguistic pluralism is addressed, and we may also say that Carnap's philosophy was not \emph{foundational} in relation to logical truth 

\resetnotes
}%ignore

\chapter{A Philosophical Kernel}\label{PK}

A kernel, as I use the concept here, is a core which provides essential and fundamental elements upon which a larger system of some kind is built.

This chapter presents a kernel for an epistemological philosophical system.
A key feature of this kernel is the definition of a notion of \emph{logical truth} which is considered foundational for knowledge and epistemology.
In preparation for promoting a particular formalisation of that conception of logical truth, I describe an ordering on logical systems according to their expressiveness, relative to which a maximally expressive system will be sought.

In the next chapter I will identify a universal family of logical systems and a \emph{logical} kernel for that family.

The first, which is the subject of this chapter, is the Kernel or fundamental core of a philosophical system within the Western tradition which is generally considered to have begun with the philosophers of Classical Greece.

In that early beginning, starting around 600 BC with Thales, and progressing through the so called `pre-socratics' to the great intellectual achievements of Plato and Aristotle, philosophy was the love of knowledge, 

The philosophy has a kernel because it is an example of `first philosophy', an idea introduced by Aristotle in his \emph{Metaphysics} \cite{aristotleMetap}.


\resetnotes

\chapter{The Fundamental Triple Trichotomy}\label{FTT}

Consideration of the domain in which extended deductive reasoning can be made safe has lead to a focus on purely abstract logical truths.
In this chapter I will talk about other important domains of discourse and how logical truths can enable deductive reasoning in those domains.

\section{Hume's Forks}\label{HF}

The triple-trichotomy is and elaboration on a theme fundmnetal to the philosophy of David Hume, who gave a central place to the dstinction which became known as `Hume's Fork' 


David Hume was a philosopher of the Scottish Enlightenment.
The enlightenment was a period of ascendency in the place of reason in
the discussion of human affairs, when science had secured its
independence from the authority of church and state and had a new
confidence in its powers occasioned particularly by the successes of
Newtonian physics.

Hume looked upon the philosophical writings of his contemporaries and
found in them two principle kinds, an ``easy'' kind which appealed to
the sentiments of the reader, and a ``hard'' kind which trawled deeper
and appealed to reason.
This latter kind, ``commonly called'' metaphysics, was preferred by
Hume, but found nevertheless, by him, to be lacking, infested with religious 
fears and prejudices.
Hume's feelings about these aspects of philosophy were not vague
misgivings.
He had a specific epistemological criterion which he saw these
philosophical doctrines as violating.

Hume's project involves an enquiry into the nature of human reason for
the purpose of eliminating those parts of metaphysics which go beyond
the limits of knowledge, and establishing a new metaphysics on a
solid foundation limited to those matters which fall within the scope
of human understanding.

David Hume wrote his philosophical {\it magnum opus}, {\it A Treatise on
  Human Nature} \cite{hume39} as a young man.
He was disappointed to find his work largely ignored and otherwise
misunderstood, and thought perhaps that his presentation had been at
fault.
To improve matters he wrote a shorter work more tightly focussed
on the core messages which he thought of greatest importance.
This he called {\it An Enquiry into Human Understanding}
\cite{hume48}.

In a central place both logically and physically in this more concise
account of his philosophy he says:

\begin{quote}
``ALL the objects of human reason or enquiry may naturally be divided
  into two kinds, to wit, Relations of Ideas, and Matters of Fact.'' 
\end{quote}

We shall see that Hume is here identifying a single dichotomy which
corresponds to all three of the distinctions which here concern us.
In his next two paragraphs he expands in turn on the kinds he has thus
introduced.

\subsection{Relations of Ideas}

\begin{quote}
``Of the first kind are the sciences of Geometry, Algebra, and
Arithmetic; and in short, every affirmation which is either
intuitively or demonstratively certain.
That the square of the hypotenuse is equal to the square of the two
sides, is a proposition which expresses a relation between these
figures.
That three times five is equal to the half of thirty, expresses a
relation between these numbers.
Propositions of this kind are discoverable by the mere operation of
thought, without dependence on what is anywhere existent in the
universe.
Though there never were a circle or triangle in nature, the truths
demonstrated by Euclid would for ever retain their certainty and
evidence.''
\end{quote}

Hume is distinctive here among empiricist philosophers in having a
broad conception of the {\it a priori} (though he does not use that
term here), allowing that the whole of mathematics is {\it a priori} (a concept encapsulated in Hume by the phrase ``discoverable by the mere operation of thought ...'').
In this he may be contrasted, for example, with Locke who allowed
only certain rather trivial logical truths to be knowable {\it a
  priori}.
Nevertheless, Hume's conception of the {\it a priori} remains narrow
by comparison with the rationalists, and in particular, as Hume will
later emphasize, excludes metaphysics.

\subsection{Matters of Fact}

\begin{quote}
``Matters of fact, which are the second objects of human reason, are not ascertained in the same manner; nor is our evidence of their truth, however great, of a like nature with the foregoing. The contrary of every matter of fact is still possible; because it can never imply a contradiction, and is conceived by the mind with the same facility and distinctness, as if ever so conformable to reality. That the sun will not rise to-morrow is no less intelligible a proposition, and implies no more contradiction than the affirmation, that it will rise. We should in vain, therefore, attempt to demonstrate its falsehood. Were it demonstratively false, it would imply a contradiction, and could never be distinctly conceived by the mind.''
\end{quote}

The evolution of the following three dichotomies is the theme of this chapter, though we will find other related dichotomies which feature in the history.

The terms which I will use to speak of them, in this chapter are:
\begin{itemize}
\item{analytic/\-synthetic}
\item{necessary/contingent}
\item{a priori/a posteriori}
\end{itemize}

As I shall use these terms these are divisions of different kinds of
entity, by different means.
For that reason they cannot be said to be identical, and their extents clearly depend upon exactly how the relevant technical concepts are defined, but with suitable and reasonable definitions, these dichotomies prove to be very closely coupled.

The first is a division of sentences, understood in sufficient context
to have a definite meaning, and is a division dependent upon that
meaning.
Meaning is not an univocal term, it is particularly uncertain in meaning,
But in this context the requirement is very specific, only one possible component of meaning is required, which is the truth conditions of the sentence.
The truth conditions of a sentence are an assignment of truth values for he sentence in every possible condition (what is a `condition' depends on the language, for natural languages a condition would include both a state of the world and sufficient context to disambiguate the sentence if its meaning is in any way context sensitive). 

The second is a division of {\it propositions}, which may be
understood for present purposes as {\it meanings} of sentences in
context.
What a proposition is need not be settled except that it must include, again, truth conditions, the context of any sentence having been settled to determine the proposition which it expresses.
Such a proposition is considered necessary if it is true in every relevant circumstance, the range of circumstances having been fixed by the language and constituting the subject matter of the language.
The concepts of analyticity and necessity are, by this kind of definition, logically related.
A sentence is analytic if once disambiguated by appropriate context, it is seen to exppress a necessary proposition.
These exact definitions are not to be found in Hume, and are not philosophically uncontroversial, but are presented here so that we can speak of that position adopted by some later philosophers which Hume may be thought to have anticipated.

The division is made according to whether the proposition
expressed must under all circumstances have the same truth value, or
whether its truth value varies according to circumstance.
In this we are concerned with two particular notions of necessity,
those of logical and of metaphysical necessity, the latter being
sometimes taken to be broader than the former.
A part of the role of Hume's fork in positivist philosophy is to
banish metaphysical necessity insofar as this goes beyond logical necessity.

The third is for our purposes also a division of propositions, on a
different basis.
It concerns the status of claims or of supposed knowledge of
propositions.
It is expected that such a claim must in some way be
{\it justified} if we are to accept it, and that the kind of
justification required depends upon the proposition to be justified.
The justification is \emph{a priori} if it makes no reference to observations
about the state of the world, i.e. to sensory observations or other results obtained on the basis of such evidence.
The distinction in this case may be described as an epistemic distinction, since it concerns what kind of justification we may expect for the kind of proposition in question.

The suggested identity between the first two concepts has a pale reflection in relation to this epistemic distinction.
In this case we do not assert an identity but rather recomend that an \emph{a priori} justification be required for necessary propositions, and that an \emph{e posteriori} justification be required for contingent propositions, thus closely if indirectly coupling the three dichotomies.

\subsection{The Place of The Fork in Hume's Philosophy}

The mere statement of the fork (which we shall see, is not original in
Hume) is of lesser significance than the role which it plays in Hume's
philosophy, which serves to clarify the distinction at stake and draw
out its significance.

Hume's philosophy, like Descartes' comes in two parts of which the
first is sceptical in character, and the second constructive.
In both cases the sceptical part clears the ground for a new approach
to philosophy which is then adopted in the constructive phase.

For our present purposes we are concerned principally With the first
sceptical phase of Hume's philosophy, because of the delineation of the scope
of deductive reason, and hence of the analytic/\-synthetic dichotomy
which is found in Hume's sceptical arguments.
This delineation is baldly stated in Hume's first description of the
distinction between ``relations between ideas'' and ``matters of
fact'', for there Hume tells us that no matter of fact is
demonstrable.

This bald statement would by itself have little persuasive force if it
were not followed up with more detail, even though ultimately this
detail does not so much underpin the distinction as depend upon it.

Hume's further discussion begins with the consideration of what
matters of fact can be known `beyond the present testimony of 
our senses or the records of our memory'.
The inference beyond this immediate data is invariable causal, we
infer from the sensory impressions or memories to the supposed causes
of those impressions.
But these are not logical inferences, causal necessity is for Hume no
necessity at all (even less the inference from effect to cause).
Hume's central thesis that matters of fact are not demonstrable is
in this way reduced first to the logical independence of cause and
effect, and then to the distinction between deductive (and hence sound)
inference and inductive inference (whereby we infer causal
regularities and their consequences).

Given that Hume considers all inferences from senses to be based on
induction, and sees no validity in causal inference, it follows that
from information provided directly to us by the senses nothing further
can be deduced which is not simply a restatement, selection or summary
of the information itself.
Further enlightenment from this sceptical doctrine is primarily the
application of this principle to various kinds of knowledge.
In the process Hume does a certain amount of 



\section{The Triple Trichotomy}

The `triple trichotomy' is a presentation of these extended domains of reasoning as a two dimensional matrix, one dimension associated with the distinct domains, the other with different kinds of characteristics which they posess.

\begin{table}[h]
    \centering
    \begin{tabular}{l | l | l | l}
         & \rotatebox{45}{Semantics} & \rotatebox{45}{Evaluation} & \rotatebox{45}{Modality} \\
        \midrule
        Spiritual & non-natural  & propriety & normative \\
        \midrule
        Empirical & concrete & utility & contingent \\
        \midrule
       Logical & abstract & proof & necessary \\
    \end{tabular}
    \caption{The Fundamental Triple Trichotomy}
    \label{tab:example}
\end{table}

In the above table, abstract semantics for logical truths (which are to be established by deductive proof and are necessary rather than contingent, or normative). is shown in the lower left corner as befits concepts which are regarded as foundational to the whole.

I will talk through this table in the sections which follow, working from bottom up and left acrosss.

\section{Semantics}

David Hume's philosophy provides us with a first view of the three domains, through two important distinctions.
The first is what later became known as the \emph{analytic/synthetic} distinction, that described by Hume as `relations between ideas' and `matters of fact'.
Because Hume talks of it in terms of subject matter, this is readily understood as a semantic distinction.
Hume also considers both those two categories as concerned with what \emph{is}, rather than what \emph{ought to be}, and says that these normative claims are not derivable from mere descriptions, one cannot derive an `ought' from an `is', giving us three distinct categories each logically beyond its predecessors.

So how can the abstract semantics which yields logical truth be foundational for these domains which seem logically beyond it?
It is foundational because we can mimic the structure of these extended domains in pure abstractions, and then describe map the abstract ontology onto the concrete entities and the normative concepts.

\section{Evaluation}

In the classical conception of knowledge as justified true belief, we see that to establish something as knowledge we must be able to show conclusively that it is true.



\resetnotes

\appendix

\chapter{The Philosophy of Rudolf Carnap}\label{PRC}

The philosophy of Rudolf Carnap is the closest predecessor I know to the position described in this momograph, but is often misrepresented or misunderstood.
For those who are acquainted with his philosophy, my own position might be easier to understand if juxtaposed and contrasted with my understanding of Carnap (whether or not that understanding is sound), and for others a sketch of his philosophy with some minimal context might nevertheless be helpful.

I am not a scholarly student of Carnap's philosophy, and only came to appreciate its relevance to my own thinking fairly late, partly because what I now see as the main thrust of his philosophy, as clearly outlined in his own intellectual autobiography \cite{carnap1963}, seems not to have been much spoken of by his critics, who have too often focussed on features of his philosophy which were either misrepresented (such as, that he was a phenomenalist) or transient (the verification principle \cite{carnap1937}, his syntactic phase).

The following account leans towards his work on logical analysis and neglects his work on confirmation theory and testing, because my own thinking and this monograph contribute little to those areas.
It pivots around his work on logical syntax \cite{carnap1935}, partly because at this time his lectures in London on that topic \cite{carnap1937} give a concise survey of the main aspects of his philosophy at that time, and both his earlier ideas, as described for example in his intellectual autobiography \cite{carnap1963} and his mature philosophy are both covered well in the Schilpp volume of the Library if Living Philosophers \cite{carnap63a}.
The other important works \cite{carnap1956,carnap1950,carnap1990}  relate to his semantic phase and the criticism and repudiation of his philosophy by W.V.~Quine.

\resetnotes

\chapter{Instructions for Grok3}

I have written this monograph with copious assistance from Grok 3, an AI chatbot from xAI.

Notwithstanding the invaluable contribution Grok has made, the writing is all mine.

The following section contains the general instructions which I have used to brief Grok.
Naturally, these instructions have evolved over the course of the writing and this their state at the end.
Different stages in the process demand different instructions, and I have tried to keep instructions of a general character in a separate file, with supplementary files giving more specific asks for particular stages.

\section{The Most General}

My principle intellectual ambitions are now primarily philosophical, with a generous and flexible sense of the scope and methods of philosophy, but, nevertheless, a focus on theoretical foundations.

I aim to practice philosophy by producing written accounts of the ideas I am working on, which are oriented toward philosophical support for the effective application of modern advances in logic.
I now think of these as monographs, but expect that this may change as I learn to work effectively with Grok.
These notes are therefore written for Grok, and I will couch them in first and second person language.

I have decided that the best way to deal with the readership targeting is to write for an audience of two, you, Grok and me, Roger Jones.
I want to produce in the first instance a monograph which seems to me to adequately capture the ideas I had hoped to express, and which is clear enough for you to get a fairly deep understanding of them.
I don't yet have any well thought out ideas about how to check your understanding, so the first idea is conversation.
Maybe I will use the common textbook practice of including questions at the end of each chapter for you to answer and I can then review your answers with you.

Once we have a manuscript which I think is a good statement and which you understand well, I will ask you to prepare (with my help) materials for a variety of other target audiences and other channels of communication, which I hope will include making the relevant ideas available via a grok3 conversation (which will be possible if in no other way, but attaching the manuscipt to a grok session, though I will be hoping for a way of gathering feedback from these sessions which will require something more sophisticated, probably supported by future developments to Grok in X about which I can only speculate),

As to channels, the first product will be a PDF suitable for distribution electronically or printing on demand.
This could be converted into an ebook in multiple formats.
I am inclined to think of X as a main channel for promulgation, and imagine that you would be able to support a bot account on X (in due course) which would explain the ideas.
I also anticipate serialising the monograph on substack, and would want you to convert the chapters one at a time into posts for substack.
You may have some ideas on what other channels might be good ways of promulgating the work.

Given the breadth of the considerations, particularly the divide between evolutionary thinking, philosophy, logic and cognitive science/engineering, I would like it to be possible for those who interest and/or competence in only some of these areas to find something of interest even though perhaps skimming of omitting parts of the document outside their competence or interest.
So I shall be constantly trying to sketch informally in introductory parts and concetrate detail and depth into separate chapters or sections.

You are to act as my research assistant and reviewer of the developing work, but you will not be contributing directly to the writing of the first manuscript, i.e. the words will be mine, though I hope and expect that your indirect contribution will be substantial.
When it comes to derived works, the ideas will still be mine (though refined in discussion with you), but the words will be yours.

In all responses concision is very important.
Prolixity will impede my progress rather than advance it, you will need to cultivate a conservative sense of what is relevant and important.
Note also, that I am not looking for creative suggestions (unless I should explicitly ask for some).
Beyond fact checking I am looking to kmow whether what I am writing is intelligible.

Your general knowledge of all the matters touched on in the exposition is superior to mine, except in those parts where the monograph presents original material, and my first hope is that you will fact check everything which I write and let me know if there are any factual errors.

My second hope is that you will mention to me other work which I ought to be aware of because of its proximity to the subject matters I am addressing.
In this I am not asking you to compile a list of those things which seem most relevant even if tenuously 

The third desideratum is that you take cognisance of the intended or likely readership and advise me on matters addressed which may be dificult for that audience to comprehend, and perhaps make suggestions as to how improvements can be made.
Similarly with any matters with which they may seem likely to disagree, particularly if their disagreement is soundly based.

When I attach a document, or re-attach one, I would like you to automatically read the document.
This will usually be the current draft manuscript, which includes as an appendix these notes.
I will also be including in the body of the monograph notes specifically for you to understand what I am doing and where I am going, for which I will wrap in a tex command which can hide them in the final manuscript.
These sections of the document will be in quote blocks marked ``Note for Grok:''.




\listoftables
%\listoffigures

%\phantomsection
\addcontentsline{toc}{section}{Bibliography}
\bibliographystyle{rbjfmu}
\bibliography{rbj3}

%\phantomsection
\renewcommand{\indexname}{Index of Defined Terms}
\addcontentsline{toc}{section}{Index of Defined Terms}
{\twocolumn[]
{\small\printindex}}

%\vfill

%\tiny{
%Started 2024-10-19
%}%tiny


\end{document}

% LocalWords:
