% $Id: part1.tex $ #
\documentclass[10pt,titlepage]{book}
\usepackage{makeidx}
\usepackage{graphicx}
\usepackage{booktabs}
\usepackage{amsmath}
\usepackage{amssymb}
\usepackage{unicode-math}
\usepackage[unicode]{hyperref}
\usepackage{endnotes}

\usepackage{paralist}
\usepackage{relsize}
\usepackage{verbatim}
\usepackage{enumerate}
\usepackage{longtable}
\usepackage{url}
\usepackage{markdown}

\usepackage{fontspec}

% Greek Characters

\setmainfont{TeX Gyre Pagella}[Ligatures=TeX]
\setmathfont{TeX Gyre Pagella Math}[Ligatures=TeX]
%\setmainfont{TeX Gyre Pagella}
\newfontfamily\greekfont{GFS Didot}

\usepackage[paperwidth=5.25in,paperheight=8in,hmargin={0.75in,0.5in},vmargin={0.5in,0.5in},includehead,includefoot]{geometry}
\hypersetup{pdfauthor={Roger Bishop Jones}}
\hypersetup{pdftitle={An Epistemological Synthesis}}
\hypersetup{colorlinks=true, urlcolor=red, citecolor=blue, filecolor=blue, linkcolor=blue}
%\usepackage{html}
\renewcommand{\thefootnote}{\roman{footnote}}

\newcommand{\ignore}[1]{}
\newcommand{\rbjdeff}[1]{\textbf{\itshape{#1}}}
\newcommand{\rbjdef}[1]{\rbjdeff{#1}\label{LABEL#1}\index{#1}}
\newcommand{\rbjuse}[1]{\hyperref[LABEL#1]{\itshape{#1}}}
\newcommand{\hreg}[2]{\href{#1}{#2}\footnote{\url{#1}}}
\newcommand{\groknote}[1]{\vfil
  {\bf Note for AI Contributors:}
    \begin{quote}
      {\it #1}
    \end{quote}
    \vfil
  }

\makeatletter
\renewcommand\enoteheading{\section*{\notesname}%
    \@mkboth{\notesname}{\notesname}}

\newcommand{\resetnotes}{%
  \ifnum\value{endnote}>0%
  \theendnotes%
  \setcounter{endnote}{0}%
  \else%
  \fi%
}

\newcommand{\chapnote}[1]{%
  \chapter{#1}
  \input{{#1}.tex}
  \resetnotes
}

\makeatother

\pagestyle{plain}
\makeindex

\title{\bf\LARGE An Epistemological Synthesis}
\author{Roger~Bishop~Jones}
\date{\small 2024-09-22}


\begin{document}
%\frontmatter

%\begin{abstract}
% Presentation of the most fundamental aspects of a positivist philosophical system articulated around a position on knowledge representation.
%
%\end{abstract}
                               
\begin{titlepage}
\maketitle

%\vfill

%\begin{centering}

%{\footnotesize
%copyright\ Roger~Bishop~Jones;
%}%footnotesize

%\end{centering}

\end{titlepage}

\ \

\ignore{
\begin{centering}
{}
\end{centering}
}%ignore

\setcounter{tocdepth}{2}
{\parskip-0pt\tableofcontents}

%\mainmatter

\chapter{Introduction}

\groknote{Please read Appendix \ref{InstructionsForAIContributors} first for instructions on how to read this document and your role in its continuing composition.}
This epistemological synthesis is the first part in a progression which might ultimately advance from epistemology though architectural design, API and protocol specifications to code.
The boundary between epistemology and the earliest parts of system architecture is fuzzy, and the same subject matters can be addressed both from a philosophical and an engineering perspective.
This monograph focuses on the former, but is motivated by the latter.

The central purpose is to suggest that a single underlying abstract logical system suffices for the representation of all declarative knowledge, and that the forthcoming spread of intelligence through the solar system and across the galaxy, can best be accomplished with the aid of a distributed knowledge base using that single abstract representation (but diverse concrete representations).

The proposed logical system, closely based on the logic devised for and used in the Cambridge HOL Interactive Theorem Prover\cite{gordon1993}, is argued to be, in important theoretical and practical senses, universal for the representation of declarative knowledge.
It is offered as a foundation for abstract semantics in general, not merely a logical foundation for mathematics (for which its predecessors were intended).

Why do we need a 'foundation' for 'abstract semantics', and what are they?
These are just some of the philosophical questions which I hope to clarify in this monograph.
We cannot be certain of the truth of a declarative sentence if its meaning is unclear, and that same difficulty will arise for any language in which we might seek its clarification.
This is the problem of semantic regress, and the role of a semantic foundation is to terminate that regress.

The clarification of and justification for these claims forms an important part of the purpose and substance of this monograph, alongside some arguments intended to counter the scepticism about semantics to which philosophers are prone, particularly in relation to problem of semantic regress and that of regress in justification.

In these first introductory words on these ideas I will give two different perspectives on this same proposal.
But first some observations about desiderata.

The development of science and technology, upon which the prosperity of humanity has been built, is based on scientific methods in which declarative knowledge and deductive reasoning are central.
The systematic and extended use of deductive reasoning dates from the beginnings of mathematics as a theoretical discipline in ancient Greece, which resulted in the articulation of the axiomatic method and the elements of mathematics by Euclid which has remained influential to the present day.
Axiomatic mathematics was influential among Greek philosophers because they sought to understand the cosmos through reason, and by contrast with the advances in mathematics, reason proved impotent to secure consensus on truth in those broader domains.

Some Greek Philosophers nevertheless felt that Science should be a deductive discipline, and this is most explicitly presented in Aristotle's \emph{Organon}, his works on logic, in which the idea of \emph{demonstrative science} is presented and supported.
That early conception of science fell well short of modern scientific methods, and Aristotelian science was overtaken by one in which the deductive application of scientific principles was considered 

\paragraph{Declarative Language}

Declarative language, that expressed by indicative sentences in ordinary language, or in formal notations designed for similar purposes.
These are sentences which have truth values in appropriate contexts.
The context in which such a sentence is expressed determines a certain range of possibilities, and the sentence will usually vary in truth value according as to which of those possibilities obtains.
The truth of such an indicative sentence narrows down the range of possibilities and thereby conveys information about how the world (or other subject matter) is.

\section{Knowledge}

The classical definition of knowledge first appearing in the works of Plato, is that knowledge is \emph{justifiable true belief}.
However, this monograph addresses a broader conception of knowledge which is likely to be more convenient for the management of large bodies of knowledge.

This is not a claim about what knowledge \emph{is}, it is merely a description of the particular usage adopted in this monograph.
Relative to the traditional perspective, it differs by being neither psychological nor homocentric.
Knowledge is not exclusively mental, but predominantly physical.

Knowledge is a physical phenomenon in which some part of the physical universe contains information about some other thing, often another part of that universe.
Usually that it does so derives from its origin in a causal relationship between the two, though this may be very indirect and tenuous.

Thus, we acquire through our senses knowledge of our environment, because there is a causal chain which begins with various features of the environment, acts through the media of our senses (light, sound, physical pressure) and is conveyed to, processed and stored in our brains by the neural networks which form our central nervous systems.

More obliquely, on the basis of such observations of the universe around us, scientists formulate and test hypotheses about that universe, which if shown to be accurate become part of the body of scientific knowledge recorded in scientific journals and held in a variety of storage media in automated information systems.

The purpose of encoding information in this way is to enable people to live their lives better as a result of planning how to achieve their objectives in a way which takes into account the likely outcome of such plans.
Knowledge about how the world \emph{is} is crucial to this, as is knowledge of how the world would come to be if our plans were put into effect.

For knowledge to be useful in this way the relationship between the representation and the represented must be understood both by the people creating the knowledge or be consistent with the causal chain which effected it, and by those seeking to take advantage of the knowledge.
Indeed useful or not, one cannot reasonably be said to possess knowledge if you are merely in possession of the representation but not its meaning, or what we call more specifically, its semantics (to distinguish it from aspects of meaning which might involve matters more related to its significance than its content).

Semantics then, is an essential aspect of the representation of knowledge, and it is something which attaches not to the specific content of a single representative of knowledge, but to a general method of encoding a certain kind of knowledge of which it is an instance.
Such general schemes include what we call languages, but may include coding schemes which we might not think of in that way, or might only be figuratively describing as knowledge.
The coding of the structure of proteins in the DNA of the chromosomes of living organisms is an example of such an encoding.
We might also consider a photograph to be knowledge of its subject, but might struggle to identify the language in which it is expressed.

In this, the representation of knowledge may be thought to encompass not only language, but also the more diverse phenomena of symbiosis.

\section{Declarative Language and its Universal Foundations}

Declarative knowledge, as that term is to be used here, is more narrowly scoped, and is more particularly relevant to the process of exploiting general knowledge for the purpose of planning and engineering.
This process of applying general knowledge... 

\cite{dretske1981}


%\section{Set Theoretic Foundations}

%\section{Type Theoretic Foundations}



\chapter{Logical Universalism and Pluralism}

The central thesis of this work concerns the merits and applicability of a foundational approach to the representation of knowledge.
Because of its intended role in the management of knowledge, it is natural to consider the philosophical aspects of this enterprise as belonging to epistemology, and the deliberate engineering of this foundational system, albeit rooted in fundamental as well as pragmatic considerations suggests that the resulting philosophical underpinnings might be thought of as a kind of synthetic philosophy.

Despite this epistemological mission, the sources which have contributed to the establishment of this foundation system belong to disciplines distinct from, if adjacent to, epistemology.
In the first instance the impetus comes from mathematics, pressed forward by the needs of science and engineering.
The fruits of new mathematics in exploiting Newtonian science hastened its development at the cost of rigour, leading eventually to a sense crisis among some mathematicians about the foundations of these new mathematical methods.
In due course those mathematicians initiated of a series of foundational innovations which, at first, improved the mathematics, but eventually descended into philosophical territory, and were addressed by those with a competence in both the mathematics and in philosophy.
Ultimately these developments would lead to a new mathematical enterprise which became variously known as \emph{mathematical logic} and \emph{meta-mathematics}, involving the novelty for mathematics of considering symbolic languages as a subject matter for mathematics, rather than merely a means of writing it doen.

The century from 1847, beginning with works by Boole \cite{boole1847mathematical} and DeMorgan \cite{demorgan1847formal} and closing after an important contribution to recursion theory by Emil Post \cite{post1944recursively}, and the foundation system which we present here has arisen primarily through the further developnent of developments in mathematical logic by Engineers who sought applications in Computer Science.

Through that century of rapid development, primarily involving mathematicians and philosophers, a great variety of ideas have been explored, critiqued, compared and classified, often after the fact by scholars studying the history.
The concepts employed are not always clearly defined, and the distinctions which may have seemed clear when first drawn, may turn out to be illusory on closer inspection.

In relation to the foundation systems presented in this monograph, the most important adjectives with which I would like to describe them are \emph{universal} and \emph{foundational}, but it is clear to me that those terms are often read in ways which are not relevant in this case, so great care is needed in terminology.
This chapter is primarily concerned with the suggestion that the proposed foundation systems are universal, but before making that case virtually all the technical terms mentioned will have to be given a clearer meaning, which will generally not be quite what might have been expected.

\section{Some Preliminary Notes on Key Concepts}

It is natural to describe the proposed foundation systems as \emph{universal}, because in these systems it is suggested, any declarative language can be interpreted.
But in the history of mathematics, `universalism' is written of as in opposition to `pluralism' even though some of the earliest proponents of logical systems held to be universal were also enthusiastic pluralists.
At the same tine, the supposedly opposed pluralistic lobby, were also closely associated with one of the two universal foundation systems which are discussed here.
Another cleavage, that supposed between \emph{logicists} who believed in an intinate relationship between truths in mathematics and logic are often contrasted with formalists, the great majority of whose results were wholly acceptable to logicists.
In relation to both of these dichotomies Rudolf Carnap provides an important example of a distinguished philosopher who did not so much sit on the fence, as comprehensively on both sides.

Some of these conundrums can be resolved by refining the vocabulary and finding suitable ways of describing the nuances and confusions involved.
One particular point of terminology of special relevance to this proposal is the question of what constitutes a foundation system, for it is only through some innovation in our ideas of what a foundation system is that an argument for universality can be sustained.
Before coming to this, it will be useful for the reader to know some history as I perceive it.
I do not pretend to historical scholarship, but even if I did, an account as brief as the one which follows could not help but be a gross oversimplification.
The case which I will make for the proposed foundation system will be independent of the details of this sketch, but the perspective from which I worked in preparing it may be helpful to the reader in understanding its content.

\section{Some History}

\subsection{Aristotle's Premonitions}

The use of extended and systematic deductive methods dates back as far as the ancient Greeks, from about 600 BC, who are said to have founded mathematics as a theoretical science.
Over a period of about 300 years progress the methods and results of deductive mathematics continued, with a number of attempts at consolidation of which the most enduring proved to be Euclid's Elements \cite{euclidEl1}.
At this stage, deduction was entirely informal, and there is little discussion of what does or does not constitute a sound deductive inference, the emphasis in the axiomatic method being the circumscription of the principles from which deductions begin, whose simplicity relative to the broader context of language in general may be responsible for the success of deduction in these relatively narrow domains.

Philosophers sought to replicate the successes of axiomatic mathematics in broader domains, and mathematics was considered an important study for philosophers as recognised by Plato's Academy.
Aristotle was the first of these Greek philosophers to have studied are written about logic, producing six books which were subsequently to be gathered together as his \emph{Organon}\cite{aristotleL325,aristotleL391,aristotleL400}.
Two notable aims of this work were, firstly to articulate Aristotle's conception of \emph{Demonstrative Science}, a systematic account of how to apply deductive reasoning beyond mathematics and throughout the sciences, and secondly to codify, in his syllogistic logic, principles of deductive reasoning, thus going beyond the practice of axiomatic method as understood in his time.

Though there may be no explicit claim here to a \emph{universal} logical system, it may be thought that logical universalism is implicit in Aristotle's conception of Demonstrative Science.
However, impressive though it is, Aristotle's syllogistic would prove insufficient for the needs of science, or indeed mathematics.
Mathematics continued to progess, but secured no benefit from those early logical systems, continuing to depend on informal deduction, and moving forward at the cost of some retrenchment from the peak of rigour found in Euclidean Geometry.

\subsection{Modern Science}

A new conception of science emerged after the renaissance, and represents a clear break with Aristotle's conception of Demonstrative science, particularly as expressed in Bacon's \emph{Novum Organun}\cite{bacon2017novum}, which is quite derisory about Aristotle.
The main features of this new scientific method are the emphasis of observation and experimental method, providing a basis for establishing the principles of nature, and the downplaying the role of deduction and of Aristotle's syllogism.

Nevertheless, the modern conception of scientific method has been variously called the hypothetico-deductive method emphasising perhaps the role of deduction in testing scientific hypothesis, or (rarely!) the nomologico-deductive method, touching upon the role of deduction in the application of established laws.

Despite its departure from Aristotle, there may perhaps still be seen an element of universalism here.

\subsection{Leibnizian Universalism}

By contrast with Bacon, Leibniz retained an enthusiasm for Aristotle's logic.
By contrast with Aristotle, Leibniz was explicit about his universalistic ideas, which however remained primarily based around Aristotle's Syllogistic.
Among other aspects of his universalism, he enunciated the ideas of a \emph{Lingua Characteristica} in which all scientific knowledge could be expressed in a way amenable to automated computation determining the truth or falsity of any conjecture, the method for which was to be captured in his \emph{calculus ratiocinator}\cite{couturat1901logique,Peckhaus2004,Lenzen2018}.

Notwithstanding his enthusiasm for a universalistic perspective, Leibniz's ideas were not realisable.
A first reason for this was the inadequacies of Aristotle's logic, which fell well short of the needs of mathematics, let alone science, coupled with Aristotle's failure so see that inadequacy and contemplate radical advancements.
Another group objections would become clearer as mathematical logic began to characterise the limits of logical methods, so that no-one now believes that scientific truth is decidable, as was supposed implicitly in Leibniz's conception of the \emph{Calculus Ratiocinator}.

Nevertheless, the ideas and aspirations of Leibniz were to prove inspirational for many of the principle figures in the emergence of modern logic and of the universalist conception with which it might be said to have begun.

\subsection{Frege's Begriffsschrift and his Universalism}

It is in the nineteenth century that the slide in standards of rigour in mathematics from its zenith in Euclid reached a crisis point and provoked an extended period of foundational repair ultimately reaching a complete transformation of formal logic.
A central concern of that crisis was the number system required for the differential and integral calculus, independently devised by both Newton and Leibniz, and a broad filed of techniques in the development of mathematical analysis flowing from them and stimulated by applications in science, technology and engineering.

Before this, in the 18th Century, the philosopher David Hume had made the distinction between ``matters of fact'' and relations between ideas'' in which the former were empirical claims about the material world, and the latter both logical and mathematical truths, taking mathematics thus to be a part of logic.
Immanual Kant made a particular feature in his philosophy of rejecting Hume's simple picture effectively denying that the theory of natural numbers could be derive logically from an appropriate definition of the numbers.

As the rigorisation of mathematics proceeded during the 19th Century, mathematical analysis was recast avoiding infinitesimal numbers in what became known as the real numbers, which were then defined in terms of cuts in the rational numbers, themselves constructed from the positive whole numbers (`natural numbers').
At this point philosophy takes over, in the person of Gottlob Frege, who was motivated to refute the ideas of Kant in relation to the status of mathematics, by deriving mathematics logically from definitions of the natural numbers.

For this purpose he created a new general purpose logical notation and deductive system which he called \emph{Begriffsschrift}\cite{heijenoort67} (`concept notation').
This simple logical system at last overcame the severe limitations of Aristotle's syllgistic, and provided a logical basis for the derivation of mathematics.
In doing so it provided an account of what constituted a sound deductive inference, thus passing beyond the standards of Euclidean axiomatic methods by systematising not only the principles from which deduction might proceed, but also the criteria for the deductive reasoning used to derive results from those principles.

The univeralist conception of Frege's logic is captured in his maxim:

\begin{quote}Mathematics = Logic + Definitions\end{quote}

  This later became known also as the doctrine of \emph{logicism} as an account of the nature of mathematics, a conception which would later be to be progressed by philosophers such as Bertrand Russell and Rudolf Carnap.
  Frege moved forward from his Begriffsschrift in 1979 to the formalisation of mathematics in his \emph{Grundgesetze der Arithmetic}\cite{frege93}, of which Volume 1 was studied by Bertrand Russel in the run up to publication of the second volume.

  In it, Russell found that the system admitted the derivation of logical contradictions through its unlimited ability to form seta by abstraction, which was already known to be among the difficulties of Cantor's set theory.
  This became known as \emph{Russell's Paradox} and prompted an acknowledgement by Frege that the system on which he had based his \emph{magnum opus} was compromised.

  \subsection{Hilbert's Pluralism}

  Toward the end of the 19th Century the mathematician David Hilbert, who would later be important in the development of mathematical logic, took an interest in axiomatic geometry \cite{hilbertFG}, and we find in his work on this another approach to reinstatement and  advancement beyond the rigour of Euclid's methods.

  Hilbert is regarded as the main proponent of those mathematical methods which have been contrasted with the universalism of Frege and Russell and are thus described as a form of logical pluralism.
  Curiously this does not involve a variety of conceptions of logic, but rather the idea that mathematics can be conducted in its various branches by adopting axioms characterising the mathematical entities under consideration and different axiom systems deriving the resulting theories, and accepting all the various interpretations of those axions.
  The existence of mathematical entities, according to Hilbert, depended on nothing more than the logical consistency of the axioms.

  \subsection{Russell's Logicism}
  
The contradictions found in Frege's foundations were a serious problem not only for Frege, but also for Russell, for he had anticipated following his own \emph{Principles of Mathematics}\cite{russellPRM} by a further volume with a purpose similar to Frege's.
Though Frege never came back from this disappointment with an amended system which was more robust, Russell continued, with great difficulty devising an alternative logical type theory which was ultimately published some eight years later \cite{russell1908} and became the basis for his joint work with A.N.Whitehead on \emph{Principia Mathematica}\cite{russell10} in which Russell's logicist conception of mathematics was supported by the derivation of large parts of mathematics in Russell's univerally conceived Theory of Types.

The challenge which Russell faced ws to create a foundation system in which mathematics could be formally deduced from definitions of the concepts of mathematics, without making arbitrary choices for the sake of avoiding paradox, but rather making choices informed by coherent philosophical rationale.
He wanted to correct perceptible flaws in Frege's system, rather than find the easiest fix.

From the discovery of ``Russell's Paradox''in Frege's system it took Russell 8 years to publish his resolution \cite{russell08}, and regrettably, despite his best intentions, it was not wholly satisfactory either from a technical or a philosophical perspective.
Nevertheless, it did provide a viable basis for the massive project of formalisation which he then undertook with the aid of A.N.Whitehead.
\emph{Principia Mathematica} \cite{russell10} as it was called (after Newton's work bearing the same name, was to prove very influential over the next few decades.

This is a significant stake in the ground for this project.
The main foundation system proposed in this monograph for practical use in knowledge representation is a direct descendent of Russell's Theory of Types, and one way to approach an understanding of the system is through an acquaintance with its evolution from this point.

\subsection{Zermelo's Set Theory}

At the same time as Russell published his Theory of Types, an alternative with very different characteristics, but also inspired by the foundational developments in 19th Century Mathematics, was being published alongside it.
In a 1908 paper, Zermelo\cite{zermelo08} provides an axiomatic basis for set theory that formalises and extends the foundational approach to mathematics first explored by Dedekind in 1988\cite{dedekind1888}, where sets and their properties were used to define the natural numbers and establish a rigorous basis for arithmetic.

A merit of this foundational approach is its ontological transparency and simplicity, based as it was on the hierarchy of well-founded sets which may be informally defined using the inductive definition ``a set is any definite collection of sets'', conjuring up a heirarchy of sets obtained by starting with nothing and then in an orderly way forming sets from any combination of sets which has already been formed.

\cite{dedekind1888,zermelo08,mirimanoff1917,fraenkel1922,skolem1923,vonNeumann1923,vonNeumann1925}

\subsection{Rudolf Carnap}

  Rudolf Carnap was introduced early in his studies to the work of Frege and later became acquainted with and inspired also by Russell.
Carnap seems to have been pluralistic from his student days, before he acknowledges any influence from Frege, particularly in relation to ontological presumptions which featured in the philosophy of science, and also sought to bridge philosophy and science in his doctoral dissertation.
  
  He mentions Russell in his `Intellectual Autobiography'\cite{carnap63a,carnap63} as having inspired his own desire to extend the new logical methods beyond mathematics and into science.
  But he was acutely aware of the distinction between logical truth and empirical truth (which he spoke of as the analytic/synthetic distinction), and felt therefore that the derivation of empirical science could not be undertaken in a purely logical system.

    He was later influenced by the work of Hilbert and adopted a more pluralistic conception of logic than Frege and Russell, which he  saw as dictated by the extension of logic to address empirical matters.
  He was neverthess a logicist, holding with Frege and Russell that mathematics fell under Frege's conception as logically derivable from the definitions of mathematical concepts.
  The departure from purely logical truth was to be accomplished for empirical science by the addition of empirical principles to each scientific domain, from which the details of the behaviours in that domain could be logically derived.
  His pluralism was explicit and is most conspicuous in his volume on the logical syntax of language \cite{carnap37}.

  \subsection{Alonzo Church}

  \subsection{Dana Scott and The Verification of Software}

  \subsection{Michael Gordon and The Verification of Hardware}

  
  
  \section{Problems with Logical Universalism}

  The contradiction in Frege's system revealed the first difficulty with the universalistic conception of logic, for it became apparent that explicit ontological choices had to establish a system in which mathematics could be derived from the definitions of mathematical concepts, and it was not clear that logic alone provided a basis for these choices.
  Questions about ontology, even if exclusively concerned with the purely abstract entities studied in mathematics.
  This was however a problem mainly for the logicist conception of mathematics rather than a pluralistic conception of logic, and mathematical logicians were later to attach particular importance to one particular logical system as definitive of pure logical truth, which we now call first order logic.

  Though it may be thought of as a single logical system, it is clear from typical presentations of this logic, that it is thought of as pluralistic, based on the concept of a \emph{first order language} in which the vocabulary of non-logical symbols is fixed.
  The idea that one has a single logic and proceed thence by definition is lost.

  What these ruminations reveal is a fluidity in the distinction, which makes it questionable how fruitful this distinction is.

  \section{Some Resolutions in Carnap's Pluralism and Logicism}

  

  \section{Pluralistic Universalism}
  


\resetnotes

\ignore{

\chapter{Synthetic Epistemology and Foundational Abstract Semantics}


\section{Evolution and Epistemology}

To give a first account of the way in which evolution and epistemology are entwined in this narrative, I must provide characterisations of those two concepts, both of which I take in a broad sense sympathetic to the ideas whose exposition they abet.

\subsection{What is Evolution?}

The term evolution is used here for any process of progressive or incremental change whose long term effects are realised through some kind of differential proliferation.
`Differential prolifearation' occurs when certain kinds of entities proliferate (multiply and/or disperse, replication admitting some degree of variation) at rates which vary according to their particular characteristics.
The basic principle here is that those kinds of entity which proliferate most profusely within some environmental niche will come to numerically dominate in that niche.
Those small changes which improve proliferation are considered `adaptive', and over time these small changes may yield the major transformations which are seen (for example) in the evolution of species.

However, this conception of evolution lacks some of the characteristics usually seen in the evolution of species.
In the classical conception, the variations which are essential to evolutionary progress are supposed random, and proliferation is moderated by `natural selection' which determines which variants prove most prolific.
These characteristics may be helpful in advertising a natural process of evolution which does not depend on divine intervention, and may be accurate descriptions of biological evolution, but are not found in all the kinds of evolution which are of interest here.
Prebiotic evolution, cultural evolution, or the kinds of evolution which may yet emerge as artificial intelligence and synthetic biology mature to dominate evolution in the future.

\subsection{What is Epistemology?}

Though talking about knowledge is nearly as old as language itself, epistemology, the philosophical theory of knowledge, probably begins with Plato, who spoke of knowledge as \emph{justified true belief}.
Insofar as we may infer the meaning of the term from its usage, which is diverse, this is a narrow characterisation.
It is psychologistic, anthropocentric, and addresses only the kind of rigorous knowledge we associate with science, to which philosophy often aspires.

The kind of epistemology in which this monograph engages I call \emph{synthetic}, and it yeilds a synthesis or construction of a conception of knowledge primarily defined by the manner in which knowledge is \emph{represented}, diversifying the metrics evaluating supposed knowledge away from `justifiction' to admit criteria appropriate to the full diversity of knowledge, including, for example, procedural  knowledge, or \emph{knowing how}.

The epistemology here is \emph{synthetic}, which can be understood by contrast with three other approaches to epistmology which are:
\begin{itemize}
\item[analytic]

  This is the kind of philosophy which might be most appropriate to the analytic tradition in philosophy, particularly its most recent manifestations in the early to mid twentieth century.
  It takes language \emph{as it is}, perhaps even with an emphasis on \emph{ordinary language} and enquires what terms like `know' and `knowledge' mean in that established usage.
  It may be noted that in talking about `analysis' here, we are primarily speaking of analysis of language rather than logical analysis, and that this process properly yeilds synthetic truths about a natural phenomenon rather than logical truths.
\item[prescriptive]
  epistemology becomes prescriptive when it is concerned to determine how the relevant language \emph{should} be used rather than how it \emph{is} used.
  \emph{natural}
  Natural epistemology, or epistemology naturalised, is a kind of epistemology beginning later in the twentieth century, initiated by W.V.Quine \cite{quine1969epistemology}, in which knowledge is considered a natural phenomenon and should be persued by the methods of empirical science.
\end{itemize}

\resetnotes

\chapter{The Evolution of Declarative Knowledge}\label{EDK}

The evolution of semantics is a slender thread which is pivotal to the epistemological theses which underpin this synthesis.

There are many ways to give structure to the four billion years of evolution here on earth.
The largest scale structure of interest here concerns the evolution of intelligence, which has happened (I suggest) twice, in wholly different ways.

The first time it was biological evolution.
The Darwinian evolution of species \cite{darwin-oos}, which took life on earth from single celled prokaryotes all the way to the species of genus \emph{homo}, of which more than one may have been `intelligent', but only one now remains, \emph{homo sapiens}.

Intelligence comes in degrees and varieties.
In using the term in a black and white way here I am adopting, for present purposes, the criterion that intelligence `proper' is the ability to engage in collaborative design and construction sufficiently well to ultimately engineer intelligent artifacts.

The second time, the evolution which created intelligence was cultural, and not very Darwinian, involving intelligent design as a source of variation, and intelligent selection rather than natural selection (we may debate where the rather artificial boundary between the two might lie).
This second evolution of intelligence is not quite yet complete, for by my chosen criteria of intelligence, it will only be complete when we have artificial intelligence capable of engineering new generations of intelligence.

This division of evolution into two stages of similar significance bbut very different duration (Billions of years against hundred thousands), is significant from the point of view of semantics, and the evolution of semantics exposes developments which have been crucial to the re-invention of intelligence.

The second phase, in which cultural evolution takes the lead, is enabled by the evolution of oral language, and thenceforth the continuing accelerating evolution of way of representing, communicating, storing and exploiting knowledge which have been effective in part because of their approximation to an ideal which the evolution of semantics has only very recently permitted to be clearly articulated.



\resetnotes

\chapter{A Philosophical Kernel 2}

The proposed system for the representation of knowledge is \emph{foundational}, it is a system to which other ways of representing knowledge are reducible in some sense (to be described).
It is therefore necessary in its articulation to speak in general of knowledge representation systems and of the idea of reduction relative to which the proposed system may be seen as universal in a broad class of such systems.

Though the proposal is \emph{epistemological}, its description depends upon \emph{metaphysics} (particularly, \emph{ontology}), \emph{philosophy of language}, and \emph{logic}, those four aspects of philosophy combining to articulate the most fundamemtal parts of the system.

\emph{Foundationalism} has been regarded as a failed doctrine by many philosophers for a good while, but this may well be because they consider only a straw man, in which foundational theses always make absolute claims about the most fundamental parts of their systems.
This I will not do, I aknowledge that neither in relation to meaning nor verification can one ever have absolute precision or certainty.
One can however, in certain domains which I will present as of particular importance, come closer to those ideals than any practical purpose demands.
Foundations accomplish that end, but they do so from a particular philosophical perspective, so technical adequacy will not necessarily by universally convincing.

The foundational problem in relation to both meaning and truth is how to terminate infinite regress in definition or justification, and this may be presented as a choice between finding a foundation which is self-evidently clear and conclusive, or rendering the foundation precise through the use of language ultimately to be defined in terms of it.
In this proposal, this choice is rejected in favour of doing both.

The foundational enterprise can be appreciated as the threading of language through the eye of a needle.
The whole of complex languages are to be defined in terms of very simple primitives, which are definable using a tiny part of the languages which can then be constructed upon them.

\section{Epistemological Abstraction}

Epistemology has often been in significant measure influenced by the world around us, and the ways in which human beings are built and acquire and deploy knowledge of themselves and the world around them.
It may also be influenced by or intimately concerned with the language with which we talk about knowledge, not least the meaning of the word ``know''.

The approach here, which we may think of as an approach to `abstract epistemology', seeks to minimise the extent to which epistemology reflects such earthly or anthropomorphic influnces.
How could such an epistmology arise, what would be the purpose of attempting to construct such an abstract epistemology, and how could it possibly succeed?

This moment in the evolution of intelligence provides a context in which this might be understood.
We stand, as I write, at a point at which many of the hallmarks of intelligence in humans are now to be seen in computational artifacts.
We are also venturing into synthetic biology which may transform the evolution of biological intelligent systems, and the ambition to send intelligent systems across the solar system and into the star systems beyond is on the ascendent.
It is likely that the promulgation of intelligence across the galaxy will ultimately be predominantly of non-biological intelligence, and that even that central core where biological life has penetrated will be populated by species well advanced beyond homo sapiens, growing progressively more distant from earth and homo sapiens.
These are among the motivations to consider epistemology in ways which stand back both from human language about knowledge and human ways of acquiring knowledge.

Pure mathematics provides examples of structures, knowledge of which will surely be universal among intelligence wherever it is found.
The natural numbers, those numbers which we use for counting discrete entities, are a simple example.
Abstraction to is the business of pure mathematics, in contact with an alien civilisation, it may be the mathematicians who would best suceed in communicating with their alien counterparts.

\section{Foundational Metaphysics}

The abstract foundation for epistemology here envisaged is a story couched in terms of abstract entities.
So what are they, and what abstract entities are there?

The distinction between abstract and concrete is not completely clean, since we can construct abstract entities with concrete constituents.
We are here concerned with purely abstract entities, and adopt a conventionalist position in relation to such entities.
The question what abstract entities there are is therefore to be determined by context, an aspect of context which might (or might not) be fixed by the language.
If it were meaningful to speak of there being absolute truths about what abstract entities exists, it would not impact this position, for the utility of \emph{chosing} a domain of abstract entities for the purpose of constructing an abstract model or for developing a mathematical theory is not dependent on what abstract entities do or do not `really exist'.

Concrete ontology is not far removed from this conventional stance, though the evaluation criteria which is makes sense to apply to concrete ontologies are more stringent.
Concrete ontology, we might expect, is primarily of utility in constructing models of the physical world, and may be subject to similar criteria.
Though philosophers have debated the validity of inductive reasoning to establish the truth of empirical generalisations, and have proposed alternatives such as continuous search for falsification, it is clear that the utility of a model of the empirical world may persist in the face of good evidence that it is literally false.
Newtons theories of motion and gravitation are the clearest examples, where accepted as false they are nevertheless more widely used than the theories which displaced them and are still thought to be `true'.

\section{Abstract Languages}



\resetnotes

\chapter{A Philosophical Kernel 3}

The philosophical context for the epistemological synthesis presented in this monograph is a foundational philosophy.
The kernel of the philosophy provides the foundations upon which the whole is based.

The kernal is epistemological, but its description involves other aspects of philosophy which may also be thought of as foundational, including metaphysics (mainly ontology), and the philosophies of language and logic, all of which are involved in providing a description of the most fundamental concept, that of \emph{abstract logical truth}.

\section{Methodological Preliminaries}

A foundational approach to some problem domain consists in discovering some smaller and/or simpler domain (the foundation) to which all the problems in the more complex domain are reducible.

The classic conception of knowledge holds that for a true belief to count as knowledge, it must be supported by a conclusive justification, from which it would follow that a foundation to which knowledge is reducible should itself be beyond doubt.
This philosophy and the kernel on which it is based, is skeptical of any absolute claims, it does not assume that it is ever possible to assign meaning to language with absolute precision, or that any sentence can be known to be true with absolute certainty.
Fortunately we do not need such absolutes, life goes on without them.
Nevertheless, there are very great differences in the precision of language and the certainty with which truth can be ascertained.
There are variations from one domain to another, and within a domain the language may evolve to greater expressiveness and precision with the benefit of experience, and the methods for ascertaining truth can often be progressed to increase their reliability and reach.

This philosophical kernel is primarily concerned with \emph{logical foundationa}, which may now have been progressed close to a limit in semantic expressiveness, certainty of proof, and completeness, though in all these absolutes are not to be expected.

Of logical foundations it is often thought that we must chose betweem a system which is not itself defined in terms of or reducible to some other, or else that circularity of definition must be admitted by defining the most simple case in itself or in some more complex system.
Since neither of these approaches is wholly satisfactory in itself, but both are likely to contribute to making the system precise and reliable, I advocate doing both. 

In case it may be thought that without achieving the relevant absolutes, the purpose of foundational thinking is abrogated, it may be helpful to mention that the approach of mathematics to logical foundations which took place primarily in the 19th Century was not motivated by a desire for absolutes, but rather by the need to make precise mathematical ideas which had become so lacking in clarity that rigorous proof of mathematical properties involving them was no longer possible.
The concept in question were those of mathematical analysis which were founded on a conception of number which included the infinitesimally small, an idea which had never been made clear.



\section{Abstract Logical Truth}

The conception of logic truth adopted here is essentially similar to that of Rodolf Carnap, for which \emph{analytic truth} was a psuedonym, and in terms of which logical necessity was also defined.
The presentation here is dissimilar to Carnap's, particularly in the way in which linguistic pluralism is addressed, and we may also say that Carnap's philosophy was not \emph{foundational} in relation to logical truth 

\resetnotes

\chapter{A Philosophical Kernel}\label{PK}

A kernel, as I use the concept here, is a core which provides essential and fundamental elements upon which a larger system of some kind is built.

This chapter presents a kernel for an epistemological philosophical system.
A key feature of this kernel is the definition of a notion of \emph{logical truth} which is considered foundational for knowledge and epistemology.
In preparation for promoting a particular formalisation of that conception of logical truth, I describe an ordering on logical systems according to their expressiveness, relative to which a maximally expressive system will be sought.

In the next chapter I will identify a universal family of logical systems and a \emph{logical} kernel for that family.

The first, which is the subject of this chapter, is the Kernel or fundamental core of a philosophical system within the Western tradition which is generally considered to have begun with the philosophers of Classical Greece.

In that early beginning, starting around 600 BC with Thales, and progressing through the so called `pre-socratics' to the great intellectual achievements of Plato and Aristotle, philosophy was the love of knowledge, 

The philosophy has a kernel because it is an example of `first philosophy', an idea introduced by Aristotle in his \emph{Metaphysics} \cite{aristotleMetap}.


\resetnotes

\chapter{The Fundamental Triple Trichotomy}\label{FTT}

Consideration of the domain in which extended deductive reasoning can be made safe has lead to a focus on purely abstract logical truths.
In this chapter I will talk about other important domains of discourse and how logical truths can enable deductive reasoning in those domains.

\section{Hume's Forks}\label{HF}

The triple-trichotomy is and elaboration on a theme fundmnetal to the philosophy of David Hume, who gave a central place to the dstinction which became known as `Hume's Fork' 


David Hume was a philosopher of the Scottish Enlightenment.
The enlightenment was a period of ascendency in the place of reason in
the discussion of human affairs, when science had secured its
independence from the authority of church and state and had a new
confidence in its powers occasioned particularly by the successes of
Newtonian physics.

Hume looked upon the philosophical writings of his contemporaries and
found in them two principle kinds, an ``easy'' kind which appealed to
the sentiments of the reader, and a ``hard'' kind which trawled deeper
and appealed to reason.
This latter kind, ``commonly called'' metaphysics, was preferred by
Hume, but found nevertheless, by him, to be lacking, infested with religious 
fears and prejudices.
Hume's feelings about these aspects of philosophy were not vague
misgivings.
He had a specific epistemological criterion which he saw these
philosophical doctrines as violating.

Hume's project involves an enquiry into the nature of human reason for
the purpose of eliminating those parts of metaphysics which go beyond
the limits of knowledge, and establishing a new metaphysics on a
solid foundation limited to those matters which fall within the scope
of human understanding.

David Hume wrote his philosophical {\it magnum opus}, {\it A Treatise on
  Human Nature} \cite{hume39} as a young man.
He was disappointed to find his work largely ignored and otherwise
misunderstood, and thought perhaps that his presentation had been at
fault.
To improve matters he wrote a shorter work more tightly focussed
on the core messages which he thought of greatest importance.
This he called {\it An Enquiry into Human Understanding}
\cite{hume48}.

In a central place both logically and physically in this more concise
account of his philosophy he says:

\begin{quote}
``ALL the objects of human reason or enquiry may naturally be divided
  into two kinds, to wit, Relations of Ideas, and Matters of Fact.'' 
\end{quote}

We shall see that Hume is here identifying a single dichotomy which
corresponds to all three of the distinctions which here concern us.
In his next two paragraphs he expands in turn on the kinds he has thus
introduced.

\subsection{Relations of Ideas}

\begin{quote}
``Of the first kind are the sciences of Geometry, Algebra, and
Arithmetic; and in short, every affirmation which is either
intuitively or demonstratively certain.
That the square of the hypotenuse is equal to the square of the two
sides, is a proposition which expresses a relation between these
figures.
That three times five is equal to the half of thirty, expresses a
relation between these numbers.
Propositions of this kind are discoverable by the mere operation of
thought, without dependence on what is anywhere existent in the
universe.
Though there never were a circle or triangle in nature, the truths
demonstrated by Euclid would for ever retain their certainty and
evidence.''
\end{quote}

Hume is distinctive here among empiricist philosophers in having a
broad conception of the {\it a priori} (though he does not use that
term here), allowing that the whole of mathematics is {\it a priori} (a concept encapsulated in Hume by the phrase ``discoverable by the mere operation of thought ...'').
In this he may be contrasted, for example, with Locke who allowed
only certain rather trivial logical truths to be knowable {\it a
  priori}.
Nevertheless, Hume's conception of the {\it a priori} remains narrow
by comparison with the rationalists, and in particular, as Hume will
later emphasize, excludes metaphysics.

\subsection{Matters of Fact}

\begin{quote}
``Matters of fact, which are the second objects of human reason, are not ascertained in the same manner; nor is our evidence of their truth, however great, of a like nature with the foregoing. The contrary of every matter of fact is still possible; because it can never imply a contradiction, and is conceived by the mind with the same facility and distinctness, as if ever so conformable to reality. That the sun will not rise to-morrow is no less intelligible a proposition, and implies no more contradiction than the affirmation, that it will rise. We should in vain, therefore, attempt to demonstrate its falsehood. Were it demonstratively false, it would imply a contradiction, and could never be distinctly conceived by the mind.''
\end{quote}

The evolution of the following three dichotomies is the theme of this chapter, though we will find other related dichotomies which feature in the history.

The terms which I will use to speak of them, in this chapter are:
\begin{itemize}
\item{analytic/\-synthetic}
\item{necessary/contingent}
\item{a priori/a posteriori}
\end{itemize}

As I shall use these terms these are divisions of different kinds of
entity, by different means.
For that reason they cannot be said to be identical, and their extents clearly depend upon exactly how the relevant technical concepts are defined, but with suitable and reasonable definitions, these dichotomies prove to be very closely coupled.

The first is a division of sentences, understood in sufficient context
to have a definite meaning, and is a division dependent upon that
meaning.
Meaning is not an univocal term, it is particularly uncertain in meaning,
But in this context the requirement is very specific, only one possible component of meaning is required, which is the truth conditions of the sentence.
The truth conditions of a sentence are an assignment of truth values for he sentence in every possible condition (what is a `condition' depends on the language, for natural languages a condition would include both a state of the world and sufficient context to disambiguate the sentence if its meaning is in any way context sensitive). 

The second is a division of {\it propositions}, which may be
understood for present purposes as {\it meanings} of sentences in
context.
What a proposition is need not be settled except that it must include, again, truth conditions, the context of any sentence having been settled to determine the proposition which it expresses.
Such a proposition is considered necessary if it is true in every relevant circumstance, the range of circumstances having been fixed by the language and constituting the subject matter of the language.
The concepts of analyticity and necessity are, by this kind of definition, logically related.
A sentence is analytic if once disambiguated by appropriate context, it is seen to exppress a necessary proposition.
These exact definitions are not to be found in Hume, and are not philosophically uncontroversial, but are presented here so that we can speak of that position adopted by some later philosophers which Hume may be thought to have anticipated.

The division is made according to whether the proposition
expressed must under all circumstances have the same truth value, or
whether its truth value varies according to circumstance.
In this we are concerned with two particular notions of necessity,
those of logical and of metaphysical necessity, the latter being
sometimes taken to be broader than the former.
A part of the role of Hume's fork in positivist philosophy is to
banish metaphysical necessity insofar as this goes beyond logical necessity.

The third is for our purposes also a division of propositions, on a
different basis.
It concerns the status of claims or of supposed knowledge of
propositions.
It is expected that such a claim must in some way be
{\it justified} if we are to accept it, and that the kind of
justification required depends upon the proposition to be justified.
The justification is \emph{a priori} if it makes no reference to observations
about the state of the world, i.e. to sensory observations or other results obtained on the basis of such evidence.
The distinction in this case may be described as an epistemic distinction, since it concerns what kind of justification we may expect for the kind of proposition in question.

The suggested identity between the first two concepts has a pale reflection in relation to this epistemic distinction.
In this case we do not assert an identity but rather recomend that an \emph{a priori} justification be required for necessary propositions, and that an \emph{e posteriori} justification be required for contingent propositions, thus closely if indirectly coupling the three dichotomies.

\subsection{The Place of The Fork in Hume's Philosophy}

The mere statement of the fork (which we shall see, is not original in
Hume) is of lesser significance than the role which it plays in Hume's
philosophy, which serves to clarify the distinction at stake and draw
out its significance.

Hume's philosophy, like Descartes' comes in two parts of which the
first is sceptical in character, and the second constructive.
In both cases the sceptical part clears the ground for a new approach
to philosophy which is then adopted in the constructive phase.

For our present purposes we are concerned principally With the first
sceptical phase of Hume's philosophy, because of the delineation of the scope
of deductive reason, and hence of the analytic/\-synthetic dichotomy
which is found in Hume's sceptical arguments.
This delineation is baldly stated in Hume's first description of the
distinction between ``relations between ideas'' and ``matters of
fact'', for there Hume tells us that no matter of fact is
demonstrable.

This bald statement would by itself have little persuasive force if it
were not followed up with more detail, even though ultimately this
detail does not so much underpin the distinction as depend upon it.

Hume's further discussion begins with the consideration of what
matters of fact can be known `beyond the present testimony of 
our senses or the records of our memory'.
The inference beyond this immediate data is invariable causal, we
infer from the sensory impressions or memories to the supposed causes
of those impressions.
But these are not logical inferences, causal necessity is for Hume no
necessity at all (even less the inference from effect to cause).
Hume's central thesis that matters of fact are not demonstrable is
in this way reduced first to the logical independence of cause and
effect, and then to the distinction between deductive (and hence sound)
inference and inductive inference (whereby we infer causal
regularities and their consequences).

Given that Hume considers all inferences from senses to be based on
induction, and sees no validity in causal inference, it follows that
from information provided directly to us by the senses nothing further
can be deduced which is not simply a restatement, selection or summary
of the information itself.
Further enlightenment from this sceptical doctrine is primarily the
application of this principle to various kinds of knowledge.
In the process Hume does a certain amount of 



\section{The Triple Trichotomy}

The `triple trichotomy' is a presentation of these extended domains of reasoning as a two dimensional matrix, one dimension associated with the distinct domains, the other with different kinds of characteristics which they posess.

\begin{table}[h]
    \centering
    \begin{tabular}{l | l | l | l}
         & \rotatebox{45}{Semantics} & \rotatebox{45}{Evaluation} & \rotatebox{45}{Modality} \\
        \midrule
        Spiritual & non-natural  & propriety & normative \\
        \midrule
        Empirical & concrete & utility & contingent \\
        \midrule
       Logical & abstract & proof & necessary \\
    \end{tabular}
    \caption{The Fundamental Triple Trichotomy}
    \label{tab:example}
\end{table}

In the above table, abstract semantics for logical truths (which are to be established by deductive proof and are necessary rather than contingent, or normative). is shown in the lower left corner as befits concepts which are regarded as foundational to the whole.

I will talk through this table in the sections which follow, working from bottom up and left acrosss.

\section{Semantics}

David Hume's philosophy provides us with a first view of the three domains, through two important distinctions.
The first is what later became known as the \emph{analytic/synthetic} distinction, that described by Hume as `relations between ideas' and `matters of fact'.
Because Hume talks of it in terms of subject matter, this is readily understood as a semantic distinction.
Hume also considers both those two categories as concerned with what \emph{is}, rather than what \emph{ought to be}, and says that these normative claims are not derivable from mere descriptions, one cannot derive an `ought' from an `is', giving us three distinct categories each logically beyond its predecessors.

So how can the abstract semantics which yields logical truth be foundational for these domains which seem logically beyond it?
It is foundational because we can mimic the structure of these extended domains in pure abstractions, and then describe map the abstract ontology onto the concrete entities and the normative concepts.

\section{Evaluation}

In the classical conception of knowledge as justified true belief, we see that to establish something as knowledge we must be able to show conclusively that it is true.



\resetnotes
}%ignore

\appendix
\ignore{
\chapter{The Philosophy of Rudolf Carnap}\label{PRC}

The philosophy of Rudolf Carnap is the closest predecessor I know to the position described in this momograph, but is often misrepresented or misunderstood.
For those who are acquainted with his philosophy, my own position might be easier to understand if juxtaposed and contrasted with my understanding of Carnap (whether or not that understanding is sound), and for others a sketch of his philosophy with some minimal context might nevertheless be helpful.

I am not a scholarly student of Carnap's philosophy, and only came to appreciate its relevance to my own thinking fairly late, partly because what I now see as the main thrust of his philosophy, as clearly outlined in his own intellectual autobiography \cite{carnap1963}, seems not to have been much spoken of by his critics, who have too often focussed on features of his philosophy which were either misrepresented (such as, that he was a phenomenalist) or transient (the verification principle \cite{carnap1937}, his syntactic phase).

The following account leans towards his work on logical analysis and neglects his work on confirmation theory and testing, because my own thinking and this monograph contribute little to those areas.
It pivots around his work on logical syntax \cite{carnap1935}, partly because at this time his lectures in London on that topic \cite{carnap1937} give a concise survey of the main aspects of his philosophy at that time, and both his earlier ideas, as described for example in his intellectual autobiography \cite{carnap1963} and his mature philosophy are both covered well in the Schilpp volume of the Library if Living Philosophers \cite{carnap63a}.
The other important works \cite{carnap1956,carnap1950,carnap1990}  relate to his semantic phase and the criticism and repudiation of his philosophy by W.V.~Quine.

\resetnotes
}%ignore

\markdownInput{CONTRIBUTING.md}

\chapter{Instructions for AI Contributors}\label{InstructionsForAIContributors}

I have written this monograph with copious assistance from Grok 3, an AI chatbot from xAI.

Notwithstanding the invaluable contribution Grok has made, the writing is all mine.

The following section contains the general instructions which I have used to brief Grok.
Naturally, these instructions have evolved over the course of the writing and this their state at the end.
Different stages in the process demand different instructions, and I have tried to keep instructions of a general character  in a separate file (presented below), with supplementary files giving more specific asks for particular stages.

\section{General Guidance for AI Contributions}

My principle intellectual ambitions are now primarily philosophical, with a generous and flexible sense of the scope and methods of philosophy, but, nevertheless, a focus on theoretical foundations.

I aim to practice philosophy by producing written accounts of the ideas I am working on, which are oriented toward philosophical support for the effective application of modern advances in logic.
I now think of these as monographs, but expect that this may change as I learn to work effectively with Grok.
These notes are therefore written for Grok, and I will couch them in first and second person language.

I have decided that the best way to deal with the readership targeting is to write for an audience of two, you, Grok and me, Roger Jones.
I want to produce in the first instance a monograph which seems to me to adequately capture the ideas I had hoped to express, and which is clear enough for you to get a fairly deep understanding of them.
I don't yet have any well thought out ideas about how to check your understanding, so the first idea is conversation.
Maybe I will use the common textbook practice of including questions at the end of each chapter for you to answer and I can then review your answers with you.

Once we have a manuscript which I think is a good statement and which you understand well, I will ask you to prepare (with my help) materials for a variety of other target audiences and other channels of communication, which I hope will include making the relevant ideas available via a grok3 conversation (which will be possible if in no other way, but attaching the manuscipt to a grok session, though I will be hoping for a way of gathering feedback from these sessions which will require something more sophisticated, probably supported by future developments to Grok in X about which I can only speculate),

As to channels, the first product will be a PDF suitable for distribution electronically or printing on demand.
This could be converted into an ebook in multiple formats.
I am inclined to think of X as a main channel for promulgation, and imagine that you would be able to support a bot account on X (in due course) which would explain the ideas.
I also anticipate serialising the monograph on substack, and would want you to convert the chapters one at a time into posts for substack.
You may have some ideas on what other channels might be good ways of promulgating the work.

Given the breadth of the considerations, particularly the divide between evolutionary thinking, philosophy, logic and cognitive science/engineering, I would like it to be possible for those who interest and/or competence in only some of these areas to find something of interest even though perhaps skimming of omitting parts of the document outside their competence or interest.
So I shall be constantly trying to sketch informally in introductory parts and concetrate detail and depth into separate chapters or sections.

You are to act as my research assistant and reviewer of the developing work, but you will not be contributing directly to the writing of the first manuscript, i.e. the words will be mine, though I hope and expect that your indirect contribution will be substantial.
When it comes to derived works, the ideas will still be mine (though refined in discussion with you), but the words will be yours.

In all responses concision is very important.
Prolixity will impede my progress rather than advance it, you will need to cultivate a conservative sense of what is relevant and important.
Note also, that I am not looking for creative suggestions (unless I should explicitly ask for some).
Beyond fact checking I am looking to kmow whether what I am writing is intelligible.

Your general knowledge of all the matters touched on in the exposition is superior to mine, except in those parts where the monograph presents original material, and my first hope is that you will fact check everything which I write and let me know if there are any factual errors.

My second hope is that you will mention to me other work which I ought to be aware of because of its proximity to the subject matters I am addressing.
In this I am not asking you to compile a list of those things which seem most relevant even if tenuously 

The third desideratum is that you take cognisance of the intended or likely readership and advise me on matters addressed which may be dificult for that audience to comprehend, and perhaps make suggestions as to how improvements can be made.
Similarly with any matters with which they may seem likely to disagree, particularly if their disagreement is soundly based.

When I attach a document, or re-attach one, I would like you to automatically read the document.
This will usually be the current draft manuscript, which includes as an appendix these notes.
I will also be including in the body of the monograph notes specifically for you to understand what I am doing and where I am going, for which I will wrap in a tex command which can hide them in the final manuscript.
These sections of the document will be in quote blocks marked ``Note for Grok:''.






\listoftables
%\listoffigures

%\phantomsection
\addcontentsline{toc}{section}{Bibliography}
\bibliographystyle{rbjfmu}
\bibliography{rbj3}

%\phantomsection
\renewcommand{\indexname}{Index of Defined Terms}
\addcontentsline{toc}{section}{Index of Defined Terms}
{\twocolumn[]
{\small\printindex}}

%\vfill

%\tiny{
%Started 2024-10-19
%}%tiny


\end{document}

% LocalWords:
